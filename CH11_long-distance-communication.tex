\chapter{Long-distance communication}
\label{sec:11_long-distance}

This chapter covers the challenges that we face when trying to communicate over long distances.
We focus on the main sources of noise and error that affect classical communication, and discuss ways of mitigating their deleterious effects.
Finally, we consider how these sources of error affect quantum communication and discover that classical mitigation techniques are not suitable for use in quantum networking.



%%%%%%%%%%%%%%%%%%%%%%
\section{Introduction}
\label{sec:ld-intro}
%%%%%%%%%%%%%%%%%%%%%%

Let's go back to the year 1852 and consider some of the historical background of long-distance communication.
\begin{table}
    \setcellgapes{3pt}
    \renewcommand\theadfont{}
    \makegapedcells
    \centering
    
    \begin{tabular}{cc}
    \hline
    \textbf{From London to} & \textbf{Days} \\
    \hline
    \textbf{New York City} & 12 \\
    \textbf{Bombay} & 33 \\
    \textbf{Singapore} & 45 \\
    \textbf{Sydney} & 73 \\
    \hline
    \end{tabular}
    \caption[Communication before the telegraph.]{Duration a letter took to deliver from London the destination city. Prior to the invention and deployment of the telegraph, long-distance communication was very wslow.}
    \label{tab:london-letter}
\end{table}
Just before the widespread deployment of the telegraph, communication was extremely slow. To give you some idea, a letter sent from London to New York City took around twelve days to arrive, as shown in Tab.~\ref{tab:london-letter}. Delivering the letter to Sydney took staggering seventy three days. 
A faster method was needed. 

Long-distance communication became considerably faster with the invention of the electric telegraph.
The first long-distance demonstration of the power of electric telegraphy was performed by Samuel Morse and Alfred Vail in 1844.
The telegram was sent from the Capitol in Washington, D.C. to Baltimore, a distance of 44 kilometers.
The virtually instantaneous communication over this formidable distance made it clear that in order to achieve fast-speed communication at global scale, submarine cables carrying electric signals were be needed.
In 1850, the first submarine telegraph cable was laid across the English Channel, connecting England and France.
This demonstrated the feasibility of telegraphic communication using submarine cables, and sparked serious interest from wealthy industrialists in funding the laying of transoceanic telegraph cables.
For the next fifteen years various telegraph companies made numerous attempts at laying a functional transatlantic cable.
In 1858, one such attempt was successful enough for Queen Victoria of the United Kingdom and President Buchanan of the United States to exchange a message. 
However, it was not until 1866 that the first truly successful transatlantic cable was laid and operated.
Despite the enormous challenges, by 1871, all of the continents except Antarctica were connected by telegraphic cables.
Between 1902 and 1906, the first transpacific cables were laid, connecting mainland US with Hawaii, Guam, later the Philippines, and finally Japan~\footnote{rdv finds it an interesting historical anecdote that one of Japan's first undersea cables came ashore in the seaside town of Kamakura, where he lives, in 1931.  It connected to Midway, Hawaii, then the continental U.S. It was in use for around a dozen years.}.

The telegraphic cables were eventually phased out and were replaced by telephone cables.
The first transatlantic telecommunications cable, named TAT-1, was laid in 1956.
This coaxial telephone cable was initially able to handle 36 concurrent telephone conversations, which was later upgraded to 72 ocncurrent conversations. 
Compare that with the state two decades later, when the last of the coaxial copper telephone cables, TAT-7, was in use.
It could handle a staggering 4000 telephone conversations concurrently.
The first transatlantic fiber optic cable, the TAT-8, was laid in 1988 and increased the bandwidth to the equivalent of forty thousand telephone conversations.

All of these cables, both copper coaxial cables as in TAT-1 through TAT-7 and fiber optic cables from TAT-8 to the present day, carry more than one conversation over a single physical connection.
The earliest telephone wires, in the nineteenth century, supported only a single conversation, because they used \textbf{\emph{baseband communication}}\index{baseband communication}.
In baseband communication, the signal to be sent is sent on the wire unadorned.
If you want to send a 3kHz sine wave, then the wire carries your 3kHz sine wave.
This requires one physical connection for every concurrent conversation, which obviously incurs an enormous cost.
It would not be practical to have one pair of copper wires for each phone conversation crossing the Atlantic.
Instead, \textbf{\emph{signal modulation}}\index{signal modulation} is used.
The original data signal is used to modulate, or modify, a \textbf{\emph{carrier signal}}\index{carrier signal}.
For example, if your data signal is just binary data, the simplest form of modulation might be \textbf{\emph{on-off keying}}\index{on-off keying}, where the carrier is simply turned on and off, like blinking a light.
For both analog and digital data signals, many modulation schemes have been developed, but this topic is beyond the scope of this book.
See the ``Further Reading'' at the end of the chapter block for more.

Generally speaking, the carrier signal must be a higher frequency than the baseband signal you are modulating.
By using separate carrier frequencies, we can have a copper cable or optical fiber carry more than one conversation at the same time. These frequencies are divided into \textbf{\emph{channels}}\index{channel (frequency or wavelength)}~\footnote{In this book, usually when we use the term ``channel'', we are referring to the physical fiber or connection, instead of this frequency-based concept. Sometimes, we are referring to an abstraction that carries a qubit through space or time, usually in the context of error analysis and correction, as in ``bit flip channel''.}.
This is a form of \textbf{\emph{frequency division multiplexing}}, or FDM\index{frequency division multiplexing (FDM)}, the same scheme used for radio stations and television channels.
When applied to fiber optics, this technique is usually referred to as \textbf{\emph{wavelength division multiplexing}} (WDM)\index{wavelength division multiplexing (WDM)}, instead of FDM.
We will see other multiplexing schemes for sharing physical resources in the next chapter.

In this chapter, we are going to focus on the main considerations when designing these cables, particularly bandwidth and noise.

\textbf{\emph{Bandwidth}}\index{bandwidth} tells us how much information a cable or fiber can carry.
The bandwidth depends on both the physical properties of the fiber itself, as well as how clever we are when it comes to encoding this information before it gets transmitted.
With modern techniques such as \textbf{\emph{dense wavelength division multiplexing}} (DWDM)\index{dense wavelength division multiplexing (DWDM)}, we are able to reach some staggering speeds.
For example, in a standard cable over the distance of 6,600 kilometers, speeds of around 65 terabits per second can be reached.
In a more specialized cable and over shorter distances, this can be increased to over 150 terabits per second.
Over very short distances, we achieved speeds of one petabit per second\footnote{For your reference, a terabit is $10^{12}$ bits, while a petabit is $10^{15}$ bits.}.

No system is perfect, and what we put in is not exactly what we are going to get out at the end of the cable.
We must consider the main sources of loss in optical fibers.
We will consider the following five losses: mode dispersion, absorption, scattering, bending of the fiber, and coupling.
The combined result of all of these sources of loss is that our signal becomes \textbf{\emph{attenuated}}\index{attenuation}, meaning that the output signal will have less power than the input signal.
It is very important to know how much the signal becomes attenuated and how we can combat this attenuation.

Another important factor that affects how much the optical signal is attenuated is the wavelength of the light used for encoding.
Different wavelengths have different absorption coefficients.
This naturally makes certain wavelengths more suitable for long-distance communication.
The three main wavelengths used for fiber optic communication are near 850, 1300, and 1550 nanometers, which all lie in the infrared spectrum.
The wavelength of light is important in quantum communication for other reasons as well.
Not all sources of light can produce single photons of the desired wavelength.
Detector efficiencies depend on the wavelength of the photons, and quantum memories can couple to light only in certain wavelength regions. When multiple connections (whether classical or quantum) share the fiber, they are assigned separate wavelengths.
If the assigned wavelengths are too far apart, then fewer channels can be carried; if they are too close together, \textbf{\emph{filtering}}\index{filtering}, or selecting out, the desired channel is hard and inter-channel interference becomes a problem. 
All these factors introduce complications which must be accounted for when designing and implementing quantum networks.

First, we will talk about the dispersion in optical fibers, then we will consider the other sources of attenuation. Next, we will move on to overcoming these losses, and finally, we will consider the challenges these sources of loss present in the context of quantum communication.





%%%%%%%%%%%%%%%%%%%%%%%%%%%%%%%%
\section{Mode dispersion}
\label{sec:11-2_mode_dispersion}
%%%%%%%%%%%%%%%%%%%%%%%%%%%%%%%%
\begin{figure}
    \centering
    \includegraphics[width=\textwidth]{lesson11/11-2_dispersion.pdf}
    \caption[Mode dispersion.]{Different modes traverse the length of the fiber in different times, leading to mode dispersion. Initially sharp signals become difficult to decode.}
    \label{fig:11-2_mode_dispersion}
\end{figure}
\textbf{\emph{Mode dispersion}}\index{mode dispersion} is the first source of signal degradation in the fiber that we are going to consider.
Let's examine the propagation of different modes in a multimode fiber\index{multimode fiber}, as in Fig.~\ref{fig:11-2_mode_dispersion}.
A multimode fiber can contain many modes, all traveling along different paths.
The shortest route is the \textbf{\emph{axial path}}\index{axial path}, which travels directly down the fiber.
The light can also take non-axial paths where it undergoes total internal reflection\index{total internal reflection}.
The total length travelled by non-axial modes is longer than for the axial mode, resulting in a longer propagation time for those modes.

It is important to quantify the time difference between the different modes.
From Fig.~\ref{fig:11-2_mode_dispersion}, it is intuitive that depending on the angle with which the light is coupled to the fiber, the path that it takes will be different.
The fastest mode, travels directly down the center of the fiber.
This is the axial mode.
The slowest mode is the one that is incident on the cladding just at the critical angle, meaning it just barely gets internally reflected.

The initial digitized signal may look something like the clean modulated square wave in the upper right of Fig.~\ref{fig:11-2_mode_dispersion}. It is easy to read out the value of the individual bits.
But as the different modes propagate down the fiber, the whole packet spreads and disperses.
After some distance, it resembles the noisy signal at the bottom right of the Fig.~\ref{fig:11-2_mode_dispersion}.
This means that the readability of the output signal worsens, making it easier to make a mistake during decoding of the signal.

\begin{figure}
    \centering
    \includegraphics[width=0.8\textwidth]{lesson11/11-2_dispersion_delay.pdf}
    \caption[Time delay due to mode dispersion.]{Time delay due to mode dispersion.}
    \label{fig:11-2_dispersion_delay}
\end{figure}

How quickly this dispersion occurs can be quantified by the time between the fastest and slowest modes.
We are going to consider a horizontal length $L$ that the axial mode traverses.
The time that the axial mode takes to cover distance $L$ is the following,
\begin{equation}
    t_{\min } =\frac{L}{v_f}.
\end{equation}
The speed of of light in the fiber, $v_f$, depends on the refractive index of the material, $n_f$, that the fiber is made of, $v_f = c / n_f$.
Therefore the minimum time can be written as
\begin{equation}
    t_{\min } =\frac{Ln_f}{c}.
\end{equation}

Before considering the time $t_{\max}$ that the slowest mode takes to cover the horizontal distance $L$, we consider a general non-axial mode as shown in Fig.~\ref{fig:11-2_dispersion_delay}.
The total distance $l$ that this mode travels is larger than $L$.
The distance traveled before the internal reflection is $l_1$, while the distance traveled after the internal reflection is $l_2$.
Naturally, we have $l = l_1 + l_2$.

Let's start by computing $l_1$,
\begin{equation}
    l_1=\frac{L_1}{\cos \theta_r},
\end{equation}
where $\theta_r$ is the angle of refraction at the interface between the core of the fiber and the fiber's surroundings.
A similar expression is true for $l_2$, namely $L_2 / \cos \theta_r$.
Therefore, the total distance that the light travels is given in terms of the axial distance $L$ and the angle of refraction $\theta_r$,
\begin{equation}
    l=\frac{L_1}{\cos \theta_r}+\frac{L_2}{\cos \theta_r}=\frac{L}{\cos \theta_r}.
\end{equation}
The time needed for this non-axial mode to cover the horizontal distance $L$ is
\begin{equation}
    t = \frac{l}{v_f} = \frac{L}{v_f \cos \theta_r} = \frac{L n_f}{c \cos \theta_r}.
    \label{eq:11-2_time_nonaxial}
\end{equation}

We are now in position to compute $t_{\max}$.
The slowest mode is the one that has to cover the longest distance $l$.
In order for this to happen, the angle at which this mode is incident at the cladding is maximized while still being completely reflected.
The mode needs to be incident at the cladding at the critical angle $\theta_c$, which in turn fixes the maximum angle of refraction such that
\begin{equation}
    \cos\theta_r = \sin\theta_c = n_c / n_f,
    \label{eq:11-2_cos_theta_r}
\end{equation}
where the last equality comes from Eq.~(\ref{eq:7-4_crit_angle}).
Substituting for $\cos\theta_r$ in Eq.~(\ref{eq:11-2_cos_theta_r}), we obtain the time it takes for the slowest mode to traverse the horizontal distance $L$,
\begin{equation}
    t_{\max }=\frac{L n_f^2}{c n_c}.
\end{equation}

It is now easy to put the expressions for $t_{\mathrm{max}}$ and $t_{\mathrm{min}}$ together, and obtain the time delay between the fastest and slowest modes, 
\begin{equation}
    \Delta t = t_{\max} - t_{\min} = \frac{L n_f}{c}\left(\frac{n_f}{n_c}-1\right).
    \label{eq:11-2_time_delay}
\end{equation}

Let's consider an example to gain some intuition about the time delays introduced by mode dispersion and their effect on the encoded message.
Consider a fiber with refractive index $n_f = 1.500$, and cladding with refractive index $n_c = 1.489$.
Using Eq.~(\ref{eq:11-2_time_delay}), we obtain the following time delay per kilometer,
\begin{equation}
    \frac{\Delta t}{L}=37 \mathrm{~ns} / \mathrm{km}.
\end{equation}
This does not seem much.
Let's see the consequences of this small delay.
The speed of light in this fiber is given by $c$ divided by the refractive index of this fiber, which is $v_f=c / n_f=2 \times 10^8 \mathrm{~m} / \mathrm{s}$.
This time delay between the fastest and slowest modes means that our pulse spreads over a distance as it travels through the fiber. We can quantify this dispersion and obtain the value $7.4$ meters per kilometer -- our slowest signal is about three-quarters of one percent slower than our fastest. Every kilometer that our signal travels, it becomes more and more spread by this distance of $7.4$ meters.
Because our wave packets are becoming more spread out, we are losing the sharpness of our signal and they become more difficult to decode.
In order to be able to read the output signal clearly, we demand that the pulses that are coming out of our fiber be separated by twice the value of the spread.
This means that in order for the dispersed signal that is coming out of the fiber to be intelligible, we require that the input pulses also be separated by at least $2\times 7.4 = 14.8$ meters even when connecting over a distance of only one kilometer.
This inadvertently places a limit on how fast we can transmit information because the pulses have to be separated by a certain amount.
Dispersion has a direct consequence on the maximum frequency of the input signal.




%%%%%%%%%%%%%%%%%%%%%%%%%%%%%
\section{Attenuation}
\label{sec:11-3_attenuation}
%%%%%%%%%%%%%%%%%%%%%%%%%%%%%

First, let's look at \textbf{\emph{absorption}}\index{absorption}.
Absorption occurs due to the interaction of light with the material of the fiber.
There are two types of absorption: \textbf{\emph{intrinsic absorption}}\index{intrinsic absorption} and \textbf{\emph{extrinsic absorption}}\index{extrinsic absorption}.
Intrinsic absorption is responsible for attenuating the signal due to electronic and vibrational resonances of the molecules in the fiber.
For silica molecules, electronic resonances occur in the ultraviolet region, and vibrational resonances in the infrared region.
These resonances cause the light to transfer energy to the molecules in the fiber resulting in a power loss in the signal.
This is an intrinsic property of the fiber and we cannot really affect it; it's something that we just have to live with and accept.
Luckily, it's not a very significant source of error, especially when we compare it to the other type of absorption.

Extrinsic absorption is due to impurities introduced during the manufacturing process.
On one hand, this is a much more significant source of absorption, but on the other we can control it by perfecting the manufacturing process.
Generally, the manufacturers of fiber optic cables try to keep the the amount of impurities below one percent.

\begin{figure}[t]
    \centering
    \includegraphics[width=\textwidth]{lesson11/11-3_attenuation.pdf}
    \caption[Attenuation losses.]{Various types of attenuation losses.}
    \label{fig:11-3_attenuation}
\end{figure}
In this section, we will consider the rest of the sources of losses that we listed in Sec.~\ref{sec:ld-intro}.
They are all pictured in Fig.~\ref{fig:11-3_attenuation}.

Another source of attenuation is \textbf{\emph{scattering}}\index{scattering}, which can be due to small fluctuations of the density in the fiber material which are introduced during manufacturing process.
This results in a non-uniform refractive index in the material causing scattering of light.
Other sources of scattering include molecules absorbing the light and re-emitting it in a random direction.

The final two sources of attenuation are \textbf{\emph{bending}}\index{bending} and \textbf{\emph{coupling}}\index{coupling}.
Losses due to bending occur when the fiber is bent at a sharp enough angle that the light leaks from it.
Remember, we said that the angle of incidence is crucial for total internal reflection to take place, and therefore for the signal to propagate down the fiber.
Typically, the manufacturers specify a minimum bending radius, usually around 10-20 times the diameter of the fiber.
Inevitably, we will have to join two fibers together.
If the coupling between the fibers introduces a gap, or if the fibers are not aligned correctly, some of the light will leak leading to attenuation of the signal.

Now that we have covered the main sources of loss, let's discuss how to quantify the attenuation in the fiber.
The degree of attenuation is expressed by a unit called the \textbf{\emph{decibel}}\index{decibel}.
Decibel is the ratio of the power that is received at the end of the fiber, $P_o$, given the initial power of the signal, $P_i$.
The number of decibels is defined as the following expression,
\begin{equation}
    \# \text { of } \mathrm{dB}=-10 \log _{10} \frac{P_o}{P_i}
    \label{eq:decibels}
\end{equation}
Remember, the power out $P_o$ is smaller than power in $P_i$.
This ratio is smaller than one, meaning the logarithm is negative.
Applying a minus sign in front of the whole expression lets us talk in terms of positive numbers of decibels.
For convenience, we multiply by ten (``deci'' means ten) and talk about decibels.
Why take the logarithm?
Often, the powers in and out are separated by orders of magnitude.
Taking the logarithm produces a nicer scale for the number of decibels.
Moreover, since every kilometer of fiber multiplies the power loss, using a logarithmic scale makes computations of total loss easier.

\begin{table}
    \setcellgapes{3pt}
    \renewcommand\theadfont{}
    \makegapedcells
    \centering
    \begin{tabular}{cc}
        \hline
        \boldmath$P_o/P_i$ & \textbf{dB} \\
        \hline $1/10$ & 10 \\
        $1/100$ & 20 \\
        $1/1000$ & 30 \\
        \hline
    \end{tabular}
    \caption[Decibels example.]{Examples of calculating loss in decibels (dB).}
    \label{tab:decibels}
\end{table}

We can see few examples what various levels of decibels mean in Tab.~\ref{tab:decibels}.
Ratio of power out to power in as $1/10$, meaning that ninety percent of the power is lost, corresponds to 10 decibels (10dB), as you can convince yourself by substituting for $P_0/P_i$ into Eq.~\ref{eq:decibels}.
Ratio is $1/100$ corresponds to 20 decibels, and ratio of $1/1000$ corresponds to 30 decibels.
You can see that when the ratio of the power out over power in shrinks by an order of magnitude, the loss in decibels increases by ten decibels.

The quality of an optical fiber is often characterized by the \textbf{\emph{attenuation parameter}}\index{attenuation parameter} $\alpha$.
Its units are decibels per kilometer (dB/km).
Dividing Eq.~(\ref{eq:decibels}) by the length of the fiber $L$, and after quick rearrangement obtain hte following,
\begin{equation}
    \alpha = -\frac{10}{L} \log _{10} \frac{P_o}{P_i}, \quad \rightarrow \quad \frac{P_o}{P_i} = 10^{-\alpha L / 10}.
    \label{eq:attenuation_parameter}
\end{equation}
The ratio $P_o/P_i$ does not only represent how weaker the output signal is compared to the initial signal, but also the \textbf{\emph{probability of a single photon travelling the length of the fiber without being lost}}.

We can compute the attenuation parameter for the optical fiber that manages to transmit just $1\%$ of the initial power over a distance of one kilometer.
This was state-of-the-art around 1970.
Using Eq.~(\ref{eq:attenuation_parameter}), the attenuation parameter is $\alpha=20$ dB/km.
The transmission probability rose to around $96\%$ over a kilometer two decades later, which corresponds to an attenuation level of 0.18 dB/km. Three decades later, attenuation levels remain around this value, though attenuation parameters as low as 0.12 dB/km have been demonstrated.

\begin{figure}
    \centering
    \includegraphics[width=0.6\textwidth]{lesson11/11-3_attenuation_dB.pdf}
    \caption[Power loss in fiber.]{Loss in the fiber with distance has improved substantially over the last few decades.}
    \label{fig:11-3_attenuation_dB}
\end{figure}
Figure~\ref{fig:11-3_attenuation_dB} shows the loss versus distance for these two values of attenuation loss.
On the horizontal axis, we plot the length of the fiber through which our signal is traveling, and on the vertical axis we have the ratio of the power out over power in.
If the length of the fiber is zero, then of course the ratio is one.
The signal has not traveled anywhere.
As it travels through the fiber, the blue line corresponds to the attenuation levels that were achieved in 1970, with $\alpha=20$ dB/km, while the orange line is the attenuation parameter achieved in 1990 with $\alpha=0.18$ dB/km.
We can see how quickly the ratio approaches zero for the high attenuation parameter, whereas for the very low attenuation parameter, it decreases much more slowly.




%%%%%%%%%%%%%%%%%%%%%%%%%%%%%%%%%%
\section{Overcoming losses}
\label{sec:11-4_overcoming_losses}
%%%%%%%%%%%%%%%%%%%%%%%%%%%%%%%%%%

Let's begin with tackling mode dispersion as it is one of the most straightforward sources of error to overcome.
We discussed how a signal propagating in a multi-mode fiber deteriorates due to the difference in lengths of the various paths that the modes take.
Decreasing the cross-section of the fiber has the effect of limiting the number of modes that the fiber can support.
\textbf{\emph{Single-mode fibers}}\index{single-mode fiber} are narrow enough that they only support the axial mode.
Having only a single mode propagating in the fiber eliminates mode dispersion completely since there is nothing that can get dispersed.

We mentioned that absorption and scattering can be reduced by improving the manufacturing process.
Bending loss is reduced by keeping in mind manufacturer's advice about the maximum bending radius.
Coupling errors can be eliminated, at least partially, by ensuring that the fibers are aligned properly and there is no gap between them.

No matter how many precautions we take when manufacturing and using the fiber, there will always be attenuation resulting in deterioration of the signal.
In Fig.~\ref{fig:11-3_attenuation_dB}, even for a fiber with low attenuation parameter of $\alpha=0.18$ dB/km, the signal becomes attenuated.
The power decreases to 80\% over a short distance of 5 kilometers.
After 20 kilometers, the power drops to 44\%; after 50 kilometers to a mere 13\%, and at a hundred kilometers it is virtually zero.
Hundred kilometers is not such a long distance.
We require systems that can transmit signals over thousands of kilometers.

Long-distance transmission is achieved with the help of \textbf{\emph{repeaters}}\index{repeater (classical)}~\footnote{Just to remind you, we will encounter this term "repeater" in the next chapter as well, where we will be talking about quantum repeaters, but they work on a very different basis than classical repeaters. Here, we are talking only about classical repeaters.}, which are devices that are used to boost the signal strength.
There are many different kinds of repeaters based on different physical principles; as an example, we introduce the \textbf{\emph{erbium-doped fiber amplifiers}}\index{amplifier}, or EDFA\index{erbium-doped fiber amplifier (EDFA)} for short.
Fig.~\ref{fig:11-4_edfa} illustrates the basic idea behind EDFA.
Erbium atoms are introduced into a portion of the fiber.
The erbium atoms are excited into their higher energy states using a strong pump, creating population inversion.
As the weak signal passes through this segment of the fiber, it stimulates emission from the erbium atoms.
We know that in the process of stimulated emission, the photons that are emitted are of the same kind as the incoming signal photons.
They have the same phase and travel in the same direction.
Basically, an EDFA is using the principle that is behind lasing to boost the weak signal.
The EDFA amplifies the signal, allowing it to travel further before it needs amplification again.
\begin{figure}[t]
    \centering
    \includegraphics[width=0.6\textwidth]{lesson11/11-4_edfa.pdf}
    \caption[Erbium-doped fiber amplifier (EDFA).]{Basic working principle of an erbium-doped fiber amplifier.}
    \label{fig:11-4_edfa}
\end{figure}
Commercially available amplifiers are capable of producing gains of up to 40 dB, and are fairly compact with the active region (red in Fig.~\ref{fig:11-4_edfa}) being only few centimeters long.

There is one problem with this approach that we have to keep in mind.
EDFAs not only amplify the signal, they also amplify any noise as well.
The excited erbium atoms amplify the signal when they undergo stimulated emission caused by the photons in the signal.
On occasion, these excited atoms emit photons via spontaneous emission, as we discussed in Section~\ref{sec:5-2_coherent_vs_incoherent}.
Photons emitted this way travel in a random direction.
It is not a big issue when they are emitted in the opposite direction from the incoming signal, as the propagating signal does not get polluted by these spontaneously emitted photons.
On the other hand, when the spontaneously emitted photons travel in the same direction as the signal, they contribute to the noise and are responsible for deterioration of the \textbf{\emph{signal-to-noise ratio (SNR)}}\index{signal-to-noise ratio (SNR)}.
Furthermore, these noisy photons may interact with other excited erbium atoms, stimulating their emission and inadvertently producing more noisy photons.


%%%%%%%%%%%%%%%%%%%%%%%%%%%%%%
\section{Quantum challenges}
\label{sec:quantum_challenges}
%%%%%%%%%%%%%%%%%%%%%%%%%%%%%%

So far, we have considered classical signals comprised of large number of photons.
In quantum communication, the signal is at the level of individual photons.
Photon loss in the fiber therefore becomes a major problem.
Losing a photon in from a classical signal slightly decreases the power of the signal.
Loss of a photon in quantum communication results in losing the entire signal itself!

Can we combat photon loss the way it is done in classical communication?
Namely, can we create backup copies of the single photons at regular intervals and resend those?
We can, but only if we limit ourselves to communication with orthogonal states.
For example, if we only wish to communicate qubits \ket{0} and \ket{1}, creating backup copies presents no issues and can be done.
This, however, would limit us to classical communication only.
All the magic of quantum communication lies in non-orthogonal states and entangled states.
We saw that creating copies of such states is not allowed by the no-cloning theorem in Sec.~\ref{sec:8-3_no-cloning}.

Since copy and resend strategy is fundamentally not allowed in quantum communication, it would seem we have to resort to send the single photons and hope for the best that they arrive at the destination.
Let's do a quick calculation that will demonstrate that this also is not a viable strategy.
The probability that we transmit the photon through a fiber with attenuation parameter $\alpha$ over distance $L$ is given by the following,
\begin{equation}
    P_{\text {success }}=10^{-\alpha L / 10}.
\end{equation}
Let's plug in some numbers to give us some intuition of what this probability is.
Consider a long fiber of 1000 km (which in the context of global communication is not that long), and assume a best-case scenario where the fiber has ultra-low attenuation of a mere $0.1$ dB/km.
The probability that a single photon gets successfully transmitted is $10^{-10}$.
This is indeed a very small number, but to gain some intuition of how small, consider the case of a source that produces one photon every second.
Every second a single photon gets sent down the fiber.
How long do we need to wait in order for the photon to get successfully transmitted?
On average, we expect to wait 317 years!
We can see that even with ultra-low loss fibers over moderate distances, sending a single photon down the fiber and hoping for the best is not very practical.

Loss is only one source of error that we have to contend with in long distance quantum communication.
Other sources include unitary errors such as Pauli errors, namely $X$ errors that randomly flip the state of our photons, or $Z$ errors where we introduce a phase to the photons.
There are several types of non-unitary errors such as relaxation that can affect the state of qubits.
We do not have to deal with most of these errors in classical communication.
The situation looks dire for quantum communication.
However, we will see in the next Chapter that quantum problems require quantum solutions, and indeed there are clever ways that address these issues.





\newpage
\begin{exercises}

\exer{
\emph{Pulse broadening due to mode dispersion.}
We would like to send a burst of light through a fiber of length $L$.
The light rays enter the fiber simultaneously but at different angles $\theta$, given by some distribution $d(\theta)$.
\subexer{
Draw the shape of the input signal.
}
\subexer{
What is the shape of the output signal if all light rays are incident onto the fiber core along the fiber's axis? What is the delay in the output signal?
}
\subexer{
What is the shape of the output signal if all light rays are incident a the maximum acceptance angle $\theta_{\text{max}}$? What is the delay in the output signal?
}
\subexer{
What is the shape of the output signal if the distribution of the incident angles is
\begin{equation}
    d(\theta) = \frac{\delta(0)+\delta(\theta-\theta_{\text{max}})}{2},
\end{equation}
that is half of the light rays are incident along the fiber's axis, and half are incident at the maximum acceptance angle? What is the delay in the output signal?
}
\subexer{
Write some code that lets the user input an arbitrary distribution of angles $d(\theta)$, refractive indices of the core the core and cladding, and computes the shape of the output pulse.
}
}

\exer{
\emph{Direct transmission of classical signals.}
Consider a signal of some initial power $P_i$.
After travelling in the fiber for length $L$, the power decreases to $P_o$.
\subexer{
How much of the initial signal remains if the signal undergoes a loss of 1 dB, 2 dB, 5 dB, 10 dB, and 50 dB?
}
\subexer{
Plot the fraction of the remaining power of the signal, $P_o/P_i$, as a function of the number of dB.
}
\subexer{
For each of the cases in subexercise \textbf{a)}, compute the attenuation parameter $\alpha$, if the length of the fiber is $L=10$ km.
}
}

\exer{
\emph{Direct transmission of single photons.}
Consider transmitting single photons through the fiber of length $L$.
\subexer{
Compute the probability of the photon being transmitted if it suffers loss of 1 dB, 2 dB, 5 dB, 10 dB, and 50 dB?
}
\subexer{
Check the claim that we would need to wait 317 years on average to receive a photon sent through a 1000 km long fiber with attenuation parameter 0.1 dB/km, given the repetition rate is 1 Hz.
}
\subexer{
How long would we expect to wait if the attenuation rate was 0.2 dB/km, but the repetition rate increased to 1 GHz?
}
}

\exer{
\emph{Amplification of classical signals.}
We have seen how Eq.~(\ref{eq:decibels}) describes decrease in power of the classical signal as it propagates through a fiber.
But decibels can describe power gain, that is amplification, as well.
\subexer{
Can you think of a way of altering Eq.~(\ref{eq:decibels}) such that it describes power gain in dB?
}
\subexer{
We suggested that amplifiers are used every 50km or so.  See if you can find more detailed information about the fiber loss and the spacing of amplifiers in real-world deployments and the amount of gain an EDFA can generate.
}
}

\end{exercises}

