\chapter{Entanglement Revisited}

In this chapter, we will return to entanglement and discuss it in the context of quantum repeater networks.
We begin with revisiting bipartite entanglement in section~\ref{sec:14-1_bipartite}.
We focus on how to determine the quality of the entanglement shared between distant nodes of a network and why it matters.
In section~\ref{sec:14-2_multipartite}, we will introduce multipartite entanglement shared between more than two nodes of a quantum network.
We will discuss how it differs from bipartite entanglement before moving on to examples of multipartite entangled states.
In the remainder of the chapter, we will shift our focus to applications of quantum repeater networks.
In section \ref{sec:14-3_clock_sync}, we discuss clock synchronization, and distributed blind quantum computation in section ~\ref{sec:14-4_distributed_bqc}.



\section{Bipartite entanglement}
\label{sec:14-1_bipartite}

\begin{figure}[t]
    \centering
    \includegraphics[width=\textwidth]{lesson14/14-1_bipartite_network.pdf}
    \caption[Bipartite and multipartite entanglement]{Bipartite and multipartite entanglement distribution in quantum networks.}
    \label{fig:14-1_bipartite_multipartite}
\end{figure}

One of the main jobs of a quantum network is to distribute entanglement.
Let's start by considering a concrete example.
Figure~\ref{fig:14-1_bipartite_multipartite} shows a quantum network.
The circles represent quantum nodes while the dashed lines represent physical links connecting neighboring quantum nodes.
Nodes $N_1$ and $N_2$ wish to establish shared entanglement in order to engage in quantum communication.
We saw in the previous chapter that in order to satisfy this request, the quantum network first creates link-level entanglement between neighboring nodes along the path connecting the nodes $N_1$ and $N_2$.
Entanglement swapping is then used to create direct end-to-end bipartite entanglement between these nodes.
This example is rather constrained in the sense that it only shows two of the nodes trying to engage in quantum communication.
A well-designed quantum network should be able to accommodate multiple simultaneous requests for end-to-end entangled connections between multiple pairs of nodes, as shown in Fig.~\ref{fig:14-1_bipartite_multipartite}. 

The quality of the distributed entanglement, whether bipartite or multipartite, matters greatly.
We saw that in entanglement-based QKD, the quality of the entanglement that is shared between Alice and Bob directly impacts the key that they are trying to establish.
If the entangled state shared by Alice and Bob is not perfect then the key that they are going to end up with after at the end of the protocol will also be only partially correlated.
This has two important consequences.
First, it affects how well Alice and Bob can communicate.

More crucially, the quality of the entangled state affects the \textit{\textbf{security of the key}} that they can establish.
The entangled state may be imperfect due to natural noise introduced by the physical channels during the distribution process.
But it could also be a result of an eavesdropper Eve, trying to listen in on the secret communication between Alice and Bob as shown in Fig.~\ref{fig:14-1_QKD}.
If Alice and Bob want to maintain security of the information that they are wishing to communicate, they must assume that any imperfections in the entangled state have been introduced by Eve.
So the quality of the entangled state places an upper bound on the security of the key that can be established between Alice and Bob in the sense that \textit{\textbf{better quality of the entangled state implies stronger security}}.

We discussed in Section~\ref{sec:3-5_fidelity} that one way of quantifying the quality of an entangled state is the fidelity.
Given a ideal state \ket{\psi}, that the quantum network is trying to distribute, the fidelity of the actual mixed state $\rho$ shared between Alice and Bob is given by the expectation value of $\rho$ with respect to \ket{\psi},
\begin{equation}
    F \left( \rho, \ket{\psi} \right) = \bra{\psi}\rho\ket{\psi}.
\end{equation}
It is usually the case in quantum networks that the ideal state is one of the Bell pairs.
If the ideal state is \ket{\Phi^+}, then the fidelity if $F (\rho, \ket{\Phi^+}) = \bra{\Phi^+}\rho\ket{\Phi^+}$.
For noiseless quantum channels and operations, perfect quantum memories and in the absence of an eavesdropper, the distributed state will be the ideal state, $\rho = \ket{\Phi^+}\bra{\Phi^+}$.
The resulting fidelity is
\begin{equation}
    F \left( \rho, \ket{\Phi^+} \right) = \langle\Phi^+\ket{\Phi^+} \langle\Phi^+\ket{\Phi^+} = 1.
\end{equation}
On the other hand, the distributed state may have some error, for example a Pauli $Z$ error on Bob's qubit.
This transforms the entangled state into an orthogonal Bell pair, $I \otimes Z \ket{\Phi^+} = \ket{\Phi^-}$, resulting in fidelity
\begin{equation}
    F \left( \rho, \ket{\Phi^+} \right) = \langle\Phi^+\ket{\Phi^-} \langle\Phi^-\ket{\Phi^+} = 0.
    \label{eq:14-1_fidelity_0}
\end{equation}
Typically, the distributed state will be mixed, and the fidelity will take values $ 0 \leq F(\rho, \ket{\Phi^+}) \leq 1$.

\begin{figure}[t]
    \centering
    \includegraphics[width=0.4\textwidth]{lesson14/14-1_QKD.pdf}
    \caption[Eavesdropper Eve]{Eavesdropper Eve tampering with the distributed entangled state between Alice and Bob in order to gain information about the shared secret key.}
    \label{fig:14-1_QKD}
\end{figure}

Note that the lowest possible value of the fidelity does not imply that the distributed state is useless -- far from it.
As we saw in Eq.~(\ref{eq:14-1_fidelity_0}), the fidelity vanishes when the distributed state is orthogonal to the ideal state \ket{\Phi^+}.
This means that the distributed state is also maximally entangled and can be used for quantum communication after an appropriate transformation.
For the case discussed in Eq.~(\ref{eq:14-1_fidelity_0}), the distributed state can be transformed into the ideal one by either Alice or Bob applying a local Pauli $Z$ operation,
\begin{align}
    \text{Alice applies } Z: \quad & Z \otimes I \ket{\Phi^-} = Z \otimes Z \ket{\Phi^+} = \ket{\Phi^+}, \\
    \text{Bob applies } Z: \quad & I \otimes Z \ket{\Phi^-} = I \otimes Z^2 \ket{\Phi^+} = \ket{\Phi^+}.
\end{align}
We used that fact that applying the Pauli $Z$ operation twice cancels its effect, $Z^2 = I$.

We saw that both extremes of the fidelity, $F(\rho,\ket{\psi}) = 1$ and $F(\rho,\ket{\psi}) = 0$, suggest that the distributed state is useful for quantum communication.
Does this mean that distributed states with intermediate values of fidelity are useful too?
Certainly not.
Let's consider a fully decohered state of two qubits given by the maximally mixed state,
\begin{equation}
    \rho = \frac{1}{4} \left( \ket{00}\bra{00} + \ket{01}\bra{01} + \ket{10}\bra{10} + \ket{11}\bra{11} \right).
\end{equation}
The fidelity of this state with respect to the ideal Bell pair \ket{\Phi^+} is
\begin{align}
    F \left( \rho, \ket{\Phi^+} \right) & = \frac{1}{4} \bra{\Phi^+} \left( \ket{00}\bra{00} + \ket{01}\bra{01} + \ket{10}\bra{10} + \ket{11}\bra{11} \right) \ket{\Phi^+} \nonumber\\
    & = \frac{1}{4} \left( \frac{1}{2} + \frac{1}{2} \right) = \frac{1}{4}.
\end{align}
We see that it is $F=0.25$ that signifies the distributed state is ``completely useless''.

Fidelity is not the only useful metric when it comes to quantifying the quality of the distributed state.
The other method relies on the CHSH inequality which we saw plays a crucial role in the E-91 QKD protocol in Chapter~\ref{sec:10_E91}.
Named after its inventors Clauser, Horne, Shimony, and Holt, the CHSH inequality is a test of the quantumness of the correlations shared between Alice and Bob.
We saw that to test that Alice and Bob share the \ket{\Psi^+} Bell pair, they have to repeatedly measure in the following bases,
\begin{align}
    \text{Alice's measurement bases:}& & A_1 & = Z, & A_2 & = X \\
    \text{Bob's measurement bases:}& & B_1 & = \frac{Z - X}{\sqrt{2}}, & B_2 & = \frac{Z + X}{\sqrt{2}}.
\end{align}
Using the outcomes of the measurements, they can then compute the CHSH correlation,
\begin{equation}
    \mathcal{S} = \langle A_1 \otimes B_1\rangle + \langle A_1 \otimes B_2\rangle + \langle A_2 \otimes B_1\rangle - \langle A_2 \otimes B_2\rangle.
    \label{eq:14-1_CHSH_Psi_plus}
\end{equation}
For the ideal state, the expectation values can be computed according to $\langle A_i B_j \rangle = \bra{\Psi^+} A_i \otimes B_j \ket{\Psi^+}$, leading to the CHSH correlation of
\begin{equation}
    \left| \mathcal{S} \right| = 2\sqrt{2}.
\end{equation}
For the distributed state given by a mixed state $\rho$, the expectation values can be calculated via the more general expression $\langle A_i \otimes B_j \rangle = \text{Tr} \{ A_i \otimes B_j \rho \}$.
The distributed state is entangled provided that
\begin{equation}
    \left| \mathcal{S} \right| > 2.
\end{equation}
The CHSH inequality acts as a witness of entanglement, with $|\mathcal{S}| = 2$ being the critical threshold.
When Alice and Bob measure $|\mathcal{S}| > 2$, they know they share an entangled state.
On the other hand, when $|\mathcal{S}| < 2$, they cannot conclude anything. The distributed state could be entangled but it could be separable as well.
CHSH inequality violation is also directly linked to the security of the key that Alice and Bob can establish.
\textit{\textbf{Stronger violation of the CHSH inequality leads to more secure secret key.}}

What if the ideal state that Alice and Bob are trying to share some other Bell pair, let's say \ket{\Phi^+}.
Computing the CHSH violation in Eq.~(\ref{eq:14-1_CHSH_Psi_plus}) but computing the expectation values with respect to the new ideal state \ket{\Phi^+}, we get
\begin{align}
    \langle A_1 \otimes B_1 \rangle & = \frac{1}{\sqrt{2}} & \langle A_1 \otimes B_2 \rangle & = \frac{1}{\sqrt{2}} \\
    \langle A_2 \otimes B_1 \rangle & = - \frac{1}{\sqrt{2}} & \langle A_2 \otimes B_2 \rangle & = \frac{1}{\sqrt{2}}.
\end{align}
This leads to a vanishing CHSH correlation, $\mathcal{S} = 0$.
This is because the expression for the correlation in Eq.~(\ref{eq:14-1_CHSH_Psi_plus}) is not designed to witness entanglement when the ideal state is \ket{\Phi^+}.
We have encountered a similar situation when we computed the fidelity between two orthogonal Bell pairs.
Can we use a similar expression to Eq.~(\ref{eq:14-1_CHSH_Psi_plus}) that witnesses entanglement when the ideal state is \ket{\Phi^+}.
All we need to do is exchange the plus and minus signs in front of the last two terms,
\begin{equation}
    \mathcal{S} = \langle A_1 \otimes B_1\rangle + \langle A_1 \otimes B_2\rangle - \langle A_2 \otimes B_1\rangle + \langle A_2 \otimes B_2\rangle.
    \label{eq:14-1_CHSH_Phi_plus}
\end{equation}
Eq.~(\ref{eq:14-1_CHSH_Phi_plus}) yields $|\mathcal{S}| = 2\sqrt{2}$ when the expectation values are computed with respect to \ket{\Phi^+}.

We have seen that both the fidelity as well as the CHSH inequality offer suitable tools to quantify the quality of the distributed bipartite state.
We have to remember to take care when computing both quantities.
In order to obtain meaningful results, we need to use the correct ideal state in the case of fidelity, and the appropriate form of the correlation in the case of the CHSH inequality.




\section{Multipartite entanglement}
\label{sec:14-2_multipartite}

We have seen how quantum networks can establish bipartite entanglement between two of its nodes.
We may wish to distribute entanglement shared between more than two nodes as shown in Fig.~\ref{fig:14-1_bipartite_multipartite}.
Sometimes three clients may wish to engage in a quantum communication protocol that might require all of them to share a tripartite entangled state.
Any entangled state of more than two subsystems is called a \textit{\textbf{entangled multipartite state}}\index{multipartite entanglement}.
It is the job a quantum network to distribute multipartite entangled states between any disjoint subsets of nodes that ask for them.

We begin by considering entangled states of three qubits.
We saw that there are 4 possible computational basis, \ket{00}, \ket{01}, \ket{10}, and \ket{11}.
How many computational basis states are there in this case?
Now there are 8 such states,
\begin{align}
    \ket{000}, \; \ket{001}, \; \ket{010}, \; \ket{011}, \; \ket{100}, \; \ket{101}, \; \ket{110}, \; \ket{111}.
\end{align}
Any general three-qubit state can be expanded as a superposition of these computational basis states,
\begin{align}
    \ket{\psi} & = c_0\ket{000} + c_1\ket{001} + c_2\ket{010} + c_3\ket{011} \nonumber\\
    & + c_4\ket{100} + c_5\ket{101} + c_6\ket{110} + c_7\ket{111}.
\end{align}
often this superposition is written in a compact form as
\begin{equation}
    \ket{\psi} = \sum_{i=0}^7 c_i \ket{i},
\end{equation}
where the notation means that the ``$i$'' in \ket{i} is the binary number corresponding to the index in the coefficients $c_i$.
Depending on the values of the coefficients, we obtain different states, some of which will be entangled.

Let's look at a few examples of important entangled multipartite states.
The first state is one of the simplest but also one of the most important.
It is an equal superposition of \ket{000} and \ket{111}, and is known as the \textit{\textbf{GHZ state}}\index{GHZ state}, named after Greenberger, Horne, and Zeilinger,
\begin{equation}
    \ket{GHZ} = \frac{1}{\sqrt{2}} \left( \ket{000} + \ket{111} \right).
\end{equation}
Let's see what are some of the properties of the GHZ state.
A good start is to measure one of the qubits and see what the state of the remaining qubits is.
Consider measuring the first qubit in the Pauli $Z$ basis.
The projectors corresponding to the +1 and -1 outcomes are
\begin{align}
    \Pi^Z_+ & = \ket{0}\bra{0} \otimes I \otimes I, \\
    \Pi^Z_- & = \ket{1}\bra{1} \otimes I \otimes I,
\end{align}
respectively.
The probability of the measurement outcome being +1 is
\begin{align}
    \text{Prob} \{+1\} & = \text{Tr} \{ \Pi^Z_+ \ket{GHZ}\bra{GHZ} \} \nonumber\\
    & = \bra{GHZ} \left( \ket{0}\bra{0} \otimes I \otimes I \right) \ket{GHZ} \nonumber\\
    & = \frac{1}{2} \left( \bra{000} + \bra{111} \right) \left( \ket{0}\bra{0} \otimes I \otimes I \right) \left( \ket{000} + \ket{111} \right) \nonumber\\
     & = \frac{1}{2} \bra{000} \left( \ket{0}\bra{0} \otimes I \otimes I \right) \ket{000} \nonumber\\
     & = \frac{1}{2}.
\end{align}
Similarly, the probability of the measurement outcome being -1 is
\begin{equation}
    \text{Prob}\{-1\} = \text{Tr}\{\Pi^Z_-\ket{GHZ}\bra{GHZ}\} = \frac{1}{2}.
\end{equation}
If the measurement outcome is +1, the post-measurement state is
\begin{align}
    \frac{1}{\sqrt{\text{Prob}\{+1\}}} \Pi^Z_+ \ket{GHZ} & = \sqrt{2} \left( \ket{0}\bra{0} \otimes I \otimes I \right) \frac{1}{\sqrt{2}} \left( \ket{000} + \ket{111} \right) \nonumber\\
    & = \left( \ket{0}\bra{0} \otimes I \otimes I \right) \ket{000} \nonumber\\
    & = \ket{000}.
\end{align}
We see that the post-measurement state is correlated, all qubits are in the same state, but it is no longer entangled.
Measurement of a single qubit in the Pauli $Z$ basis is enough to destroy the entanglement shared between all three qubits.
This is also the case if the measurement outcome is -1,
\begin{equation}
    \frac{1}{\sqrt{\text{Prob}\{-1\}}} \Pi^Z_- \ket{GHZ} = \ket{111}.
\end{equation}

The next example of an entangled state of three qubits is called the \textit{\textbf{W state}}\index{W state}.
In the computational basis, it is a superposition of three terms,
\begin{equation}
    \ket{W} = \frac{1}{\sqrt{3}} \left( \ket{001} + \ket{010} + \ket{100} \right).
\end{equation}
We can again measure the first qubit in the computational basis again.
this time, the probabilities of the two possible outcomes are
\begin{align}
    \text{Prob}\{+1\} & = \text{Tr} \{ \Pi^Z_+ \ket{W}\bra{W} \} = \frac{2}{3}, \\
    \text{Prob}\{-1\} & = \text{Tr} \{ \Pi^Z_- \ket{W}\bra{W} \} = \frac{1}{3}.
\end{align}
This time the probabilities of the two outcomes are different from each other, unlike in the case of the GHZ state.
What is more interesting though are the post-measurement states corresponding to the outcomes of the measurement,
\begin{align}
    \text{outcome +1:} & \quad \ket{0}\ket{\Psi^+}, \label{eq:14-2_W_post_meas_0} \\
    \text{outcome -1:} & \quad \ket{1}\ket{00},
\end{align}
We see that when the outcome of the measurement is +1, the state of the remaining two qubits is a maximally entangled Bell pair.
This is in stark contrast to the GHZ state, where a single measurement destroys all entanglement regardless of the outcome.
This points at a fundamental difference between the GHZ state and the W state in terms of their entanglement properties.

This difference goes even deeper.
We can consider what happens to the entanglement if one of the qubits is lost.
Losing a single qubit of a GHZ state will completely destroy the entanglement.
On the other hand, a loss of a single qubit from a W state leaves the remaining two qubits entangled, albeit not maximally.
The W state is more robust to qubit loss than the GHZ state.

\begin{figure}[t]
    \centering
    \includegraphics[width=\textwidth]{lesson14/14-2_monogamy.pdf}
    \caption[Monogamy of entanglement.]{Monogamy of entanglement restricts how three qubits can be correlated.}
    \label{fig:14-2_monogamy}
\end{figure}

Before moving onto $n$-partite entangled states, we will pause and think how the entanglement is shared between the three qubits.
There is a fundamental tradeoff to how strongly each pair of the qubits can be correlated, known as \textit{\textbf{monogamy of entanglement}}\index{monogamy of entnaglement}.
It states that the strength of entanglement between a pair of qubits places an upper bound on the amount of entanglement that the third qubit can share with either of the first two qubits.
Left panel of Figure~\ref{fig:14-2_monogamy} shows a general entnagled state of three qubits, where all three qubits are entangled with each other.
Examples corresponding to this situation are the GHZ and W states.
If a pair of qubits is in a maximally entangled state, for example \ket{\Phi+}, then monogamy of entanglement tell us that the third qubit must in a separable state with the first pair.
Total state of the three qubits can be written as
\begin{equation}
    \ket{\Phi^+}\ket{\psi},
\end{equation}
where \ket{\psi} is a pure state of the third qubit.
This is portrayed by the middle panel in Fig.~\ref{fig:14-2_monogamy}.
Similarly, if qubits 2 and 3 are maximally entangled like in the right panel of Fig.~\ref{fig:14-2_monogamy}, then the first qubit must be completely disentangled with either of them.
An example of this situation is given by hte post-measurement state of Eq.~(\ref{eq:14-2_W_post_meas_0}), where we measured the first qubit of a W state in Pauli $Z$ basis and obtained the +1 outcome.

Monogamy of entanglement is one of the most fundamental properties of quantum mechanics. It is unlike anything that exists in classical physics, and is one of the cornerstones and building blocks which we use in quantum technologies, and particularly in quantum communication.
Monogamy of entanglement is instrumental in guaranteeing security in the E91 QKD protocol.
The stronger Alice's and Bob's qubits are entangled the less information an eavesdropper can learn about the secret key.
If the shared pair of qubits between and Alice and Bob is maximally entangled, the eavesdropper can learn nothing.

We can extend our discussion of GHZ and W states to $N$ qubits.
The $N$-qubit GHZ state has the following form,
\begin{equation}
    \ket{GHZ} = \frac{1}{\sqrt{2}} ( \underbrace{\ket{00\ldots0}}_{N\text{ zeroes}} + \underbrace{\ket{11\ldots1}}_{N \text{ ones}} ).
\end{equation}
Notice that the normalization factor has not changed, even though this is now a GHZ state of $N$ qubits.
The W state can be extended to arbitrary number of qubits as well,
\begin{equation}
    \ket{W} = \frac{1}{\sqrt{N}} ( \underbrace{\ket{00\ldots01} + \ket{00\ldots10} + \ldots \ket{10\ldots00}}_{N \text{ terms}} )
\end{equation}
Since the $N$-qubit W state is an equal superposition of $N$ terms, he normalization factor is $1 / \sqrt{N}$.

\begin{figure}[t]
    \centering
    \includegraphics[width=0.6\textwidth]{lesson14/14-2_graph_state.pdf}
    \caption[Graph state.]{Graph state of four qubits and its underlying graph.}
    \label{fig:14-2_graph_state}
\end{figure}

GHZ and W states are not the only important examples of multipartite entangled states.
A prominent example of $N$-partite example is the \textbf{\textit{graph state}}\index{graph state}.
This state can be represented visually as a set of vertices $V$ which are connected by a set of edges $E$.
Such an object is known as a \textit{\textbf{graph}}\index{graph}\footnote{Not to be confused with a graph of a function.}
Figure~\ref{fig:14-2_graph_state} shows an example of a graph with vertex set $V=\{V_1, V_2, V_3, V_4\}$, connected by edges from the edge set $E=\{ E_{12}, E_{23}. E_{13}, E_{34} \}$.
The corresponding graph state is obtained by placing a qubit at each vertex of the graph, initialized in $\ket{+} = (\ket{0} + \ket{1}) / \sqrt{2}$, and applying the \textbf{\textit{control-phase gate}}\index{control-phase gate} $CZ_{ij}$ between qubits whose vertices are connected by and edge $E_{ij}$.
The control-pahse gate $CZ$ is a two-qubit gate defined as
\begin{equation}
    CZ_{ij} = \ket{0}\bra{0}_i \otimes I_j + \ket{1}\bra{1}_i \otimes Z_j = \begin{pmatrix}
        1 & 0 & 0 & 0 \\
        0 & 1 & 0 & 0 \\
        0 & 0 & 1 & 0 \\
        0 & 0 & 0 & -1
    \end{pmatrix}.
\end{equation}
\begin{figure}[t]
    \centering
    \includegraphics[width=\textwidth]{lesson14/14-2_cluster_state.pdf}
    \caption[Cluster state.]{Cluster states in different spatial dimensions.}
    \label{fig:14-2_cluster_state}
\end{figure}
The control-phase gate does nothing to the target qubit $j$ if the control qubit $i$ is in the \ket{0} state.
If the control $i$ is in the \ket{1}, the control-phase gate applies Pauli $Z$ to the target qubit $j$.
Applying this rule we can find the state vector for the graph state \ket{G} in Fig.~\ref{fig:14-2_graph_state},
\begin{align}
    \ket{G} & = CZ_{12} CZ_{23} CZ_{13} CZ_{34} |++++\rangle_{1234} \nonumber\\
    & = \frac{1}{4} ( |0000\rangle + |0001\rangle + |0010\rangle - |0011\rangle \nonumber\\
    & \; \; \; \; + |0100\rangle + |0101\rangle - |0110\rangle + |0111\rangle \\
    & \; \; \; \; + |1000\rangle + |1001\rangle - |1010\rangle + |1011\rangle \\
    & \; \; \; \; - |1100\rangle - |1101\rangle - |1110\rangle + |1111\rangle \\
    & = \frac{1}{2} ( |0+0+\rangle + |0-1-\rangle + |1-0+\rangle - |1+1-\rangle ).
\end{align}
If the underlying graph has a regular structure the resulting graph state is usually called a \textbf{\textit{cluster state}}\index{cluster states}.

Graph states and cluster states play an important role in quantum computation and quantum communication.
Cluster states are a resource state for a particular computational model known as ``measurement-based quantum computation''.
Graph states are useful in quantum error-correction as well as in many protocols in quantum communication.





\section{Clock synchronization}
\label{sec:14-3_clock_sync}

\begin{figure}[t]
    \centering
    \includegraphics[width=0.8\textwidth]{lesson14/14-3_clock_sync.pdf}
    \caption[Clock synchronization.]{Clock synchronization using photons or transport of matter.}
    \label{fig:14-3_clock_sync}
\end{figure}

So far, we have discussed two applications of quantum networks; quantum teleportation in Chapter ~\ref{sec:8_teleportation} and quantum key distribution in Chapters~\ref{sec:9_bb84} and \ref{sec:10_E91}.
Before concluding this chapter, we will briefly discuss two other applications.
In this Section, we look at \textbf{\textit{clock synchronization}}\index{clock synchronization}.

Establishing universal time standard is fundamentally important in many areas of modern life.
Telecommunication network requires accurate synchronization in order to deliver reliable operation of its multitude of components.
Synchronization plays an important role in the Global Positioning System (GPS), ensuring accurate position tracking.
Further examples of why clock synchronization is important include financial markets and transportation networks.

There are two main traditional methods of clock synchronization.
\textbf{\textit{Einstein's clock synchronization}}\index{Einstein's clock synchronization} relies on sending light signals between distant clocks.
Figure~\ref{fig:14-3_clock_sync} shows clocks 1 and 2 that we wish to synchronize.
Clock 1 fires a light pulse towards clock 2.
Once this pulse reaches clock 2, it is reflected back via a mirror and clock 2 is set ticking.
By measuring the time of emission as well as time of arrival of the light pulse, clock 1 can estimate when clock 2 started ticking.
This way clock 1 can adjust its time in order to match that of clock 2.
Another method is know as \textbf{\textit{Eddington's clock synchronization}}\index{Eddington clock synchronization}.
Using this approach, we first locally synchronize clock 1 with another small clock.
We then transport this smaller clock very slowly to the location of clock 2, which subsequently locally synchronizes itself to it.

\begin{figure}[t]
    \centering
    \includegraphics[width=0.8\textwidth]{lesson14/14-3_bloch.pdf}
    \caption[Qubit clocks.]{Two qubits acting as clocks.}
    \label{fig:14-3_bloch}
\end{figure}

Qubits can also act as small clocks.
Imagine preparing its state in $|+\rangle = (|0\rangle + |1\rangle) / \sqrt{2}$.
By applying a suitable operation we can make the state of the qubit precess in the $x$-$y$ plane of the Bloch sphere at a constant angular velocti $\Omega$ as shown in the left panel of Fig.~\ref{fig:14-3_bloch}.
The state of the qubit at time $t$ will then be given by
\begin{equation}
    |\psi(t)\rangle = \frac{|0\rangle + e^{i\Omega t} |1\rangle}{\sqrt{2}}.
\end{equation}
The dynamics of this qubit has a period of $T = 2\pi / \Omega$, meaning that the state of the qubit will repeat itself after every period, $|\psi(t+T)\rangle = |\psi(t)\rangle$.

We can now consider a second qubit, also precessing at the angular frequency $\Omega$, but initialized at a slightly different state, given by the offset $\delta$, as shown in the right panel of Fig.~\ref{fig:14-3_bloch}.
Eliminating this phase lag is very similar to the traditional problem of clock synchronization.
Schemes of Einstein and Eddington can be applied in this scenario as well.
However, they both with a downside that they require a direct transmission of light or matter.
Depending on the scenario, this may be impractical or outright impossible.
Thankfully, we can employ distributed entanglement between the two parties in order to synchronize their qubit clocks.
The precise steps of the protocol that achieves this are beyond the scope of this book but more details can be found in the Further Reading section.
Clock synchronization is another example where entanglement is used as a global resource.
The correlations of a Bell pair can be used to establish a single time frame shared between two distant locations.
Clock synchronization is not just another application of quantum networks but often is a hidden assumption behind many of the applications of quantum technologies, some of which we have discussed already.






\section{Distributed and blind quantum computation}
\label{sec:14-4_distributed_bqc}

\begin{figure}[t]
    \centering
    \includegraphics[width=0.4\textwidth]{lesson14/14-4_distributed_qc.pdf}
    \caption[Distributed quantum computation.]{Distributed quantum computation is another application of quantum networks.}
    \label{fig:14-4_distributed_qc}
\end{figure}

Imagine the scenario where a single quantum computer is not capable of executing the computation that we have in mind.
Either the quantum processing unit does not enough qubits, or they decohere too quickly, or the gates are of insufficiently fidelity.
The situation may seem dire but we need not despair.
We can get around this problem by using \textbf{\textit{distributed quantum computation}}\index{distributed quantum computation}.
Figure~\ref{fig:14-4_distributed_qc} shows five quantum computing nodes connected together via a quantum network.
Each QC node has limited quantum resources in the sense that we mentioned above.
Distributed quantum computation takes the original large problem which cannot be solved by a single QC node, and breaks it into smaller pieces which can be tackled by the QC nodes.
The individual QC nodes share their quantum resources in order to boost the performance or the efficiency of a single QC node.

Primary means of exchanging quantum information between the QC nodes is teleportation using pre-shared Bell pairs.
Another way of achieving distributed quantum computation is to entangle the QC nodes to form a larger distributed multipartite entangled state, which can serve as a substrate for measurement-based quantum computation.
In such a case the nodes then need to coordinate their measurements accordingly.
Whatever the means of achieving distributed quantum computation might be, quantum networks play an essential role in connecting the individually weak QC nodes using distributed entangled states.
The main questions of interest in distributed quantum computation are how much entanglement is needed in order to exchange quantum messages between the nodes, and how to efficiently coordinate the computation between the QC nodes.
Important question that needs to be addressed is whether the QC nodes can trust each other, and how much is one node willing to reveal to others about the local computation that it can execute.

The issue of trust brings us to our next application of quantum networks.
Early quantum computers will be sparse due to immense resources needed to run and maintain them.
Users with very limited quantum resources will have to delegate their computation to these quantum servers.
One way is to encode the instructions about the computation into classical bits and send them to the quantum server which then executes the computation.
By doing this, the server will learn the input, the computation itself, as well as the output.
What if the client does not wish to disclose any of this information?
Remarkably, the client can hide all this information by resorting to \textbf{\textit{blind quantum computation}}\index{blind quantum computation} (BQC).
The client only needs very limited quantum resources to engage in BQC.
One of the initial proposals showed that the ability to generate single-qubits is enough for the client to hide the input, the computation, and the output from the server.
The server can only find out an upper limit on the size of the quantum computation that it is being asked to execute.
Different variants of BQC work on the basis that the client can only do single-qubit measurements and it is the server which sends the qubits of a large entangled state one-by-one.
In either case, quantum networks will be needed in order to communicate the information contained in those qubits between the client and the server.




\newpage
\begin{exercises}
\exer{Let's look at measuring the W state in more detail.
Consider the projectors $\Pi_{\pm 1}^Z$ in Tab.~\ref{tab:3-2_projectors}.  To extend the measurement operators to measuring a subset of qubits, we need to tensor the projection operators with the identity matrix. 
\subexer{First, consider measuring a single qubit of the W state in the Z basis.
Calculate the probabilities of finding the $+1$ and $-1$ results using
$$\text{Pr} \{\pm 1\} = \langle W | \Pi_{\pm 1}^Z \otimes I \otimes I | W \rangle$$.}
\subexer{Calculate the probabilities of finding the $+1$ and $-1$ results when measuring in the X basis.
}
\subexer{Calculate the probabilities of finding the $+1$ and $-1$ results when measuring in the Y basis.
}
}

\exer{Now consider measuring two of the three qubits in the W state. You will need for projectors instead of just two.

\subexer{Calculate the probabilities of finding the $(+1,+1)$, $(+1,-1)$, $(-1,+1)$ and $(-1,-1)$ results when measuring in the Z basis.
}

\subexer{Calculate the probabilities of finding the $(+1,+1)$, $(+1,-1)$, $(-1,+1)$ and $(-1,-1)$ results when measuring in the X basis.
}

\subexer{Calculate the probabilities of finding the $(+1,+1)$, $(+1,-1)$, $(-1,+1)$ and $(-1,-1)$ results when measuring in the Y basis.
}

}


\end{exercises}

