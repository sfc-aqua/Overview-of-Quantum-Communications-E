\chapter{E91: Entanglement-based QKD}
\label{sec:10_E91}

Continuing with our study of quantum key distribution, in this Chapter we discuss an entanglement-based protocol, known as E91.
This protocol was introduced by Artur Ekert in 1991.
Unlike BB84, the E91 protocol relies on entanglement shared between Alice and Bob to establish a correlated secret key.
We will see that this difference offers Alice and Bob a very powerful tool when it comes to verifying the security of their key.



%%%%%%%%%%%%%%%%%%%%%%%%%%%%
\section{Introduction}
\label{sec:E91-introduction}
%%%%%%%%%%%%%%%%%%%%%%%%%%%%

In the previous chapter, we learned about the BB84 single photon-based quantum key distribution protocol.
However, that is not the end of the story for QKD.
In fact, we are more concerned with entanglement-based services than single-photon states in this book.
Before we get into the \textbf{\emph{how}} of entanglement-based QKD, let's look at one reason \textbf{\emph{why}}, beginning with a little review of BB84.

In BB84, Alice and Bob use a public quantum channel to establish a secret key.
Alice prepares qubits in four different states, chosen at random from $\{\ket{0}, \ket{1}, \ket{+}, \ket{-}\}$.
She transmits these states to Bob, who randomly measures them in either the $X$ or $Z$ basis.
After the measurement is finished, Alice and Bob exchange information about the preparation and measurement bases.
If the bases coincide, they keep the results for those measurements, forming the basis for their secret key.
If they dedicate a portion of the key material to eavesdropper detection, they can determine if anyone is listening in on the state, thanks to the non-orthogonality of the original encoded qubit states.

Let's consider how Eve's knowledge of the preparation bases affects the security of the protocol.
In Chapter~\ref{sec:9_bb84}, we saw that if Eve has no knowledge of Alice's preparation bases, the probability that her eavesdropping is detecting is $1/4$.
In the exercises following that Chapter, we saw that having some knowledge of the preparation bases gives Eve a better chance of staying undetected.

Let's consider the limit of this situation, where Eve has perfect knowledge of the preparation bases, as shown in Fig.~\ref{fig:eve-bb84}.
She intercepts the first qubit.
Because she knows that this qubit was prepared in the $Z$ basis, she measures it in the $Z$ basis and obtains the corresponding classical bit, and then she sends the qubit on to Bob\footnote{In many implementations, actually, she would create an identical photon that she sends on to Bob.}.
She intercepts the second qubit, and again because she knows the information about Alice's preparation basis, she measures in the appropriate basis, for example the $X$ basis.
In this case, if the state is \ket{-}, she obtains a classical bit one. She repeats this procedure for every qubit.
Although she is measuring these qubits, she is not disturbing them at all because she always measures in the same bases in which they were prepared.
In this way, she can build up a secret key that's perfectly correlated with the key that Alice and Bob end up sharing sharing.
\begin{figure}[t]
    \centering
    \includegraphics[width=0.8\textwidth]{lesson10/eavesdropping-on-bb84.png}
    \caption[Successful eavesdropping on BB84.]{If Eve can learn the qubit preparation bases for the BB84 photons, she can measure the qubits without disturbing them, remaining undetected as she forwards the qubits to Bob.}
    \label{fig:eve-bb84}
\end{figure}
This sneaky measurement attack by Eve is a big problem, causing the whole procedure of BB84 to fail.
Even though Alice and Bob can try to detect Eve as they would in the normal protocol, she has not disturbed any of the qubits.
Therefore, they will never detect her presence.

In this chapter, we will show that the E91 protocol is more robust against these kind of attacks.
The protocol relies on pre-shared entanglement between Alice and Bob.
We will assume that Alice and Bob can communicate over a classical channel, and also that there is some source of entangled states, as in Fig.~\ref{fig:e91-setup}.
This source generates multiple copies of an entangled state and distributes the qubits to Alice and to Bob.
We will see that in this protocol, even if Eve controls the source of the qubits, as in Fig.~\ref{fig:eve-e91}, the protocol still remains secure in the sense that Alice and Bob can easily detect an eavesdropping Eve.

\begin{figure}[H]
    \centering
    \includegraphics[width=0.8\textwidth]{lesson10/e91-setup.png}
        \caption[E91 setup.]{In E91, a source of Bell pairs distributes them to Alice and Bob.}
    \label{fig:e91-setup}
\end{figure}

\begin{figure}[H]
    \centering
    \includegraphics[width=0.8\textwidth]{lesson10/e91-eve-as-source.png}
    \caption[E91 with Eve as Bell pair source.]{Even if Eve controls the Bell pair source, Alice and Bob can maintain secure operation.}
    \label{fig:eve-e91}
\end{figure}





%%%%%%%%%%%%%%%%%%%%%%%%%%%%%%%%%%
\section{Basic ingredients}
\label{sec:E91-basic-ingredients}
%%%%%%%%%%%%%%%%%%%%%%%%%%%%%%%%%%

There are two basic ingredients to entanglement-based QKD protocols.
The first ingredient is the procedure for establishing a secret key.
For that purpose, we will use an entangled state of two qubits.
Let's consider the case where Alice and Bob are sharing the following Bell pair,
\begin{align}
    |\Psi^{+}\rangle = \frac{1}{\sqrt{2}}(\ket{01}+\ket{10}).
\end{align}
If we measure these qubits in the same basis, the outcomes will be
correlated or anti-correlated, depending on which basis is used to measure them, and the probability of these outcomes is uniformly random.

Let's look at an example to demonstrate how this measurement works.
Let's say that both Alice and Bob measure in the $X$ basis.
We can compute the probabilities of all four possible outcomes.
\begin{align}
\begin{aligned}
    \operatorname{Prob}\left\{\ket{++}_{A B}\right\} & = \frac{1}{2}, \quad  \operatorname{Prob}\left\{\ket{+-}_{A B}\right\} = 0, \\
    \operatorname{Prob}\left\{\ket{--}_{A B}\right\} & = \frac{1}{2}, \quad \operatorname{Prob}\left\{\ket{-+}_{A B}\right\} = 0.
\end{aligned}
\end{align}
The probability that both Alice and Bob obtain a correlated result of \ket{++} is given by a half.
The probability that they get a \ket{--} outcome is also a half,
leaving the other wiht no probability of being observed.

In this way, when Alice measures state \ket{+}, Bob always measures state \ket{+}.
When Alice measures state \ket{-}, Bob always measures state \ket{-}.
The classical bits that are the outcomes of their measurements will always be either 00 with probability a half, or 11 with the same probability.
In this way, they can establish a secret, random correlated key.

What if they measure in the $Z$ basis? The scenario is very similar, although now the results are anti-correlated,
\begin{align}
\begin{aligned}
    \operatorname{Prob}\left\{\ket{00}_{A B}\right\} & = 0, \quad \operatorname{Prob}\left\{\ket{01}_{A B}\right\} = \frac{1}{2}, \\ 
    \operatorname{Prob}\left\{\ket{11}_{A B}\right\} & = 0, \quad \operatorname{Prob}\left\{\ket{10}_{A B}\right\} = \frac{1}{2}.
\end{aligned}
\end{align}
The probability of correlated outcomes is zero this time.
When Alice and Bob both measure in the $Z$ basis, the outcomes will never be $|00\rangle_{AB}$ or $|11\rangle_{AB}$.
The outcomes are always anti-correlated, meaning that when Alice's outcome is $|0\rangle_A$, Bob will measure $|1\rangle_B$.
Similarly, when Alice measures $|1\rangle_A$, Bob's outcome will be $|0\rangle_B$.
An important thing to note is that both possibilities are equally probable, with $50\%$ probability they share $|01\rangle_{AB}$, and with the same probability they share $|10\rangle_{AB}$.
The corresponding classical keys are anti-correlated as well.
All that Bob has to do is flip his bits in order to obtain a random, correlated key, which can then be used to encrypt data for communication.

The second ingredient of en entanglement-based QKD protocol is to verify that they have an entangled state.
Why do they need to do this verification?
The first reason, as we just saw, is that entangled states can be used to generate a correlated random key, so it is crucial to confirm that we really have entanglement before we try to use it.
There is a very important second reason: entanglement can be used for security as well, in a manner that goes beyond what BB84 can achieve. Maximally entangled states are guaranteed to be secure due to \textbf{\emph{monogamy of entanglement}}\index{monogamy of entanglement}.

Monogamy of entanglement is a fundamental property of quantum states,
and it constrains how correlated multiple qubits can be.
In particular, if Alice and Bob share a maximally entangled state,
then we are guaranteed that they cannot share any correlations with a third party, such as Eve.
This makes monogamy of entanglement an important tool for QKD security.
If Alice and Bob can demonstrate and verify that they have a maximally entangled state, they are automatically demonstrating that whatever
key they establish is secure and Eve does not have any information about their secret key.

In general, there is a trade-off: if Alice and Bob share a non-maximally entangled state, they can still share some correlations with Eve.
The stronger the entanglement that they share, the less correlated they are with Eve, until the point where they
are maximally entangled and therefore they share no correlations with Eve.
\textbf{\emph{Stronger entanglement between Alice and Bob implies more a secure key between Alice and Bob.}}

How can Alice and Bob verify that they share a maximally entangled key?
This is done via the use of the \textbf{\emph{CHSH inequality}}\index{CHSH inequality}, which we hinted at back in Sec.~\ref{sec:chsh-game} when we introduced the CHSH game.

Let's start by considering four classical random variables, denoted $A$,
$\bar{A}$, $B$, and $\bar{B}$, each of which can have values $+1$ or $-1$.
Let's say that we form the function
\begin{equation}
    A(B+\bar{B})+\bar{A}(B-\bar{B})=\pm 2
\end{equation}
You can easily convince yourself that for any combination of $+1$ or $-1$, the maximum value that you can get for this expression is $+2$, and the minimum value that you can get is $-2$.

Now imagine that we are constantly generating these random variables.
We are interested in the average value of this function,
\begin{equation}
    |\langle A(B+\bar{B})\rangle+\langle\bar{A}(B-\bar{B})\rangle| \leq 2.
\end{equation}
The angular brackets $\langle\cdot\rangle$ denote the average value of the expression.
The two extremes values are $+2$ and $-2$.
The expectation value will therefore be somewhere in between, depending on the details of the probability distributions for these random classical variables.
We expand these expectation values, then take the absolute value of the whole sum, which is constrained to be less than or equal to two.
We get the following expression which we are going to denote by the symbol $\mathcal{S}$,
\begin{equation}
    \mathcal{S}=|\langle A B\rangle+\langle A \bar{B}\rangle+\langle\bar{A} B\rangle-\langle\bar{A} \bar{B}\rangle| \leq 2.
    \label{eq:chsh-inequality}
\end{equation}
This inequality is known as the CHSH inequality.
Any set of classical random variables $A$, $\bar{A}$, $B$, and $\bar{B}$, have to satisfy this constraint, even if $A$ and $B$ are classically correlated.

What happens in the quantum case?
We can consider $A$, $\bar{A}$, $B$, and $\bar{B}$ to be the measurement outcomes when state \ket{\psi} is measured in a certain basis.
Just to remind you, the expectation value of an observable where Alice measures observable $A$ and Bob measures observable $B$ is given by the expression
\begin{equation}
    \langle A B\rangle=\langle\psi|A \otimes B| \psi\rangle.
\end{equation}
Given that the measurement outcomes are still $\pm1$, we might expect that the CHSH inequality in Eq.~(\ref{eq:chsh-inequality}) applies to the quantum case too.

Amazingly, for some quantum states, we can violate the CHSH inequality.
By ``violating'', we mean that we can obtain a value $\mathcal{S}$ that is larger than 2.

In particular, in an experiment where we measure and compute these four expectation values and then we sum them up in this manner, if we obtain a CHSH expression which is less than two,
then we can say \textbf{\emph{maybe}} the states are classically correlated.
But, if we measure a CHSH expression which is larger than two, then we can, say that \textbf{\emph{definitely}} these states are entangled.
In quantum mechanics, the CHSH expression can go all the way up to a value of $2\sqrt{2}$, which happens for maximally entangled states.

\begin{figure}[H]
    \centering
    \includegraphics[width=0.8\textwidth]{lesson10/chsh-values.png}
    \caption{Possible CHSH values.}
    \label{fig:chsh-values}
\end{figure}

Let's consider a particular example.
Take one of the Bell pairs, \ket{\Psi^+}.
For the measurement settings, we consider the following: $A$ is the Pauli $Z$ basis, $\bar{A}$ is the $X$ observable, while $B$ and $\bar{B}$ are given by combinations of $Z$ and $X$, giving us rotated measurement bases $B=\frac{1}{\sqrt{2}}(Z-X)$ and $\bar{B}=\frac{1}{\sqrt{2}}(Z+X)$.

We can go through the algebra of computing the expectation values.
We find that for a maximally entangled state, we obtain the CHSH expression of $\mathcal{S}=2\sqrt{2}$.
Keep in mind that this is a statistical measure; we have to produce and measure a lot of copies of the state \ket{\psi} to estimate each of the four averages in Eq.~\ref{eq:chsh-inequality}.
The four measurement bases and four possible outcomes mean that the calculation requires sixteen values.
An example set of values is shown in Tab.~\ref{tab:chsh-calculation}.  Each entry $P_{ij}$ is the probability that the measurement basis on the left produces the outcomes $i$ at Alice and $j$ at Bob, normalized from counts so that the four $P_{ij}$ values in each line sum to 1.
For this set of values,
\begin{equation}
\begin{aligned}
    \mathcal{S} &= |\langle A B\rangle+\langle A \bar{B}\rangle+\langle\bar{A} B\rangle-\langle\bar{A} \bar{B}\rangle| \\
    &= | -0.71 -0.71 - 0.65 - 0.65 | \\
    &= 2.72 \geq 2,
\end{aligned}
\label{eq:chsh-example}
\end{equation}
giving us a strong CHSH inequality violation.

\begin{table}[t]
    \setcellgapes{3pt}
    \renewcommand\theadfont{}
    \makegapedcells
    \centering
    \begin{tabular}{cccccc}
        \hline
        & \boldmath$P_{++}$ & \boldmath$P_{+-}$ & \boldmath$P_{-+}$ & \boldmath$P_{--}$ & \boldmath$\langle MN \rangle$ \\
        \hline
        \boldmath$AB$ & 0.04 & 0.26 & 0.60 & 0.10 & -0.71 \\
        \boldmath$A\bar{B}$ & 0.04 & 0.26 & 0.60 & 0.10 & -0.71 \\
        \boldmath$\bar{A}B$ & 0.16 & 0.34 & 0.48 & 0.02 & -0.65 \\
        \boldmath$\bar{A}\bar{B}$ & 0.16 & 0.34 & 0.48 & 0.02 & 0.65 \\
        \hline
    \end{tabular}
    \caption[Calculation of the $\mathcal{S}$ value.]{Calculation of the CHSH $\mathcal{S}$ value. $P_{++}$ is the probability of getting the $+1$ eigenstate for both measurements when measuring in the basis in the left column, etc. $\langle MN \rangle$ is the expectation value calculated using $\langle MN \rangle = \bra{\psi}M\otimes N\ket{B} = P_{++} + P_{--} - P_{+-} - P_{-+}$, where $M$ is either $A$ or $\bar{A}$, and $N$ is $B$ or $\bar{B}$,  giving the bases in the left column.}
    \label{tab:chsh-calculation}
\end{table}

This calculation gives us a way of verifying entangled states, and particularly verifying maximally entangled states, a very important condition for the entanglement-based QKD protocol.
If we can demonstrate that we violate the CHSH inequality maximally, with a value of $S = 2\sqrt{2}$, then we can certify that
the state Alice and Bob share is a maximally entangled state.
If the states we are making are maximally entangled, then the principle of monogamy of entanglement\index{monogamy of entanglement} tells us that they are \textbf{\emph{not}} correlated with Eve, and therefore we can guarantee the security of Alice and Bob's secret key.



%%%%%%%%%%%%%%%%%%%%%%%%%
\section{Protocol}
\label{sec:E91-protocol}
%%%%%%%%%%%%%%%%%%%%%%%%%

We have described the two basic ingredients of the E91 protocol.
Let's put them together.
The setting is the following: Alice and Bob can communicate over a classical channel, and they share multiple copies of a maximally entangled state \ket{\Psi^+}.
These copies can be generated by Eve herself.

Alice and Bob randomly choose a measurement basis in which they measure their qubits.
Alice chooses from three measurement bases, as shown in Fig.~\ref{fig:e91-bases}.
The circle represents the $XZ$ plane of a Bloch sphere.
Alice's measurement setting or measurement basis $A_1$ corresponds to measurement in the $Z$ basis.
If she chooses this basis, she projects the state either onto \ket{0} or onto \ket{1}.
She can also measure in the $A_2$ basis, which corresponds to $X$ basis, given by the horizontal direction.
She can also measure in a rotated basis $A_3=\frac{1}{\sqrt{2}}(Z+X)$, which is a linear combination of $Z$ and $X$.
Bob can also measure in the $Z$ basis, given by $B_1$, or in the rotated basis $B_2 = \frac{1}{\sqrt{2}}(Z-X)$, or in the basis $B_3=\frac{1}{\sqrt{2}}(Z+X)$.

\begin{figure}[H]
    \centering
    \includegraphics[width=0.7\textwidth]{lesson10/10-3_bases.pdf}
    \caption[E91 measurement bases.]{E91 basis choices for Alice and Bob. The circle is the $XZ$ plane of the Bloch sphere.}
    \label{fig:e91-bases}
\end{figure}

Why do we have three different measurements for Alice and three different measurements for Bob rather than two, like we had in the previous protocol, BB84?

Notice that some of these measurement bases overlap.
If both Alice and Bob measure the entangled state in the same basis, they can use the classical outcomes to generate and establish a classical, correlated random key.
This follows similar logic to BB84.
Data from the basis choices $(A_1, B_1)$ or $(A_3,B_3)$ can be used for key generation.
On the other hand, we need some rotated bases in order to compute the CHSH expression $\mathcal{S}$, and see if it violates the classical CHSH inequality.
Violation of the inequality establish that Alice and Bob are really sharing an entangled state.
For this calculation, we use the basis choices $(A_1,B_3)$, $(A_1,B_2)$, $(A_2,B_2)$, $(A_2,B_3)$.

\begin{figure}[H]
    \centering
    \includegraphics[width=0.8\textwidth]{lesson10/10-3-bases_key.pdf}
    \caption[E91 example of a series of basis choices for measurements.]{E91 example of a series of basis choices for measurements. The $A_1$ and $A_3$ bases for Alice and $B_1$ and $B_3$ bases for Bob can be used to generate key bits when Alice and Bob select the same basis, as shown with the green check marks.}
    \label{fig:e91-example}
\end{figure}

In order to establish the key, Alice measures either in $A_1$ or $A_3$, and Bob measures in $B_1$ or $B_3$. They randomly measure their multiple copies of entangled states, and then they exchange information about the basis of their measurements. 
They exchange the information about these bases and look at the places where their measurement basis choices coincide, as shown in Fig.~\ref{fig:e91-example}.
Each partner then keeps the measurement outcome as a bit in their shared secret.

Those two cases where Alice and Bob chose the same measurement basis take care of generating the key.
In some other cases, they will not measure in bases that coincide.
They don't discard these results, but instead  use them to compute the CHSH expression and check for a violation of the classical bound on the CHSH inequality.
In particular, they look for scenarios where they chose $(A_1,B_2), (A_1,B_3), (A_2,B_2),$ or $(A_2,B_3)$ as shown in Fig.~\ref{fig:10-3_e91_example_CHSH}.
Using these measurement outcomes, Alice and Bob then calculate the CHSH correlation function $\mathcal{S}$,
\begin{equation}
    \mathcal{S} = |\langle A_1B_2\rangle + \langle A_1B_3\rangle + \langle A_2B_2\rangle - \langle A_2B_3\rangle|.
    \label{eq:CHSH-inequality-basis}
\end{equation}
\begin{figure}[t]
    \centering
    \includegraphics[width=0.7\textwidth]{lesson10/10-3_bases_CHSH.pdf}
    \caption[E91 - CHSH bases.]{When the basis choices don't coincide the measurement outcomes can be still used for the CHSH test.}
    \label{fig:10-3_e91_example_CHSH}
\end{figure}

This way, they don't need to discard information as was done in BB84. They get to use the information to calculate either the secret correlated key or the CHSH violation.
If they obtain a CHSH correlation function such that $\mathcal{S} \leq 2$, they say ``okay, we cannot conclude whether or not we have an entangled state, but it's safer to just abort.''
If they have $\mathcal{S} > 2$, then they conclude, ``yes, we are sharing an entangled state, therefore we can proceed with the protocol.''
Remember, we said that monogamy of entanglement ensures that if they have an entangled state, then Eve cannot be not strongly correlated with either of them.
In particular, if Alice and Bob have a maximally entangled state, then Eve is not correlated with them at all.

So far, we have considered the case where everything was ideal, with no noise.
But what happens in real life, where noise is always present? How does noise affect the E91 protocol?
In real life, the CHSH value will not equal exactly $2\sqrt{2}$, as we saw in Tab.~\ref{tab:chsh-calculation}.
Moreover, Alice and Bob will not be able to generate a perfectly correlated key, meaning that either noise or the tinkering of Eve will introduce some inconsistencies into the key.

Alice and Bob have to decide on the acceptable security risk even if the keys are not perfectly correlated.
They have to agree, ``Okay, if the correlation is not one hundred percent, but it's very close, we can still use this to do something useful.''
If they agree to this in principle, then they have to engage in two more protocols.
One is called  \textbf{\emph{information reconciliation}}\index{information reconciliation}, which takes the initial secret key that is not perfectly correlated and produces a more correlated key.
It increases the correlation between the secret bit strings obtained by Alice and Bob.
As cryptosystems are generally designed so that a single bit difference in the keys changes half of the bits in the encrypted message~\footnote{A characteristic sometimes called the \textbf{\emph{avalanche effect}} or \textbf{\emph{strict avalanche criterion}} by cryptographers.}, they require that exactly the same key be used at both ends.
Alice and Bob can also perform something known as \textbf{\emph{privacy amplification}}\index{privacy amplification}, where they take their generated secret key and produce a shorter, more secure key.
Privacy amplification is a procedure that attempts to eliminate any possible correlations with Eve.

Having cover the basics of both single-photon and entanglement-based QKD, can we say which protocol is better more secure?
BB84 is based on the indistinguishability of single photons prepared in non-orthogonal bases to discover an eavesdropper, while E91 uses quantum correlations between two qubits to bound the information that is potentially compromised.
Let's compare the two protocols from the point when the secret key is generated.

In the case of BB84, candidate bits for the key are generated when Alice generates her random string $a$ right at the beginning of the protocol.
Remember, she generates two random $n$-bit strings.
String $b$ encodes the information about the basis of preparation, and string $a$ encodes the bits themselves.
If Alice chooses the $Z$ basis, the state she prepares is a \ket{0} or a \ket{1}, whereas if she chooses the $X$ basis, she prepares a \ket{+} or a \ket{-}.
The secret key (or an extended form of it before photons are lost and their corresponding bits are discarded) exists right from the beginning, before any communication between Bob and Alice takes place.

In BB84, a clever Eve can find a way to obtain some information about this secret bit string.
In particular, we can consider a very paranoid scenario where the random number generator (RNG) that Alice is using to generate a random bit string was built by Eve herself.
Eve constructed the RNG in such a way that any information about the random bit string that the device produces gets passed on to Eve iwthout Alice's knowledge.
We saw at the beginning of this chapter that this weakness poses a huge security risk for the BB84 protocol.

In contrast, in the E91 protocol, the secret key is really generated after the entangled pairs of qubits are measured.
The key is not produced when the entangled qubit pairs are created, nor when they arrive at Alice and Bob, but only after Alice and Bob measure them in their random bases.
Even if Eve controls the generation and distribution of entangled pairs, she cannot defeat E91.
\emph{However}, if Eve controls or can predict the output of the RNGs that Alice and Bob use for basis selection, she can defeat E91.
Coupled with the ability to measure the entangled states in the quantum channel before Alice and Bob receive them, she can measure only the key pairs and recover the key in parallel with Alice and Bob, undetected. This process is harder for Eve, so perhaps we can say that E91 is more secure than BB84.

Let's conclude this chapter by talking about some entanglement-based QKD experiments.
We saw in the discussion of BB84 that there were network testbeds for single photon QKD networks.
The development of entanglement-based QKD is not as advanced, but it exists at the level of establishing a secret key over a single link.
One such experiment was performed over free space, meaning that the entangled photons traveled through through air.
This experiment was done over a distance of 144 kilometers between La Palma and Tenerife, two islands in the Canary Islands.
Photon pairs were produced by the spontaneous parametric down-conversion (SPDC)\index{spontaneous parametric down-conversion (SPDC)} process which we saw in Sec.~\ref{sec:4-4_spdc}.
The qubits were encoded in the polarization of the photons.
The obtained CHSH value, $\mathcal{S} =2.508$, was quite a substantial amount above the maximum classical value of 2, giving a strong CHSH violation.

A different, more recent experiment was done over a distance of hundreds of kilometers, but in a lab over optical fiber wound around a spool.  Two distances were tested.
The first distance was 311 kilometers over standard fiber, and the second distance was 404 kilometers over an ultra-low loss fiber.
In this experiment, the obtained bit rate for the secret key was of the order of $10^{-3}$ bits per second for the shorter distance, or $10^{-4}$ bits per second for the longer distance. These bit rates don't actually include the information reconciliation and privacy amplification parts of a full protocol.
If we wish to use this scheme in real life, we would have to add these functions, further lowering the bit rate.

Another fantastic experiment was performed with satellites, where the satellite was use to distribute entangled pairs between two ground stations 1,120 kilometers apart.
Remember, we said that the light travels in a straight line, so by using a satellite we can overcome the complication of curved earth's surface (and reduce the atmospheric disturbance as well) to establish a quantum key over much longer distances.
The measured CHSH value was $\mathcal{S} = 2.56$, again showing a strong CHSH violation.
The obtained bit rate was 0.12 bits per second.
One could ask the question, what if we actually used a single fiber to connect these two ground stations?
Well, the paper that reported these results estimated that the fiber would have been around eleven orders of magnitude less efficient than using the satellites.
At the beginning of this chapter, we discussed the security-related reason for preferring E91 to BB84; arguably, an even bigger reason is simply the loss in fiber, which we will address in Ch.~\ref{sec:11_long-distance} and beyond.

Thus concludes our discussion of quantum key distribution protocols as a key use of quantum effects in communication systems and this block of lessons.
In the final two blocks, we will investigate the construction of quantum repeaters and networks.





\newpage
\begin{exercises}
\exer{
\emph{More measurement bases in E91.}
In E91, Alice and Bob each have three choices of measurement basis, which they are expected to select randomly.  We described how six different pairs of choices are used to generate the secret key and to calculate the CHSH inequality to check for the presence of an eavesdropper.
\subexer{
Sharp-eyed readers may have recognized that, in the explanation, we neglected three choices of basis.  Which ones did we not mention, and why?  What should be done with the data collected when those bases are chosen?
}
\subexer{Taking into account the answer to the prior question, what fraction of total measurements are dedicated to key generation, and what fraction to CHSH testing?}
}

\exer{
\emph{Optimality of CHSH violation.}
On the Bloch sphere, the four basis choice angles for CHSH are 45\degree apart.
In this exercise we will explore why we measure at these angles and not some other ones.
\subexer{
Let's fix Alice's measurements to be the same like in Eq.~(\ref{eq:CHSH-inequality-basis}).
But Bob's measurement basis can be different. It is at some angle $\theta$ as shown in the figure below.
Write down the two observables, $B_1$ and $B_2$, that Bob measures in terms of the angle $\theta$.
}
\subexer{
Write down the four expectation values $\langle A_1B_1\rangle$, $\langle A_1B_2\rangle$, $\langle A_2B_1\rangle$, and $\langle A_2B_2\rangle$ in terms of the angle $\theta$.
}
\subexer{
Construct the CHSH value and find the angle $\theta$ that maximizes it.
}
}

\exer{
\emph{E91 with other Bell pairs.}
We described E91 using \ket{\Psi^+} Bell pairs.
Let's explore what happens when we use one of the other Bell pairs,
\begin{equation}
    |\Phi^+\rangle = \frac{1}{\sqrt{2}} (\ket{00} + \ket{11}).
\end{equation}
\subexer{
Compute the CHSH value $\mathcal{S}$ given in Eq.~(\ref{eq:CHSH-inequality-basis}). Did you expect this value?
}
\subexer{
Write a new CHSH expression tailored to the new Bell pair $|\Phi^+\rangle$. Does it recover the expected violation of $\mathcal{S}=2\sqrt{2}$?
}
\subexer{
What is maximum value of this new CHSH value if Alice and Bob only classical states?
}
\subexer{
Write down a CHSH inequality for the state $|\Psi^-\rangle$.
}
}

\exer{
\emph{CHSH with noisy states.}
Typically, the distributed entangled states will not be pure.
Let's consider an imperfect source of entangled states, which produces the desired state \ket{\Psi^+} with probability $\operatorname{Pr}(\ket{\Psi^+})=1-p_{\epsilon}$.
With equal probability the source produces one of the other Bell pairs, $\operatorname{Pr}(\ket{\Phi^+})=\operatorname{Pr}(\ket{\Phi^-})=\operatorname{Pr}(\ket{\Psi^-})=p_{\epsilon}$.
\subexer{
Write down the density matrix $\rho$ of the two-qubit state that the source generates in Dirac notation.
}
\subexer{
Compute the CHSH value $\mathcal{S}$ for the desired output state \ket{\Psi^+} in terms of the error parameter $p_{\epsilon}$.
}
\subexer{
Compute the fidelity $F$ of the mixed state $\rho$.
}
\subexer{
Plot the CHSH value $\mathcal{S}$ and fidelity $F$ as functions of the error parameter $p_{\epsilon}$.
}
\subexer{
At what value of the error parameter $p_{\epsilon}$ can we no longer guarantee that the state is entangled? What is the fidelity of such a state?
}
}

% \exer{
% If the success probability for transmitting a photon through $1,120$km of fiber is $10^{-14}$, determine the loss per kilometer, in dB.
% }
\end{exercises}




\newpage
\section*{Quiz}
\addcontentsline{toc}{section}{Quiz}

A quiz is available in the online learning system.

% \section{Learning more}

\section*{Further reading for chapters 8-10}
  \addcontentsline{toc}{section}{Further reading for chapters 8-10}

Anton Zeilinger, Alain Aspect and John Clauser were awarded the 2022 Nobel Prize in Physics for their experimental demonstrations of the existence of entanglement, with teleportation being referenced indirectly.

A good, popular science treatment of the history of this area is \emph{The Age of Entanglement}, by Louisa Gilder~\cite{gilder08:_age_of_entanglement}.

{\bf Chapter 8}

This chapter introduced one the most fundamental protocols of quantum communication, teleportation. “Mike \& Ike” Chapter 1 goes through the mathematics of the protocol. The original paper is a great read and we highly recommend it:

Charles H. Bennett, Gilles Brassard, Claude Crépeau, Richard Jozsa, Asher Peres, William K. Wootters, Teleporting an unknown quantum state via a dual classical and Einstein-Podolski-Rosen Channels, \emph{Physical Review Letters} 70, 1895 (1993)~\cite{bennett:teleportation}.

{\bf Chapter 9}

An enlightening discussion of the BB84 protocol can be found in Chapter 12 of “Mike \& Ike” along with some exercises that will deepen your understanding of the protocol.
The original 1984 paper can be found here, along with modern commentary:

Charles H. Bennett, Gilles Brassard, Quantum cryptography: Public key distribution and coin tossing, \emph{Theoretical Computer Science} 560, 7 (2014)~\cite{bennett:bb84}.

For a further discussion of classical cryptography aimed at those working in quantum computing, see the forthcoming paper "What Every Quantum Researcher and Engineer Should Know about Classical Cryptography", by Van Meter and Aono.  Portions of this paper are available on Van Meter's blog.

The Chinese QKD network is described in ~\cite{chen2021integrated}.

{\bf Chapter 10}

Entanglement-based QKD was first introduced in:

Artur K. Ekert, Quantum cryptography based on Bell’s theorem, \emph{Physical Review Letters} 67, 661 (1991)~\cite{ekert1991qcb}.

This paper is unfortunately behind a paywall but you should be able to access it through your university’s library system.
A brief discussion of entanglement-based QKD can be also found in Chapter 12 of “Mike \& Ike”.

The exact security statistics necessary for effective QKD protocols, whether BB84 or E91, are a complex matter.  Information reconciliation and privacy amplification are major research topics in their own right.
Fantastic introduction to all these topics as well as other parts of quantum cryptography can be found in the following new book:

Thomas Vidick, and Stephanie Wehner, \emph{Introduction to Quantum Cryptography} (2023)~\cite{vidick_wehner_2023}.