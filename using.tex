\chapter*{Using This Book}

\section*{In a Degree Curriculum}

This book represents about eleven hours of online video; coupled with homework assignments, it could be twenty to thirty hours of work for well-prepared students (and substantially more for not well-prepared students who build their understanding along the way).  As such, it is intended to present one credit toward graduation by Japanese standards, where an undergraduate degree is typically 124 credits over eight semesters. As typical courses are two credits, this content is expected to be paired with additional material in a Japanese curriculum to form a for-credit course. In the United States or elsewhere where the typical course is a larger chunk of learning, this material may form a third or a quarter of a course.

In conjunction with the videos and this book, we recommend using the material in an active learning context, such as a \emph{flipped classroom} environment.

Additional details, such as the recommended mathematics background, are covered in Sec.~\ref{sec:mod-over}.

\section*{In a Tutorial}

It is also possible to take a subset of this material and use it in a half-day or full-day tutorial, with appropriate background preparation.  For example, at the Third Workshop for Quantum Repeaters and Networks, held in Chicago in August 2022, we asked attendees to prepare by watching the following subset of the videos:
\begin{itemize}
\item QSI Seminar: Prof. Rodney Van Meter, Keio University, Engineering the Quantum Internet, 30/06/2020
At minimum, watch from 9:45 to 27:30 on 
“Applications of a Quantum Internet”
\item 2 Quantum States
\begin{itemize}
\item 2-1 Qubits
\item 2-2 Unitary operations				
\item 2-3 Measurement
\item 2-4 Probabilities, expectation, variance
\item 2-5 Multiple qubits
\end{itemize}
\item 3 Pure and Mixed States
\begin{itemize}
\item 3-1 Noisy world
\item 3-2 Outer product
\item 3-3 Density matrices
\item 3-4 Pure vs mixed states
\item 3-5 Fidelity
\end{itemize}
\item 4 Entanglement
\begin{itemize}
\item 4-1 CHSH Game
\item 4-2 Entangled states
\item 4-3 Bell states
\item 4-4 SPDC
\item 4-5 Entanglement as a resource
\end{itemize}
\item 8 Teleportation
\begin{itemize}
\item 8-1 Introduction to teleportation
\item 8-2 Teleportation protocol
\item 8-3 No-cloning theorem and FTL communication
\end{itemize}
\item 12 Quantum Repeaters
\begin{itemize}
\item 12-1 The need for repeaters
\item 12-2 Making link-level entanglement
\item 12-3 Reaching for distance: Entanglement swapping
\item 12-4 Detecting errors: purification
\item 12-5 Making a network 
\end{itemize}
\end{itemize}

At the workshop, we conducted a four-hour session consisting primarily of hands-on work.  The hands-on work demonstration of the three key concepts of \emph{teleportation}, \emph{entanglement swapping}, and \emph{purification}. The Qiskit Jupyter notebook and QuISP (Quantum Internet Simulation Package) demos used in the hands-on session are also available on the web. \rdv{Add links and maybe even actual text/code as an appendix?}
