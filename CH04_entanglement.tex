\chapter{Entanglement}
\label{sec:4_entanglement}
%\begin{abstract}

In this chapter, we will finally learn about \textbf{\emph{entanglement}}: a special and unique feature of quantum mechanics.
We will begin by demonstrating how strange and counter-intuitive entanglement can be.
Then, we will move on to the definition of entangled states.
Following that, we will discuss a particular class of entangled states called ``Bell states'', which play an important role in quantum communication.
Finally, we will finish by framing entanglement as a resource for computational and communication tasks.

%\end{abstract}

\section{CHSH Game}
\label{sec:chsh-game}
\index{CHSH game}

We begin with a game that will demonstrate just how wonderful and strange entangled states are.
The game is related to a proposal of four scientists named Clauser, Horn, Shimony, and Holt, though it is normally referred to as the \textbf{\emph{CHSH game}}\index{CHSH game}.
It consists of two players, labelled $A$ and $B$, and a referee denoted by $R$, as shown in Fig.~\ref{fig:chsh-game}.
The rules of the game are the following:
\begin{itemize}
    \item Referee $R$ generates two random bits, $x,y\in\{0,1\}$. Bit $x$ is sent to player $A$, and bit $y$ is sent to player $B$.
    \item Upon receiving the referee's bits, players $A$ and $B$ reply with their own bits. Player $A$ sends back a bit $a$, and player $B$ sends back bit $b$.
    \item Referee $R$ checks whether the players' bits satisfy the winning condition,
    \begin{equation}
        x\cdot y = a \oplus b,
    \end{equation}
    where $\oplus$ represents binary addition. This concludes a single round of the game.
    \item The game continues to see what percentage of the rounds the players can win.
    \item IMPORTANT: The players are not allowed to communicate once the game has started.
\end{itemize}

% game diagram
\begin{figure}[t]
    \centering
    % \includegraphics[width=0.5\textwidth]{lesson4/CHSH_diagram.pdf}
    \begin{tikzpicture}[->, >=stealth', auto, semithick, node distance=3cm]
    \tikzstyle{every state}=[draw=black,thick,text=black,scale=1]

    \node[state,fill=red!50]    (R)                     {R};
    \node[state,fill=newgreen!50]   (A)[below left of=R]    {A};
    \node[state,fill=newgreen!50]   (B)[below right of=R]   {B};

    \path
    (R) edge[bend right,above,red!60,thick]	node[yshift=1pt]{$x$}	(A)
    (R) edge[bend left,above,red!60,thick]	node[yshift=-1pt,xshift=1pt]{$y$}	(B)
    (A) edge[bend right,left,newgreen,thick]	node[yshift=1pt]{$a$}	(R)
    (B) edge[bend left,right,newgreen,thick]	node[yshift=2pt]{$b$}	(R);

    \end{tikzpicture}
    
    \caption[CHSH game.]{CHSH game view. The players A and B attempt to collaborate to beat the referee R.}
    \label{fig:chsh-game}
\end{figure}

The players cannot communicate in order to make the game interesting, otherwise they would win every single round.
This does not mean that the players are not allowed to communicate at all.
Before the game starts, they can agree on a common strategy in order to maximize their winning chances.

Before discussing what an optimal strategy would look like, let's have a look at a few rounds of the CHSH game to become more comfortable with the rules.
\begin{table}[h]
    \setcellgapes{5pt}
    \renewcommand\theadfont{}
    \makegapedcells
    \centering
    \begin{tabular}{cccccccc}
        \hline
        & $x$ & $y$ & $a$ & $b$ & $x\cdot y$ & $a\oplus b$ \\
        \hline
        \textbf{Round 1} & 0 & 1 & 0 & 1 & 0 & 1 & \textcolor{darkred}{Loss} \\
        \textbf{Round 2} & 0 & 1 & 1 & 1 & 0 & 0 & \textcolor{newgreen}{Win} \\
        \textbf{Round 3} & 1 & 1 & 1 & 0 & 1 & 1 & \textcolor{newgreen}{Win} \\
        \hline
    \end{tabular}
    \caption[CHSH game example.]{Three example rounds of the CHSH game.}
    \label{tab:4-1_chsh_rounds}
\end{table}
Table~\ref{tab:4-1_chsh_rounds} shows three example rounds of the game.
In Round 1, the referee $R$ generates the input bits with values $x=0$ and $y=1$, which are sent to the players.
The players generate their answers according to a pre-agreed strategy, and reply with $a=0$ and $b=1$.
The product of the input bits is $x\cdot y=0$, while the binary sum of the output bits is $a\oplus b=1$.
The referee determines that winning condition is not satisfied, and the players lose this round.
In Round 2, the product of input qubits is $x\cdot y=0$, and so is the binary sum of the outputs, $a\oplus b=0$. The players win this round.
Similarly for Round 3 where $x\cdot y=a\oplus b=1$.

So what is the optimal strategy that players $A$ and $B$ should follow in order to maximize their chances of winning?
We will now address this important question.
Let's begin by looking at the possible input bit pairs that the referee $R$ can generate.
There are four possible pairs, as shown in Tab.~\ref{tab:4-2_chsh_strategy}, and they are $(0,0)$, $(0,1)$, $(1,0)$, and $(1,1)$.
\begin{table}[h]
    \setcellgapes{5pt}
    \renewcommand\theadfont{}
    \makegapedcells
    \centering
    \begin{tabular}{cccccc}
        \hline
        $x$ & $y$ & $x\cdot y$ & $a=b$ & $a\oplus b$ & \\
        \hline
        0 & 0 & 0 & 0 & 0 & \textcolor{newgreen}{Win} \\
        0 & 1 & 0 & 0 & 0 & \textcolor{newgreen}{Win} \\
        1 & 0 & 0 & 0 & 0 & \textcolor{newgreen}{Win} \\
        1 & 1 & 1 & 0 & 0 & \textcolor{darkred}{Loss} \\
        \hline
    \end{tabular}
    \caption[CHSH game classical strategy.]{An example strategy for the CHSH game. Players $A$ and $B$ always reply with 0, regardless of the values of the input bits sent by the referee $R$.}
    \label{tab:4-2_chsh_strategy}
\end{table}
In the first three cases, the product of the input bits is $x\cdot y=0$. Only in the last case, when $x=y=1$, is the product equal to 1.
Realizing this, the players can agree on a strategy where they always output $a=b=0$, regardless of the value of the input bits sent by the referee.
This strategy results in the output bit binary sum $a\oplus b=0$, meaning the players win the CHSH game in three out of the four possible cases.
Since the input bits are generated uniformly at random, the probability of of a particular input pair $(x,y)$ is 1/4.
Therefore, the players have a 75\% chance of winning every round.
Not bad for such a simple strategy.

But can the players do better than that?
The answer is no, if the players are restricted to using only classical strategies.
However they can do better if they use pre-shared \textbf{\emph{entangled states}}\index{entangled state}.
A particular quantum strategy is depicted in Fig.~(\ref{fig:chsh-game}).
Players $A$ and $B$ now share an entangled state,
\begin{equation}
    |\Phi^+\rangle = \frac{1}{\sqrt{2}} (|00\rangle + |11\rangle),
    \label{eq:4_1-phiPlus}
\end{equation}
where the first qubit corresponds to qubit $A$, while the second qubit to corresponds to qubit $B$.
\footnote{We will discuss what it means for a quantum state to be entangled in the following section.
For now, we will just accept that Eq.~(\ref{eq:4_1-phiPlus}) is entangled.
In fact, it is good to get used to this state because we will keep encountering it again and again.}
The capital Greek letter $\Phi$ is pronounced ``Phi''.
If the referee's input bit is $x=0$, player $A$ measures qubit $A$ in the $Z$ basis.
If $x=1$, player $A$ measures in the $X$ basis.
For +1 measurement outcome, player $A$ replies with $a=0$.
If the measurement outcome is -1, player $A$ replies with $a=1$.
Similar procedure applies to player $B$.
Only difference is that player $B$ measures in a \textbf{\emph{rotated basis}}\index{rotated basis}.
If the input bit is $y=0$, player $B$ measures in the $(Z+X)/\sqrt{2}$ basis.
For $y=1$, player $B$ measures in the $(Z-X)/\sqrt{2}$ basis.

% insert quantum diagram
\begin{figure}[t]
    \centering
    % \includegraphics[width=0.5\textwidth]{lesson4/CHSH_quantum_diagram.pdf}
    \begin{tikzpicture}[auto, semithick, node distance=3cm]
    \tikzstyle{every state}=[draw=black,thick,text=black,scale=1]

    \node[state,fill=red!50]    (R)                     {R};
    \node[state,fill=newgreen!50]   (A)[below left of=R]    {A};
    \node[state,fill=newgreen!50]   (B)[below right of=R]   {B};

    \path
    (R) edge[bend right,above,red!60,thick,->,>=stealth']	node[yshift=1pt]{$x$}	(A)
    (R) edge[bend left,above,red!60,thick,->,>=stealth']	node[yshift=-1pt,xshift=1pt]{$y$}	(B)
    (A) edge[bend right,left,newgreen,thick,->,>=stealth']	node[yshift=1pt]{$a$}	(R)
    (B) edge[bend left,right,newgreen,thick,->,>=stealth']	node[yshift=2pt]{$b$}	(R);

    \coordinate (qubitA) at ($(A)+(0,-1)$);
    \coordinate (qubitB) at ($(B)+(0,-1)$);

    \draw[decorate,decoration={coil,aspect=0},thick] (qubitA)  -- (qubitB);
    \filldraw[black!30,draw=black,thick] (qubitA) circle (5pt);
    \filldraw[black!30,draw=black,thick] (qubitB) circle (5pt);

    \node[] at (-3, -3.15) {qubit A};
    \node[] at (3, -3.15) {qubit B};
    \node[] at (0, -3.5) {$|\Phi^+\rangle$};

    \end{tikzpicture}
    
    \caption[Optimal quantum CHSH game strategy.]{Optimal quantum strategy for the CHSH game requires the players to pre-share an entangled state of two qubits.}
    \label{fig:chsh-game}
\end{figure}

The above quantum strategy is a lot more complicated that the optimal classical one.
Actions of the players now depend on the inputs received from the referee.
They also need to pre-share one entangled for every round they play.
All of this extra work is well worth it however.
By following this quantum strategy, the players have probability of winning a round of 85\%.
This is an increase of 10\% compared to the optimal classical strategy.
We do not yet have the necessary tools to analyze the quantum strategy quantitatively in order to derive the winning probability.
We are about to remedy that in the remainder of this chapter.



%%%%%%%%%%%%%%%%%%%%%%%%%%%%%%%%%%%%%%%%%
\section{Entangled states}
\label{sec:4_2-entagnled_states}
%%%%%%%%%%%%%%%%%%%%%%%%%%%%%%%%%%%%%%%%%

Having seen how entangled states can be useful in the CHSH game, it is time to dicuss them in more detail.
To keep things simple, we will consider a system of two qubits.
We will use the term \textbf{\emph{local state}}\index{local state} to refer to the state of either of two qubits alone, and the term \textbf{\emph{global state}}\index{global state} to refer to the two qubits together.~\footnote{These terms can easily be extended to discuss more than one qubit in each location.}

We saw in Sec.~\ref{sec:multi-qubit} that in order to describe states of many qubits, we have to use the tensor product. For the case of the two qubits $A$ and $B$, we have the local state $\ket{\psi}_A$ of the qubit $A$, and a local state $\ket{\psi}_B$ of qubit $B$.
For concreteness, let's say that they are $\ket{0}$ and $\ket{0}$.
In order to write the global state of both qubits, we form the tensor product and we call this state $\ket{\psi}_{AB}$, which in this case is simply $\ket{00}$.
We can also consider a different state where $A$ is still in $\ket{0}$ but $B$ is now in the $\ket{+}$, a superposition of zero and one.
Again, the global state is straightforward: form the tensor product of $\ket{0}$ and $\ket{+}$,
\begin{equation}
\ket{\psi}_{AB} = \ket{0}\ket{+} = \ket{0}\left(\frac{\ket{0} + \ket{1}}{\sqrt{2}}\right) = \frac{\ket{00} + \ket{01}}{\sqrt2}.
\end{equation}

The above two examples both started with local states, which were then used to formulate the global state.
We can also ask the reverse question: given the global state, how do we write the local states of the qubits?
We can consider the global state of two qubits that we have encountered in the CHSH game in the previous section,
\begin{equation}
    \ket{\Phi^+}_{AB} = \frac{1}{\sqrt2}(\ket{00} + \ket{11})
    \label{eq:entangled_state}
\end{equation}
Identifying the local states is not that straightforward in this case.
By looking at Eq.~(\ref{eq:entangled_state}), we can see that the local state of either qubit is not quite $|0\rangle$, and it is also not quite $|1\rangle$.

At this point, it is better to get a bit more rigorous.
Let's write the local states as general pure states,
\begin{equation}
    |\psi\rangle_A = a_0|0\rangle + a_1|1\rangle, \quad |\psi\rangle_B = b_0|0\rangle + b_1|1\rangle.
\end{equation}
Taking the tensor product of $\ket{\psi}_A$ with $\ket{\psi}_B$, we arrive at the following two-qubit state, 
\begin{align}
    |\psi\rangle_{A} \otimes|\psi\rangle_{B} & = \left(a_{0}|0\rangle+a_{1}|1\rangle\right) \otimes\left(b_{0}|0\rangle+b_{1}|1\rangle\right) \nonumber\\
    & = a_0 b_0\ket{00} + a_0 b_1\ket{01} + a_1 b_0\ket{10} + a_1 b_1\ket{11}.
    \label{eq:two_qubit_separable}
\end{align}
Our goal is to find suitable probability amplitudes $a_i$ and $b_j$ such that $|\psi\rangle_{A} \otimes|\psi\rangle_{B} = |\Phi^+\rangle_{AB}$.
By comparing Eq.~(\ref{eq:entangled_state}) with Eq.~(\ref{eq:two_qubit_separable}), we see that the probablity amplitudes must satisfy two conditions,
\begin{equation}
    a_0 b_0 = a_1 b_1 = \frac{1}{\sqrt{2}}, \quad \text{ and }\quad a_0 b_1 = a_1 b_0 = 0.
\end{equation}
Let's look at the second condition first.
Either $a_0$ or $b_1$ is zero.
But setting either of them to 0, then we see that $a_0 b_0$ or $a_1 b_1$ will also be 0, meaning the first condition cannot be satisfied.
This is a remarkable discovery.
It says that \emph{not all global states can be written as a tensor product of local states}.

This leads us to the realization that there are two major classes of states.
There are \textbf{\emph{product states}}\index{product state}.
A product state is one that \textbf{\emph{can}} be written as a tensor product of local states.
Given two local states $\ket{\psi}_A$ and $\ket{\psi}_B$, we can easily find the global state by forming the tensor product.
Since we can write down the state vectors for both the global state and the local states, we say that we perfect knowledge of the local states.

The other class of quantum states are \textbf{\emph{entangled states}}\index{entangled state}.
An entangled state is a state whose global state \textbf{\emph{cannot}} be written as the tensor product of local states.
In the previous example, we have demonstrated this for an equal superposition of $\ket{00}$ and $\ket{11}$, this is in fact the case.
This is very interesting because it implies that we have perfect knowledge of the global state, but we have imperfect knowledge of local states.



%%%%%%%%%%%%%%%%%%%%%%%%%%%%%%%%%%%%%%%%%%%
\section{Bell states}
\label{sec:4_3-bell_states}
%%%%%%%%%%%%%%%%%%%%%%%%%%%%%%%%%%%%%%%%%%%

After defining what entangled states are, it is time to take a look at a very useful class of entangled states, called the \textbf{\emph{Bell states}}\index{Bell states}.
There are a total of four Bell states, 
\begin{equation}
    \begin{aligned}
        |\Phi^+\rangle & = \frac{1}{\sqrt{2}} \left( |00\rangle + |11\rangle \right), & 
        |\Phi^-\rangle = \frac{1}{\sqrt2} \left( |00\rangle - |11\rangle \right), \\
        |\Psi^+\rangle & = \frac{1}{\sqrt{2}} \left( |01\rangle + |10\rangle \right), & 
        |\Psi^-\rangle = \frac{1}{\sqrt2} \left(|01\rangle - |10\rangle \right).
    \end{aligned}
\end{equation}
We have already encountered the equal superposition of $|\Phi^+\rangle$.
Its counterpart is the $\ket{\Phi^-}$ (``phi minus'') state, where the $\ket{00}$ and $\ket{11}$ have opposite phase, as shown represented by the minus sign.
There are two more states, $\ket{\Psi^+}$ (``psi plus'') and $\ket{\Psi^-}$ (``psi minus''), which are superpositions of \ket{01} and \ket{10}, with \ket{\Psi^-} having opposite phase between the two terms. 
We will encounter these states countless number of times in the context of quantum communication.~\footnote{In some texts and papers, more commonly but not exclusively older ones, you will see these states written using the lower-case Greek letters $\phi$ (phi) and $\psi$ (psi).  In this book, we will use the capital Greek letters when referring to the Bell pairs and the lower-case letters when referring to specific variables.} 

The Bell states form an orthogonal basis for the space of two qubits.
This means that any state vector of two qubits can be written in terms of the four Bell states.
We begin with rewriting the computational basis states of two qubits $\{\ket{00},\ket{01},\ket{10},\ket{11}\}$ in terms of the Bell states.
It is not too difficult to verify that the following equalities are indeed true,
\begin{equation}
    \begin{aligned}
    |00\rangle &=\frac{1}{\sqrt{2}}\left(\left|\Phi^{+}\right\rangle+\left|\Phi^{-}\right\rangle\right), & 
    \ket{01} = \frac{1}{\sqrt2}(\ket{\Psi^{+}} + \ket{\Psi^{-}}), \\
    |10\rangle &=\frac{1}{\sqrt{2}}\left(\left|\Psi^{+}\right\rangle-\left|\Psi^{-}\right\rangle\right), & 
    \ket{11} = \frac{1}{\sqrt2}(\ket{\Phi^{+}} - \ket{\Phi^{-}}).
    \label{eq:comp_to_bell_basis}
    \end{aligned}
\end{equation}
An arbitrary two-qubit pure state can be written in the computational basis as follows,
\begin{equation}
    |\psi\rangle=\alpha|00\rangle+\beta|01\rangle+\gamma|10\rangle+\delta|11\rangle,
\end{equation}
where the probability amplitude satisfy the usual normalization condition, $|\alpha|^2+|\beta|^2+|\gamma|^2+|\delta|^2=1$.
Using Eq.~(\ref{eq:comp_to_bell_basis}), this pure state can be readily written in the Bell basis,
\begin{equation}
    |\psi\rangle=\frac{\alpha+\delta}{\sqrt{2}}\left|\Phi^{+}\right\rangle+\frac{\alpha-\delta}{\sqrt{2}}\left|\Phi^{-}\right\rangle+\frac{\beta+\gamma}{\sqrt{2}}\left|\Psi^{+}\right\rangle+\frac{\beta-\gamma}{\sqrt{2}}\left|\Psi^{-}\right\rangle.
    \label{eq:psi_bell_basis}
\end{equation}
Note that since we have changed the basis the probability amplitudes have transformed accordingly.

We have so far been dealing with single-qubit measurements.
Such measurements only had two possible outcomes.
Measuring in the Pauli $Z$ basis would project the initial state onto either $\ket{0}$ or $\ket{1}$.
If the basis of measurement is the Pauli $X$, then the post-measurement state will be either $\ket{+}$ or $\ket{-}$.
However, there is no reason to stop at single-qubit measurements.
We can extend our previous discussion to two-qubit measurements.
Not only that, we can consider two-qubit measurement in an entangled basis.
An example of this is \textbf{\emph{measurement in the Bell basis}}\index{Bell-basis measurement}.
Measuring a general two-qubit state $\ket{\psi}$ in the Bell basis has now four possible outcomes.
The post-measurement state can be any of the four Bell states $\{\ket{\Phi^+}, \ket{\Phi^-}, \ket{\Psi^+}, \ket{\Psi^-}\}$.
Using Eq.~(\ref{eq:psi_bell_basis}), we can easily determine the probabilities of these outcomes,
\begin{equation}
    \begin{aligned}
    &\operatorname{Prob}\left\{|\Phi^{+}\rangle\right\}=\frac{|\alpha+\delta|^{2}}{2} \quad \operatorname{Prob}\left\{|\Phi^{-}\rangle\right\}=\frac{|\alpha-\delta|^{2}}{2} \\
    &\operatorname{Prob}\left\{|\Psi^{+}\rangle\right\}=\frac{|\beta+\gamma|^{2}}{2} \quad \operatorname{Prob}\left\{|\Psi^{-}\rangle\right\}=\frac{|\beta-\gamma|^{2}}{2}
    \end{aligned}
\end{equation}
This will be again very important because measurements in the Bell basis are crucial in many protocols in quantum communication, especially teleportation and entanglement swapping, which we will look at quite closely later in this book.

We mentioned in Sec.~\ref{sec:4_2-entagnled_states} that entangled states have the curious property that the global state is known perfectly, while it is not possible to express the state of the local qubits using pure states.
This suggests that we only have imperfect knowledge of the local states.
We will make this notion more concrete.

Consider the two-qubit state to be the $|\Phi^+\rangle$ Bell state.
Measuring this state in the Bell basis will yield the $|\Phi^+\rangle$ outcome with unit probability.
This is consistent with our interpretation of having full knowledge of the global state.
Let's see what happens if we perform a single-qubit measurement on the state $|\Phi^+\rangle$.
For example, we measure qubit $A$ in the Pauli $Z$ basis.
Since this is a single-qubit measurement, we can only two possible outcomes.
The projectors associated with these two outcomes are
\begin{equation}
    \Pi^A_{Z,+} = |0\rangle_A\langle0| \otimes I_B, \quad \text{ and } \Pi^A_{Z,-} = |1\rangle_A\langle1| \otimes I_B.
    \label{eq:local_projectors}
\end{equation}
Note the slight but necessary change of notation in Eq.~(\ref{eq:local_projectors}).
The superscript $A$ labels which local qubit is being measured, while the subscript now labels both the basis and the measurement outcome.
The probability of obtaining the +1 outcome is given by
\begin{align}
    \operatorname{Pr}\{+1\} & = \operatorname{Tr}\left\{\Pi^A_{Z,+} |\Phi^+\rangle_{AB}\langle\Phi^+|\right\} \\
    & = \langle\Phi^+|_{AB} \left( |0\rangle_A\langle0| \otimes I_B \right) |\Phi^+\rangle_{AB} \nonumber\\
    & = \frac{1}{2} \left( \langle00|_{AB} + \langle11|_{AB} \right) \left( |0\rangle_A\langle0| \otimes I_B \right) \left( |00\rangle_{AB} + |11\rangle_{AB} \right) \nonumber\\
    & = \frac{1}{2} \left( \langle0|0\rangle_A\langle0|0\rangle_A \right)
\end{align}

\label{page:plus-is-pure}
At this point, you may be a little puzzled; didn't we see this same fifty-fifty behavior with a single qubit in the $\ket{+}$ state?  What is different with entanglement?  When measuring in the Z basis, it is true that we always see the fifty-fifty statistics.  However, if we measure a $\ket{+}$ in the X basis, we will \emph{always} find that the state is in the $+1$ state.  We can also choose to apply a Hadamard gate to the $\ket{+}$ state and return it to a known $\ket{0}$ state.  Thus, although the qubit is in superposition, we have complete information about that superposition.  In contrast, when we hold only one qubit of an entangled pair and attempt to measure it, our results will differ.

%\rdv{This par fails to completely close the deal in demonstrating that we don't have any local information, since the outcomes are 50/50 both when we are talking about a single plus state and a Bell state.  Needs a counterpoint where we have full information?  Show that the single plus state always has a specific outcome when measuring in the X basis.} Added above.

If we change the measurement basis to a Pauli Y basis, again we find that the probability of the $+1$ outcome is the same as the probability of the $-1$ outcome. It is the same for the X basis, and in fact in any basis that you measure qubit A, you will get the same result: both outcomes are fifty-fifty. In fact, the local measurement results are uniformly random in any basis. Remember the state of the global system is pure. We know exactly what it is, yet no matter in what basis we measure the states locally we are getting 50 percent one outcome and 50 percent the other outcome. We can say that we have no knowledge of the local states of qubit A and qubit B. This goes back to our previous discussion about the difference between entangled states and product states.  Since this state is entangled, the correct description of the local qubits must be given in terms of density matrices, and we can write that the state of qubit A is a maximally mixed state. With probability half it's in the zero state, and with probability half it's in the one state. The same holds for qubit B. Again we write it as a maximally mixed state,
\begin{equation}
\begin{aligned}
&\rho_A=\frac{1}{2}(|0\rangle\langle 0|+| 1\rangle\langle 1|) \\
&\rho_B=\frac{1}{2}(|0\rangle\langle 0|+| 1\rangle\langle 1|).
\end{aligned}
\end{equation}

Just to stress how strange this is that we have a full knowledge of the global state yet we have zero knowledge of the local states, let's look at this again: qubit A is a maximally mixed state. We have no knowledge about its state. The same is true for qubit B considered in isolation. Yet, somehow, when we look at the state globally, we have perfect knowledge of the entire state. This distinction is addressed more in the chapter exercises.


%%%%%%%%%%%%%%%%%%%%%%%%%%%%%%%%%%%%
\section{SPDC}
\label{sec:4-4_spdc}
%%%%%%%%%%%%%%%%%%%%%%%%%%%%%%%%%%%%

% pic
\begin{figure}[H]
    \centering
    \includegraphics[width=0.5\textwidth]{lesson4/pump.pdf}
    
        \caption[Symmetric parametric down conversion (SPDC)]{Symmetric parametric down conversion (SPDC) turns one high-energy pump photon into two lower-energy entangled photons, with some very low probability. The two output photons are called \emph{signal} and \emph{idler}.}
    \label{fig:spdc}
    
\end{figure}

\begin{figure}[H]
    \centering
    \includegraphics[width=0.8\textwidth]{lesson4/4-8_spdc.pdf}
    
        \caption[pump photon and the energy levels of the atom]{A representation of the pump photon of energy $E_p$ and the energy levels of the atom. Energy level diagrams such as these are an abstraction that allows us to describe absorption and emission of photons by an atom. The circle indicates that there is a single electron in the ground state.
        The atom absorbs the energy provided by the pump photon and transitions to the excited state. The atom decays spontaneously after some time, releasing energy in the form of two photons with energies $E_s$ and $E_i$. Due to conservation of energy, we have $E_p = E_s + E_i$.}
    
    \label{fig:spdc-energy-levels}
\end{figure}

% \begin{figure}[H]
%   \centering
%   \begin{minipage}[b]{0.3\textwidth}
%     \includegraphics[width=\textwidth]{lesson4/state1.pdf}
%     \caption{A representation of the pump photon and the energy levels of the atom.}
%   \end{minipage}
%   \hfill
%   \begin{minipage}[b]{0.3\textwidth}
%     \includegraphics[width=\textwidth]{lesson4/state2.pdf}
%     \caption{After absorption of the photon, the electron moves to the excited state.}
%   \end{minipage}
%   \hfill
%   \begin{minipage}[b]{0.3\textwidth}
%     \includegraphics[width=\textwidth]{lesson4/state3.pdf}
%     \caption{A representation of the emission of two photons when the electron decays from the excited state back to the ground state.}
%   \end{minipage}
% \end{figure}

\begin{figure}[H]
    \centering
    \includegraphics[width=0.8\textwidth]{lesson4/linear_polarization.pdf}
    
        \caption[Horizontally, vertically and diagonally polarized light]{If horizontally and vertically polarized light are used to encode $\ket{0}$ and $\ket{1}$, then diagonally polarized light encodes $\ket{+}$.}
    
    \label{fig:hvd-light}
\end{figure}

\begin{figure}[H]
    \centering
    \includegraphics[width=0.5\textwidth]{lesson4/horizontal_optical_axis.pdf}
    
        \caption[Horizontal to vertical SPDC]{Horizontally polarized light matching the optical axis of the crystal can be down-converted to vertically polarized pairs of photons.}
    
    \label{fig:horizontal-opt-axis}
\end{figure}

\begin{figure}[H]
   \centering
    \includegraphics[width=0.5\textwidth]{lesson4/vertical_optical_axis.pdf}
    
        \caption[Vertical to horizontal SPDC]{Vertically polarized light matching the optical axis of the crystal can be down-converted to horizontally polarized pairs of photons.}
    
    \label{fig:vertical-opt-axis}
\end{figure}

% Step Four: SPDC.

In this section, we will learn how to actually generate entangled states physically. Specifically, we will look at the process of \emph{spontaneous parametric down-conversion}\index{spontaneous parametric down-conversion (SPDC)}, or SPDC for short. There are many physical processes that can generate entangled pairs of photons, but this one is  common in laboratories and will serve as our example. 

"Spontaneous” means that the process is not stimulated.
In a spontaneous process, no other input is needed, only the incoming photon of high energy that gets converted into two photons of lower energies.
(In contrast, a \emph{stimulated} process requires both the input photon and some other photon as well; a good example would be lasing.) “Parametric” means the whole process does not depend only on the intensity of the electric field but also on the phase of the field. “Parametric” in the context of SPDC means there is a relationship between the input and output electric fields.

We will illustrate SPDC with the little graphic in Fig.~\ref{fig:spdc}. Imagine that you have some laser light, represented by the green arrow, which we call the \emph{pump}, incident normally on a non-linear crystal represented by the rectangle. In this particular case, we are using a beta barium borate, or BBO, crystal, which is a popular choice in experiments. This pump laser gets transformed and split into two beams. One beam is called the "signal" and the other one is called the "idler". If we actually zoom into the crystal to see what happens to individual atoms, we get the following picture: the pump light is tuned to be resonant with some transition frequency of the atoms inside the BBO crystal, so to a good approximation it affects only two energy levels of the material. The lower level we will call the "ground state" and the upper level we will call the "excited state". The atom in the BBO crystal starts in the ground state, but then it gets blasted by this pump laser.  The atom becomes excited and absorbs the energy from the incoming pump laser, and after a short time the atom de-excites.

The vast majority of the time, this de-excitation is coupled with the emission of a single, high-energy photon. With some very low probability, however, the atom ejects two photons with energy $E_s$ for the signal photon and energy $E_i$ for the idler photon. Because energy is conserved in this transition, we must have that the energy of the pump photon must be equal to the sum of the energies of the outgoing photons, $E_p = E_s + E_i$. This is the basic process of spontaneous parametric down-conversion.

Now let's see how polarization of light interacts with this process. Consider linearly polarized light. If we again take our green pump laser, we can make it oscillate only in one plane. For example, if we make it oscillate in the horizontal plane, this can encode our $\ket{0}$ state of a qubit, as in Fig.~\ref{fig:hvd-light}. We can also make the light oscillate only in the vertical plane, in which case we will say that this encodes a $\ket{1}$. We already know from previous sections and chapters that we can talk about superpositions of these states. When we go to the polarization picture, we can get diagonally polarized light which we call \ket{D}, and this will encode our equal superposition of $\ket{0}$ and $\ket{1}$, or as we call it a $\ket{+}$ state. That's the next ingredient that we're going to need in producing entangled pairs of photons.

How does polarized light actually interact with this nonlinear BBO crystal? That depends on the optical axis of the crystal. For now, we don't have to worry too much about what an optical axis is and how it's defined. Just remember that there is some kind of special direction in which the BBO crystal can be aligned and that's very important relative to the incoming light and its polarization. If the optical axis aligns horizontally, and the incoming light is horizontally polarized, then the photons that are coming out from the crystal (the signal and the idler) will both be vertically polarized. So this setup, as shown in Fig.~\ref{fig:horizontal-opt-axis} -- horizontally polarized pump light and optical axis in the direction represented by the purple arrow -- implements the physical transformation
\begin{equation}
\begin{aligned}
\ket{H} &\rightarrow \ket{VV}.
\end{aligned}
\end{equation}
It takes a horizontally polarized state and outputs two vertically polarized states.

Naturally, if we rotate the optical axis of the BBO crystal ninety degrees and use vertically polarized light, as in Fig.~\ref{fig:vertical-opt-axis}, then the two photons output will both be horizontally polarized.  In this case we are implementing the physical transformation
\begin{equation}
\begin{aligned}
\ket{V} &\rightarrow \ket{HH}.
\end{aligned}
\end{equation}

On the other hand, if the optical axis of the crystal and the polarization are orthogonal, the light passes through unchanged.

\begin{figure}[t]
    \centering
    \includegraphics[width=0.6\textwidth]{lesson4/4-4_SPDC_crystal.png}
    \caption[BBO crystal arrangement]{Arrangement of the two BBO crystals that leads to an entangled photon pair when diagonally polarized light is input.}
    \label{fig:4-4_SPDC_crystals}
\end{figure}

We have one last step to produce entangled photon pairs.  We take two of these BBO crystals and put them next to each other, and arrange them such that their optical axes are orthogonal to each other, as seen in Fig.~\ref{fig:4-4_SPDC_crystals}.

In such a geometry, when we set the pump laser to be diagonally polarized, since $\ket{D} = (\ket{H}+\ket{V})/\sqrt{2}$, the \ket{H} component will be transformed by the first crystal and the \ket{V} component will be transformed by the second crystal, giving us the  transformation
\begin{equation}
    \ket{D} \rightarrow \frac{\ket{HH}+\ket{VV}}{\sqrt{2}}.
\end{equation}
Diagonally polarized light becomes a superposition of the horizontally polarized photons and vertically polarized photons. You can see that we are changing a $\ket{+}$ state into a Bell pair $(\ket{00}+\ket{11})/\sqrt{2}$, and this is how we get entangled states using SPDC.  In order for this to work, the BBO crystals have to be very, very thin. This will ensure that we are not quite sure whether the photons were produced in the first crystal or in the second crystal. We call this the \emph{indistinguishability criterion}, and if this is satisfied then we can talk about genuinely entangled photons. If, however, we can actually tell where the two photons originate, then we will only get horizontally polarized photons or we will get vertically polarized photons, and there will be no entanglement shared between these two photons. 

Another important property of SPDC is that it's an extremely rare process.  We obtain only about one photon pair per $10^6$ pump photons, and even that level is achieved only in state-of-the-art experiments. In a more common laboratory setup, it takes several orders of magnitude more pump photons to successfully create each entangled pair.  The photons that are not down-converted simply pass through the crystal unaffected and are discarded.

There are many details around SPDC that we have not introduced here, but this basic understanding is adequate for our purposes in this book.

% Usually this number- this ten-to-the-six pump photons is much higher or orders of magnitudes higher in commercially available lasers.


%%%%%%%%%%%%%%%%%%%%%%%%%%%%%%%%%%%%%%%%%%%%%%%%
\section{Entanglement as a resource}

% insert broken connection 
\begin{figure}[H]
   \centering
    \includegraphics[width=0.8\textwidth]{lesson4/CHSH_broken_entanglement.pdf}
    
        \caption[CHSH game collapses entanglement.]{In the CHSH game, each time the players play, they collapse (break) one entangled state.}
    
    \label{fig:chsh-broken}
\end{figure}

% Step Five: Entanglement as a Resource.

Let's go back and revisit our CHSH game from the first section of this chapter. We have seen that entanglement can actually help players A and B win the game more often. If they share an entangled state, they can perform measurements on it and this will help them in winning the game with higher probability. But in this process of measurement, as shown in Fig.~\ref{fig:chsh-broken}, A and B destroy the entanglement shared between the two grey qubits.  Maybe they managed to win the game, at least this particular round, but if they want to win again, they have to find a way of re-establishing this entanglement or creating a new entangled pair which they can share and then use in the new round of the game. 

Entanglement is also crucial in quantum networks.
Figure~\ref{fig:4-5_resource_network} shows a simple quantum network, where the blue circles represent the network nodes.
They could be individual qubits or even small quantum computers; for now, the distinction is not important.
The lines joining the network nodes represent shared entanglement.
There is a sender node in possession of a qubit, represented by the red circle, that it is trying to communicate to the receiver node.
The usual way to send quantum information is to use the teleportation protocol, which we will discuss in detail in Chapter~\ref{sec:8_teleportation}.
For now, all we need to know is that the protocol consumes a single Bell pair to transfer the quantum information from one node to its neighbor.
By executing the teleportation protocol first at the sender's node, and then at nodes $N_1$ and $N_2$, the quantum information will be delivered to the receiver node.
We can see that delivering the quantum information has consumed all of the entangled pairs along the path connecting the sender with the receiver.
In order for the quantum network to be fully functional again, we must re-establish the destroyed entangled links.

\begin{figure}[t]
    \centering
    \includegraphics[width=\textwidth]{lesson4/4-5_resource_network.pdf}
    \caption[Consumption of entanglement in a quantum network.]{Entanglement is consumed as quantum information is propagated from the sender to the receiver.}
    \label{fig:4-5_resource_network}
\end{figure}

We can think of entanglement as the fuel that drives many quantum technologies. Entanglement offers improved, and sometimes completely new, functionality that is not seen in either classical networks or classical computation.
One of the key takeaways is that entanglement is a \textbf{\textit{resource}}, and is consumed just like fuel in your car or battery charge in your phone.
It is the job of quantum networks to distribute entanglement in order to satisfy the demand for this precious resource.


\newpage
\begin{exercises}
\exer{Prove that the four Bell states are orthogonal to each other.}

\exer{\rdv{Find the probabilities of the different measurement outcomes for several one- and two-qubit states.}}

\exer{\rdv{More of the math of the CHSH game to be added as exercise.}
}

\exer{On p.~\pageref{page:plus-is-pure}, we stated without proof that measuring the state of a single qubit in $\ket{+}$ provides different outcome statistics than measuring one qubit that is part of a Bell pair.  Prove this by calculating the outcome probabilities for measuring both the single qubit and the half of the Bell pair, when using the X, Y, and Z measurement bases.  (Hint: this problem will be easier if you have done the problems in Ch. 3.)}

\exer{\rdv{Do some rough calculations of number of entangled photon pairs output from SPDC for a certain wattage of laser, wavelength, and conversion efficiency.}}

\end{exercises}


\newpage
\section*{Quiz}
  \addcontentsline{toc}{section}{Quiz}

% \section{Learning more}

The online version of this course includes a quiz for this block of chapters. Discussion of the quiz questions will be provided there.

\section*{Further reading for chapters 1-4}
  \addcontentsline{toc}{section}{Further reading for chapters 1-4}

{\bf Chapter 1}\\

The first chapter serves as a gentle introduction to the area of quantum communication. Its goal is to provide a qualitative overview and discuss ideas behind communication and its evolution rather than focusing on any quantitative discussion.

An excellent popular science book on this topic is James Gleick, \emph{The Information: A History, The Theory, A Flood}~\cite{gleick2012information}. It covers the evolution of communication and spends a sizeable portion of its length on discussing the switch from analog to digital. However, it does not cover quantum communication.

There are also a number of great science YouTube channels with excellent, concise introductions on the topic of quantum communication and computation:
\begin{enumerate}
    \item Veritasium
    \item Science Girl
    \item PBS Space Time
\end{enumerate}

Understanding the role that encryption plays in achieving the overall goals of computer security is valuable. We recommend Matt Bishop's authoritative textbook, \emph{Computer Security: Art and Science}~\cite{bishop2002art}.

{\bf Chapter 2}\\

The second chapter is a lot more technical and introduces fundamental concepts such as qubits and measurements, which we will encounter throughout the entire curriculum. Being comfortable with these concepts and knowing how to describe them mathematically is crucial.

An article by Perry \emph{et al.} called "Quantum Computing as a High School Module" is a great extension to the content of this and following chapters. It introduces basic mathematical descriptions and contains many short exercises designed to check your understanding.

A more advanced introduction to the mathematical concepts in this chapter can be found in the classic textbook by Michael A. Nielsen and Isaac L. Chuang, \emph{Quantum Computation and Quantum Information}~\cite{nielsen-chuang:qci}. Almost all serious quantum researchers have a copy of this book, but its learning curve is steep and we recommend attacking it after a more introductory book or course (which you presumably are gaining through this learning module and other courses). This book, known as "Mike and Ike", covers key ideas in computer science theory, the basic quantum information ideas and algorithms, and the principles of quantum hardware. Its descriptions of hardware and error correction are now rather out of date, and the description of algorithms is limited to a few important cases, but the principles are foundational and the explanation clear.

An alternative to Mike and Ike is John Preskill's lecture notes, which are available chapter by chapter as open access PDF files~\cite{preskill:PH-CS219}.

A new textbook that covers the basics of quantum information and their role in quantum computation and communication is Robert Sutor's \emph{Dancing with Qubits}~\cite{sutor19:dancing}. This textbook spends great effort in explaining the basics and goes over the fundamental calculations in great detail. The first half of the book covers the basic mathematics you will need, including complex numbers, probability and linear algebra (vectors and matrices). We strongly recommend this book for all beginning students, especially those who are worried about the math required. We recommend it to our own beginning undergraduates.

An alternative to Sutor that focuses primarily on basic algorithms is Eleanor Rieffel and Wolfgang Polak's \emph{Quantum Computing: a Gentle Introduction}~\cite{rieffel2011quantum}.

{\bf Chapter 3}\\

The third chapter continues with our exposition of the basic mathematical formalism underpinning quantum communication as well as quantum computation. The primary focus is description of noisy quantum states using density matrices.

Chapter 2 of “Mike \& Ike” and Perry’s article mentioned above are great for this chapter as well. Another fantastic book on quantum information (though primarily targeted at graduate students so parts of it might be too technical to follow) is Mark Wilde's \emph{Quantum Information Theory}~\cite{wilde2013quantum}.

Chapter 3 of Wilde’s textbook deals with noisy states. Discussion on fidelity of quantum states can be found in Chapter 9 of “Mike \& Ike” and in Chapter 9 of Wilde’s textbook.

{\bf Chapter 4}\\

This chapter is the conclusion of the introductory Block and deals with the scenario when we have multiple qubits which naturally leads to entanglement. Perry’s article, “Mike \& Ike” as well as Wilde’s textbook remain great introductions and ordered in this way represent the ramping up difficulty of mathematical rigour. 

The CHSH game is discussed in Chapter 3 of Wilde’s textbook~\cite{wilde2013quantum}. Another excellent exposition of this game is by Umesh Vazirani as part of his lecture series on YouTube.

A great introduction to SPDC can be found in Betchart’s bachelor's thesis~\footnote{\url{https://etd.ohiolink.edu/apexprod/rws_olink/r/1501/10?clear=10&p10_accession_num=oberlin1206296667}}.

