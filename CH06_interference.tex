\chapter{Interference}
\label{sec:6_interference}

In this chapter, we will discuss the fundamental phenomenon of interference.
Most of you have very likely encountered interference of waves in either water, light or even sound.
We begin this chapter with a mathematical description of what happens when two waves of different frequencies interfere together.
After that, we will move to interference at the quantum level, using single photons.
Finally, we will conclude this chapter by learning that interference can occur at the level of individual qubits as well.


%%%%%%%%%%%%%%%%%%%%%%%%%%%%%%%%%%%%%%%%%%%%%%%%%%%
\section{Constructive and destructive interference}
\label{sec:6-1_constructive_and_destructive_interference}
%%%%%%%%%%%%%%%%%%%%%%%%%%%%%%%%%%%%%%%%%%%%%%%%%%%


Good starting point is to settle on the notation that will be used in this chapter.
Let's consider a wave oscillating at a single frequency, and which is propagating in time along a single dimension denoted by coordinate $x$,

\begin{equation}
    \Psi(x,t) = A \sin [\omega t-(k x+\phi_0)].
    \label{eq:simple_wave}
\end{equation}

We denote the wave with $\Psi(x,t)$.
The \textbf{\emph{amplitude}}\index{amplitude}, denoted by $A$, is the maximum value that $\Psi(x,t)$ takes.
It signifies how how much the wave is displaced from its rest state.
The symbol $\omega$ (small Greek letter ``omega'') is the \textbf{\emph{angular frequency}}\index{angular frequency}, and it determines how quickly the wave propagates in time.
Time is denoted by $t$.
The symbol $k$ is known as the \textbf{\emph{wave number}}\index{wave number}.
It tells us how quickly the wave oscillates in space, and therefore is related to the wavelength of the oscillations.
The wave number also tells us about the direction in which the wave propagates.
Finally, $\phi_0$ (small Greek letter ``phi'') is called the initial phase of oscillations.
The angular frequency $\omega$, and the wave number $k$, are related to the period of oscillations $T$ and the wavelength $\lambda$, respectively,
\begin{equation}
    \omega=\frac{2 \pi}{T}, \qquad k=\frac{2 \pi}{\lambda},
    \label{eq:freq-period_k-wavelength}
\end{equation}

Now is a good time to have a look at some simple examples of the simple wave described by Eq.~(\ref{eq:simple_wave}).
Let's freeze the wave at time $t=0$, and for convenience we also set $\phi_0 = 0$, and only vary the wave number $k$.
The blue wave in Fig.~\ref{fig:two-waves} represents the case when $k=1$, whereas the orange one represents the case when $k=0.5$.
The wavelength is the shortest distance between two points of the where it begins to repeat itself.
We can see that halving the wave number $k$ doubles the wavelength.
In other words, increasing the wave number ``compresses the oscillations'', while decreasing the wave number ``stretches'' them.

\begin{figure}[t]
    \centering
    \resizebox {0.6\textwidth} {!} {
    \begin{tikzpicture}
        \begin{axis}[xtick=\empty,
                    ytick={-1,1},
                    yticklabels={$-A$,$A$},
                    axis lines=middle,
                    enlarge x limits={abs=0.2},
                    enlarge y limits={abs=0.3},
                    x label style={anchor=north},
                    y label style={anchor=east},
                    xlabel={$x$},
                    ylabel={$\Psi(x,0)$},
                    width=9cm,
                    height=6cm]
        \addplot[domain=-1:18,samples=100,smooth,myblue,thick] {sin(deg(-x))};
        \addlegendentry{$k=1$}
        \addplot[domain=-1:18,samples=100,smooth,myorange,thick] {sin(deg(-0.5*x))};
        \addlegendentry{$k=0.5$}
        \node[circle,fill,inner sep=1.5pt,myblue] at (axis cs:3*pi/2,1) {};
        \node[circle,fill,inner sep=1.5pt,myblue] at (axis cs:7*pi/2,1) {};
        \draw[dashed, blue!80, thick] (axis cs:3*pi/2,1) -- node[above] {$\lambda=2\pi/1.0$} (axis cs:7*pi/2,1);
        \node[circle,fill,inner sep=1.5pt,myorange] at (axis cs:pi,-1) {};
        \node[circle,fill,inner sep=1.5pt,myorange] at (axis cs:5*pi,-1) {};
        \draw[dashed, orange!80, thick] (axis cs:pi,-1) -- node[below] {$\lambda=2\pi/0.5$} (axis cs:5*pi,-1);
        \end{axis}
    \end{tikzpicture}
    }
    \caption[Same frequency, different wave numbers.]{Two independent waves with $k=1$ (blue), and $k=0.5$ (orange). The time is frozen at $t=0$, and we set the initial phase to be $\phi_0=0$ for convenience.}    
    \label{fig:two-waves}
\end{figure}

Let's consider what happens when we fix the wave number but we vary the initial phase, as shown in Fig.~\ref{fig:phase-diff-waves}.
In this case, we have shifted the second wave by an initial phase of $\phi_0=\pi/2$.
Varying the initial phase has the effect of shifting the wave along the $x$ coordinate.
The wavelength of the wave remains unaffected.

\begin{figure}[t]
    \centering
    \resizebox {0.6\textwidth} {!} {
    \begin{tikzpicture}
        \begin{axis}[xtick=\empty,
                    ytick={-1,1},
                    yticklabels={$-A$,$A$},
                    axis lines=middle,
                    enlarge x limits={abs=0.2},
                    enlarge y limits={abs=0.3},
                    x label style={anchor=north},
                    y label style={anchor=east},
                    xlabel={$x$},
                    ylabel={$\Psi(x,0)$},
                    width=9cm,
                    height=6cm]
        \addplot[domain=-1:18,samples=100,smooth,myblue,thick] {sin(deg(-x))};
        \addlegendentry{$\phi_0=0$}
        \addplot[domain=-1:18,samples=100,smooth,myorange,thick] {sin(deg(-x+pi/2))};
        \addlegendentry{$\phi_0=\pi/2$}
        \node[circle,fill,inner sep=1.5pt,black!80] at (axis cs:3*pi/2,1) {};
        \node[circle,fill,inner sep=1.5pt,black!80] at (axis cs:2*pi,1) {};
        \draw[dashed, black!80, thick] (axis cs:3*pi/2,1) -- node[above] {$\pi/2$} (axis cs:2*pi,1);
        \end{axis}
    \end{tikzpicture}
    }
    \caption[Same wavelength, different initial phase.]{Two waves with the same wavelength and frequency but different initial phases, $\phi_0 = 0$ (blue), $\phi_0=\pi/2$ (orange).}
    \label{fig:phase-diff-waves}
\end{figure}

Finally, let's add time dependence into the picture, as shown in Fig.~\ref{fig:propagating-waves}, where we have set the wave number $k=1$, and the initial phase $\phi=0$.
The angular frequency of the blue wave is set to $\omega=0.1$.
Figure~\ref{fig:propagating-waves} plots the wave at three different times $t_1$, $t_2$, $t_3$, where $t_1<t_2<t_3$.
The angular frequency of the orange wave is set to $\omega=0.2$.
We observe that the wave is propagating faster as it covers longer distance in the same amount of time.
We can compare Fig.~\ref{fig:propagating-waves} with Fig.~\ref{fig:two-waves} to better appreciate the difference between the angular frequency $\omega$ and the wave number $k$.
The wave number $k$ controls the ``shape'' of the wave.
More specifically, it stretches or contracts the oscillations by determining their wavelength.
On the other hand, the angular frequency $\omega$ controls how much the whole wave shifts in time.
We will see in Sec~\ref{sec:6-2_phase_group_velocity} that both $\omega$ and $k$ are related to the velocity at which the wave propagates.

\begin{figure}[t]
   \centering
    % \includegraphics[width=0.8\textwidth]{lesson6/w.pdf}
    \resizebox {0.6\textwidth} {!} {
    \begin{tikzpicture}
        \begin{axis}[name=plotA,
                    xtick=\empty,
                    ytick={-1,1},
                    yticklabels={$-A$,$A$},
                    axis lines=middle,
                    enlarge x limits={abs=0.2},
                    enlarge y limits={abs=0.3},
                    x label style={anchor=north},
                    y label style={anchor=east},
                    xlabel={$x$},
                    ylabel={$\Psi(x,t)$},
                    width=9cm,
                    height=6cm]
        \addplot[domain=-1:18,samples=100,smooth,myblue,thick] {sin(deg(0.1*15-x))};
        \addplot[domain=-1:18,samples=100,smooth,myblue,thick,opacity=0.2] {sin(deg(0.1*0-x))};
        \addplot[domain=-1:18,samples=100,smooth,myblue,thick,opacity=0.6] {sin(deg(0.1*7.5-x))};
        \addlegendentry{$\omega=0.1$}
        \node[myblue,opacity=0.2] at (axis cs:3*pi/2,1.15) {$t_1$};
        \node[myblue,opacity=0.6] at (axis cs:3*pi/2+0.1*7.5,1.15) {$t_2$};
        \node[myblue] at (axis cs:3*pi/2+0.1*15,1.15) {$t_3$};
        \end{axis}
    
        \begin{axis}[name=plotB,
                    at=(plotA.outer south west),
                    anchor=outer north west,
                    xtick=\empty,
                    ytick={-1,1},
                    yticklabels={$-A$,$A$},
                    axis lines=middle,
                    enlarge x limits={abs=0.2},
                    enlarge y limits={abs=0.3},
                    x label style={anchor=north},
                    y label style={anchor=east},
                    xlabel={$x$},
                    ylabel={$\Psi(x,t)$},
                    width=9cm,
                    height=6cm]
        \addplot[domain=-1:18,samples=100,smooth,myorange,thick] {sin(deg(0.2*15-x))};
        \addplot[domain=-1:18,samples=100,smooth,myorange,thick,opacity=0.2] {sin(deg(0.2*0-x))};
        \addplot[domain=-1:18,samples=100,smooth,myorange,thick,opacity=0.6] {sin(deg(0.2*7.5-x))};
        \addlegendentry{$\omega=0.2$}
        \node[myorange,opacity=0.2] at (axis cs:3*pi/2,1.15) {$t_1$};
        \node[myorange,opacity=0.6] at (axis cs:3*pi/2+0.2*7.5,1.15) {$t_2$};
        \node[myorange] at (axis cs:3*pi/2+0.2*15,1.15) {$t_3$};
        \end{axis}
    \end{tikzpicture}
    }
    \caption[Propagation of waves in time.]{Propagation of waves in time. We observe that the wave with angular frequency $\omega=0.1$ (blue) propagates more slowly compared to the wave with angular frequency $\omega=0.2$ (orange).}
    \label{fig:propagating-waves}    
\end{figure}

Having gained some intuition how the various parameters in Eq.~(\ref{eq:simple_wave}) affect the behavior of the wave, it is time to discuss what happens when two waves meet.
Let's consider a special case of two waves, $\Psi_1(x,t)$ and $\Psi_2(x,t)$, each with a different amplitude and initial phase, but both having the same frequency,
\begin{align}
    \Psi_1(x,t) & = A_1 \sin \left(\omega t+\alpha_1\right), \quad\text{where } \alpha_1=-\left(k x+\phi_1\right), \\
    \Psi_2(x,t) & = A_2 \sin \left(\omega t+\alpha_2\right), \quad\text{where } \alpha_2=-\left(k x+\phi_2\right).
    \label{eq:superposition}
\end{align}
These two waves produce a new wave given by their sum,
\begin{equation}
    \Psi(x,t) = \Psi_1(x,t) + \Psi_2(x,t).
    \label{eq:wave_superposition}
\end{equation}
This is known as the \textbf{\emph{principle of superposition}}\index{principle of superposition},
and it should look familiar.
We have in fact encountered this principle in Chapter~\ref{sec:2_quantum_states}, where we talked about superposition of two quantum state vectors.

The superposition in Eq.~(\ref{eq:wave_superposition}) will have the same form as its constituent waves,
\begin{equation}
    \Psi(x,t) = A \sin (\omega t-\alpha).
    \label{eq:resulting_wave}
\end{equation}
The question that we would like to answer is to determine how the amplitudes and initial phases of $\Psi_1(x,t)$ and $\Psi_2(x,t)$ affect the amplitude and the phase of the superpostion.
Using the trigonometric identity $\sin (a + b)=\sin a \cos b +\cos a \sin b$, we can rewrite Eq.~\ref{eq:wave_superposition} as
\begin{align}
    \Psi(x,t) & = A_1 \left( \sin\omega t \cos\alpha_1 + \cos\omega t \sin \alpha_1 \right) \nonumber\\
    & + A_2 \left( \sin\omega t \cos\alpha_2 + \cos\omega t \sin\alpha_2 \right).
\end{align}
We can group the time dependent terms together to obtain the following, 
\begin{equation}
    \Psi(x,t) = ( \underbrace{A_1 \cos\alpha_1 + A_2 \cos\alpha_2}_{\equiv A\cos\alpha} ) \sin \omega t + ( \underbrace{A_1 \sin\alpha_1 + A_2 \sin\alpha_2}_{\equiv A\sin\alpha} ) \cos \omega t.
    \label{eq:resulting_wave_2}
\end{equation}
We observe that the resulting wave $\Psi(x,t)$ does indeed oscillate with angular frequency $\omega$ as we anticipated in Eq.~(\ref{eq:resulting_wave}).
The coefficients in front of the time-dependent terms are functions of the original two waves as they depend on $A_1$, $A_2$, and $\alpha_1$, $\alpha_2$.

It would be nice to obtain a cleaner expression for the amplitude of the resulting wave $A$, and its phase $\alpha$.
We can do that by using the trigonometric identity $\cos^2\theta + \sin^2\theta=1$,
\begin{equation}
    A^2 = A^2 (\cos^2\alpha + \sin^2\alpha) = A^2\cos^2\alpha + A^2 \sin^2\alpha.
\end{equation}
We can now substitute the definition for $A\cos\alpha$ and $A\sin\alpha$ from Eq.~(\ref{eq:resulting_wave_2}), which after some rearranging yields the following expression for the square of the amplitude $A$,
\begin{equation}
    A^2 = A_1^2 + A_2^2 + 2 A_1 A_2 \cos(\alpha_2-\alpha_1).
    \label{eq:superposition-amplitude}
\end{equation}
The phase $\alpha$ can be obtained in the following way,
\begin{equation}
    \tan \alpha = \frac{A\sin\alpha}{A\cos\alpha} = \frac{A_1\sin\alpha_1 + A_2\sin\alpha_2}{A_1\cos\alpha_1 + A_2\cos\alpha_2}.
    \label{eq:superposition-alpha}
\end{equation}

We started with two initial waves that only differed in their amplitudes and initial phases.
Using the principle of superposition, we determined the mathematical description of the resulting wave $\Psi(x,t)$.
From Eq.~(\ref{eq:superposition-amplitude}), we can see that the amplitude of $\Psi(x,t)$ is determined by the amplitudes and phases of $\Psi_1(x,t)$ and $\Psi_2(x,t)$.
The last term in Eq.~(\ref{eq:superposition-amplitude}) is known as the \textbf{\emph{interference term}}\index{interference term}.
It oscillates as a function of the phase difference $\delta\equiv\alpha_2 - \alpha_1$, resulting in either positive or negative contribution to the amplitude of $\Psi(x,t)$.
When the phase difference is such that $\cos\delta>1$, the interference term contributes by increasing the amplitude $A$, which is known as \textbf{\emph{constructive interference}}\index{constructive interference}.
On the other hand, when $\cos\delta<1$, the interference term contributes negatively, situation known as \textbf{\emph{destructive interference}}\index{destructive interference}.

Let's look at the example, where the amplitudes of the initial two waves are equal, $A_1=A_2$, but their wave numbers are different.
This difference results in a phase difference, allowing us to observe both constructive and destructive interference.
The two waves have the following form,
\begin{equation}
    \Psi_1(x,t) = \sin (\omega t - k_1x), \qquad \Psi_2(x,t) = \sin (\omega t - k_2 x).
\end{equation}
The interference of these two waves is plotted in Fig.~\ref{fig:interference_example}, where we set $k_1=1.0$ (blue) and $k_2=0.8$ (orange).
The resulting wave $\Psi(x,t)$ is shown in green.
We observe that when the phase difference $\delta$ is small, which occurs in the region where $x$ is small, the two waves interfere constructively.
The amplitude of the resulting wave almost reaches all the way to $2A$.
In the region where $x$ is large however, the phase difference also increases.
We also say that the waves are ``out of phase''.
In this region, the interference is destructive.
\begin{figure}[t]
    \centering
    \resizebox {0.6\textwidth} {!} {
    \begin{tikzpicture}
        \begin{axis}[xtick=\empty,
                    ytick={-2,-1,1,2},
                    yticklabels={$-2A$,$-A$,$A$,$2A$},
                    axis lines=middle,
                    enlarge x limits={abs=0.2},
                    enlarge y limits={abs=0.8},
                    x label style={anchor=north},
                    y label style={anchor=east},
                    xlabel={$x$},
                    ylabel={$\Psi(x,t)$},
                    width=9cm,
                    height=6cm]
        \addplot[domain=-1:18,samples=100,smooth,myblue,thick,opacity=0.5] {sin(deg(-x))};
        \addlegendentry{$k_1=1.0$}
        \addplot[domain=-1:18,samples=100,smooth,myorange,thick,opacity=0.5] {sin(deg(-0.8*x))};
        \addlegendentry{$k_2=0.8$}
        \addplot[domain=-1:18,samples=100,smooth,mygreen,very thick] {sin(deg(-x))+sin(deg(-0.8*x))};
        \end{axis}
    \end{tikzpicture}
    }
    \caption[Superposition of two waves.]{Interference of two waves with different wave numbers.}
    \label{fig:interference_example}    
\end{figure}


%%%%%%%%%%%%%%%%%%%%%%%%%%%%%%%%%%%%
\section{Phase and group velocities}
\label{sec:6-2_phase_group_velocity}
%%%%%%%%%%%%%%%%%%%%%%%%%%%%%%%%%%%%

In this section, we will investigate the speed at which a wave propagates.
We will begin with case of waves with single frequency before moving onto more complicated scenario resulting from interference of multiple waves.

\emph{Phase velocity.}
First, let's consider a single wave of a single frequency $\omega$ propagating through space.
We saw in the previous section that we can write down such a wave as
\begin{equation}
    \Psi(x,t) = A \sin(\omega t - kx - \phi_0),
\end{equation}
where the phase at a particular point in time $t$ and space $x$ is $\theta(x,t)=\omega t-k x-\phi_0$.
The speed of propagation can be determined by inspecting any point on the wave.
This point is characterized by constant phase $\theta(x,t)$.
How fast does this point of constant phase propagate?
We need to differentiate the expression for phase with respect to time and set it equal to zero (since the phase does not change in time),
\begin{equation}
    \frac{d\theta(x,t)}{d t} = \omega - k\frac{dx}{dt} = 0.
\end{equation}
The rate of change of the position, $dx/dt$, is exactly the speed of propagation of the point of constant phase.
This quantity is called the \textbf{\emph{phase velocity}}\index{phase velocity},
\begin{equation}
    v_p \equiv \frac{dx}{dt}, \longrightarrow v_p = \frac{\omega}{k}.
    \label{eq:6-2_phase_velocity}
\end{equation}
We see that in the simple case of a single-frequency wave, the phase velocity is given as the ratio of the angular velocity $\omega$ and the wave number $k$.

\begin{figure}[t]
    \centering
    \resizebox {0.6\textwidth} {!} {
    \begin{tikzpicture}
        \begin{axis}[xtick=\empty,
                    ytick={-2,-1,1,2},
                    yticklabels={$-2A$,$-A$,$A$,$2A$},
                    axis lines=middle,
                    enlarge x limits={abs=0.2},
                    enlarge y limits={abs=0.3},
                    x label style={anchor=north},
                    y label style={anchor=east},
                    xlabel={$x$},
                    ylabel={$\Psi(x,t)$},
                    width=9cm,
                    height=6cm]
        \addplot[domain=-1:18,samples=100,smooth,myblue,thick] {sin(deg(x))};
        \node[circle,fill,inner sep=1.5pt,black!80] at (axis cs:5*pi/2,1) {};
        \node[black!80] at (axis cs:5*pi/2,1.15) {$P_1$};
        \node[circle,fill,inner sep=1.5pt,black!80] at (axis cs:9*pi/2,1) {};
        \node[black!80] at (axis cs:9*pi/2,1.15) {$P_2$};
        \draw[dashed, black!80, thick] (axis cs:5*pi/2,1) -- node[above] {$\lambda$} (axis cs:9*pi/2,1);
        \end{axis}
    \end{tikzpicture}
    }   
    \caption[Phase velocity.]{The black points represent points of same phase and are separated by the wavelength $\lambda$.}
    \label{fig:6-2_phase_velocity}
\end{figure}

We can check that the expression for phase velocity in Eq.~(\ref{eq:6-2_phase_velocity}) makes sense.
Consider one of the peaks on the wave in Fig.~\ref{fig:6-2_phase_velocity}, say point $P_1$.
This point will move to point $P_2$, covering the distance of one wavelength $\lambda$.
This represents one full oscillation of the wave, therefore the time taken to cover this distance is one period $T$.
The wave propagates at constant speed $v$, given by the ratio $\lambda/T$.
Using Eq.~(\ref{eq:freq-period_k-wavelength}), we can express this ratio in terms of the angular frequency and the wave number,
\begin{equation}
    v = \frac{\frac{2\pi}{k}}{\frac{2\pi}{\omega}} = \frac{\omega}{k} = v_p,
\end{equation}
which agrees with the phase velocity of Eq.~(\ref{eq:6-2_phase_velocity}).

\emph{Group velocity.}
How is the speed of the wave affected when the wave itself is a superposition of multiple single-frequency waves?
Let's take things slow at first by considering only two interfering waves.,
\begin{equation}
    \Psi_1(x,t) = A \sin (\omega_1 t - k_1 x), \qquad \Psi_2(x,t) = A \sin (\omega_2 t - k_2 x).
\end{equation}
We have assumed that the two waves have same amplitude, and we set the initial phase to $\phi_0=0$ for convenience.
\begin{figure}[t]
    \centering
    \resizebox {0.9\textwidth} {!} {
    \begin{tikzpicture}
        \begin{axis}[xtick=\empty,
                    ytick={-2,2},
                    yticklabels={$-2A$,$2A$},
                    axis lines=middle,
                    enlarge x limits={abs=2},
                    enlarge y limits={abs=0.6},
                    x label style={anchor=north},
                    y label style={anchor=east},
                    xlabel={$x$},
                    ylabel={$\Psi(x,t)$},
                    width=12cm,
                    height=6cm]
        \addplot[domain=-1:100,samples=500,smooth,myblue,thick] {sin(deg(-2*x))+sin(deg(-2.1*x))};
        \addplot[domain=-1:100,samples=500,smooth,myorange,thick,dashed] {2*cos(deg(-0.5*(2.0-2.1)*x))};
        \addplot[domain=-1:100,samples=500,smooth,myorange,thick,dashed] {-2*cos(deg(-0.5*(2.0-2.1)*x))};
        \end{axis}
    \end{tikzpicture}
    }   
    \caption[Group velocity.]{Superposition of two sinusoidal waves. The fast oscillations (blue solid line) are modulated by a slowly-varying envelope (orange dashed line).}
    \label{fig:6-2_superposition}
\end{figure}
Interference of these two waves results in the following superposition, 
\begin{align} 
    \Psi(x,t) & = \Psi_1(x,t) + \Psi_2(x,t) \nonumber\\
     & = A\left[\sin \left(\omega_{1} t-k_{1} x\right)+\sin \left(\omega_{2} t-k_{2} x\right)\right] \nonumber\\
    & = 2 A \underbrace{\sin \left( \frac{\omega_{1}+\omega_{2}}{2} t-\frac{k_{1}+k_{2}}{2} x \right)}_{\text{fast oscillations}} \underbrace{ \cos \left( \frac{\omega_{1}-\omega_{2}}{2} t-\frac{k_{1}-k_{2}}{2} x \right)}_{\text{slow oscillations}}.
    \label{eq:fast_slow_oscillations}
\end{align}
We see that the superposition can be split into two terms.
The first term captures the fast oscillations of the superposition at frequency $(\omega_1+\omega_2)/2$.
These fast oscillations are pictured in blue in Fig.~\ref{fig:6-2_superposition}.
Similar to the case of a single-frequency wave, we can ask the question at what speed a point of constant phase travels to obtain the \textit{\textbf{phase velocity}} for the superposition.
This information is contained in the fast-oscillation term of Eq.~(\ref{eq:fast_slow_oscillations}),
\begin{equation}
    v_p = \frac{\omega_1 + \omega_2}{k_1 + k_2}.
\end{equation}
These fast oscillations are modulated by a slowly-varying envelope as seen in by the oranged dashed line in Fig.~\ref{fig:6-2_superposition}.
This is captured by the second term in Eq.~(\ref{eq:fast_slow_oscillations}).
The frequency of these slow oscillations is $(\omega_1-\omega_2)/2$.
This term also gives rise to \textit{\textbf{group velocity}},
\begin{equation}
    v_g = \frac{\omega_1 - \omega_2}{k_1 - k_2}.
\end{equation}
The group veocity tells us how fast the entire envelope is propagating, rather than just a single point of constant phase.
The phase and group velocities of a superposition are, in general, different.
There are scenarios where one is larger than the other but also where they are equal,
\begin{equation}
    v_p > v_g, \quad v_p = v_g, \quad v_p < v_g.
\end{equation}
Real wave packets and signals are never single-frequency waves but rather results of a superposition of a number of single-frequency components.
When we talk about the speed of such a wave, we usually refer to the group velocity.
It is also the group velocity that tells us how quickly a signal carries information.

\begin{figure}[t]
    \centering
    % \includegraphics[width=0.8\textwidth]{lesson6/6-2_superpostion_many.pdf} 
    \resizebox {0.9\textwidth} {!} {
    \begin{tikzpicture}
        \begin{axis}[xtick=\empty,
                    ytick={-11,11},
                    yticklabels={$-11A$,$11A$},
                    axis lines=middle,
                    enlarge x limits={abs=2},
                    enlarge y limits={abs=4},
                    x label style={anchor=north},
                    y label style={anchor=east},
                    xlabel={$x$},
                    ylabel={$\Psi(x,0)$},
                    width=12cm,
                    height=6cm]
        \addplot[domain=-1:100,samples=500,smooth,myblue,thick] {sin(deg(-2*x))+sin(deg(-2.1*x))+sin(deg(-2.2*x))+sin(deg(-2.3*x))+sin(deg(-2.4*x))+sin(deg(-2.5*x))+sin(deg(-2.6*x))+sin(deg(-2.7*x))+sin(deg(-2.8*x))+sin(deg(-2.9*x))+sin(deg(-3.0*x))};
        \end{axis}
    \end{tikzpicture}
    }   
    \caption[Superposition of more than two waves.]{Superposition of more than two sinusoidal waves resulting in a pulse.}
    \label{fig:6-2_pulse}
\end{figure}
To complete our discussion, let's look at the case when we have more than two single-frequency components interfering together.
Let's say we have a superposition of the following form,
\begin{equation}
    \Psi(x,t) = A [\sin(-2.0x) + \sin(-2.1x) + \ldots + \sin(-3.0x)].
    \label{eq:6-2_pulse}
\end{equation}
We have assumed that we are looking at the superposition frozen in time at $t=0$, hence the simple form.
The resultant superposition is pictured in Fig.~\ref{fig:6-2_pulse}.
Unlike before, we can observe long regions where destructive interference results in almost no disturbance.
Then the constructive interference kicks in and we observe a short \textbf{\emph{pulse}}\index{pulse}.
It is these pulses that can be used to carry information at the group velocity of the superposition.


%%%%%%%%%%%%%%%%%%%%%%%%%%%%%%%%%%%%%%%%%%%
\section{Interference with single photons}
\label{sec:6-3_interference_single_photons}
%%%%%%%%%%%%%%%%%%%%%%%%%%%%%%%%%%%%%%%%%%%

% double_slit standard
\begin{figure}[H]
   \centering
    % \includegraphics[width=0.8\textwidth]{lesson6/standard_double_slit.pdf}

    \resizebox {0.95\textwidth} {!} {
    \begin{tikzpicture}[nodal/.style={black!75,dashed,very thin},
            declare function={
            xnode(\n,\dn,\lam,\f) = \lam/\f*sqrt( \n^2*(\f^2-\dn^2)+\n*\dn*(\f^2-\dn^2)+\dn^2*\f^2/2-(\f^4+\dn^4)/4 );
            ynode(\n,\dn,\lam,\a) = (2*\n*\dn+\dn^2)*\lam/(2*\f);
            intensity(\y,\lam,\a,\L) = exp(-\y*\y/5)*cos(180*\a*\y/(2*\lam*sqrt(\L*\L+\y*\y)))^2;}]
  
    \def\L{3.8}       % distance between walls
    \def\H{5.4}       % total wall height
    \def\h{2.8}       % plane wave height
    \def\t{0.15}      % wall thickness
    \def\a{1.15}      % slit distance
    \def\d{0.20}      % slit size
    \def\N{21}        % number of waves
    \def\lambd{0.20}  % wavelength
    \def\R{\N*\lambd} % wave radius
    \def\Nlines{3}    % number of nodal lines
    \def\A{1.6}       % amplitude
    \def\nsamples{100}
    \def\ang{62}
  
    \begin{scope}
    \clip (-\t/2,-\H/2) rectangle (\L,\H/2);
    
    % NODAL LINES
    \draw[nodal] (0.08*\N*\lambd,0) -- (1.06*\R,0);
    \coordinate (NP0) at (\L,0);  % to avoid "Dimension too large error"
    \foreach \dn [evaluate={
                   \f=\a/\lambd;
                   \nmin=2.5+0.2*\dn; %0.501*(-\dn+\f)
                   \nmax=10; %(NP0)
                   \c=int(\dn<\f);
                   \y=\L/sqrt((\a/(\lambd*\dn))^2-1);
                 }] in {1,...,\Nlines}{
      \coordinate (NP+\dn) at (\L,\y);  % to avoid "Dimension too large error"
      \coordinate (NP-\dn) at (\L,-\y); % to avoid "Dimension too large error"
      \ifnum\c=1
        \draw[nodal,variable=\n,samples=\nsamples,smooth]
          plot[domain=\nmin:\nmax] ({xnode(\n,\dn,\lambd,\f)},{ynode(\n,\dn,\lambd,\f)})
          -- (NP+\dn);
        \draw[nodal,variable=\n,samples=\nsamples,smooth]
          plot[domain=\nmin:\nmax] ({xnode(\n,\dn,\lambd,\f)},{-ynode(\n,\dn,\lambd,\f)})
          -- (NP-\dn);
      \fi
    }
    
    % WAVES
    \foreach \i [evaluate={\R=\i*\lambd;}] in {1,...,\N}{
      \ifodd\i
        \draw[myred,line width=0.8] (0,\a/2)++(\ang:\R) arc (\ang:-\ang:\R);
        \draw[myred,line width=0.8] (0,-\a/2)++(\ang:\R) arc (\ang:-\ang:\R);
      \else
        \draw[myred!75,line width=0.1] (0,\a/2)++(\ang:\R) arc (\ang:-\ang:\R);
        \draw[myred!75,line width=0.1] (0,-\a/2)++(\ang:\R) arc (\ang:-\ang:\R);
      \fi
    }
  \end{scope}
  
  % PLANE WAVES
  \foreach \i [evaluate={\x=-\i*\lambd;}] in {0,...,5}{
    \ifodd\i
      \draw[myred,line width=0.8] (\x,-\h/2) -- (\x,\h/2);
    \else
      \draw[myred!75,line width=0.1] (\x,-\h/2) -- (\x,\h/2);
    \fi
  }
  
  % WALL
  \fill[wall]
    (\t/2,\a/2-\d/2) rectangle (-\t/2,-\a/2+\d/2)
    (\t/2,\a/2+\d/2) rectangle (-\t/2,\H/2)
    (\t/2,-\a/2-\d/2) rectangle (-\t/2,-\H/2)
    (\L,-\H/2) rectangle (\L+\t,\H/2);
  
  % SHADES
  \begin{scope}[shift={(1.08*\L,0)}]
    \def\yz{\L/sqrt((\a/\lambd)^2-1)} % m = +- 1/2
    \def\yZ{\L/sqrt((\a/\lambd/2)^2-1)} % m = +- 1
    \clip (0,-\H/2) rectangle (1.1*\A,\H/2);
    \fill[white] (0,-\H/2) rectangle++ (\A,\H); % to fill seams
    \foreach \i [evaluate={\n=0.5*\i;\yn=\L/sqrt((\a/(2*\lambd*\n))^2-1);
                 }] in {1,...,\Nlines}{
      \ifodd\i % if even
        \fill[myred] (0,{-\yn-0.1}) rectangle++ (\A,0.2); % to fill seams
        \fill[myred] (0,{ \yn-0.1}) rectangle++ (\A,0.2); % to fill seams
      \fi
    }
    \path[left color=black,right color=black,middle color=myred,shading angle={180}]
      (0,{-\yz}) rectangle (\A,{\yz});
    \foreach \i [evaluate={
                  \n=0.5*\i;
                  \m=0.5*(\i+1);
                  \yn=\L/sqrt((\a/(2*\lambd*\n))^2-1);
                  \ym=\L/sqrt((\a/(2*\lambd*\m))^2-1);
                  \dang=mod(\i,2)*180;
                 }] in {1,...,\Nlines}{
      \path[left color=black,right color=myred,shading angle={\dang}]
        (0,\yn) rectangle (\A,\ym);
      \path[left color=black,right color=myred,shading angle={180+\dang}]
        (0,-\yn) rectangle (\A,-\ym);
    }
  \end{scope}
  
  % INTENSITY
  \begin{scope}[shift={(1.1*\L+1.1*\A,0)}]
    \draw[->,thick] (-0.08*\A,0) -- (1.3*\A,0) node[right=0] {$\langle I\rangle$}; % I axis
    \draw[->,thick] (0,-0.52*\H) -- (0,0.54*\H) node[right] {$\delta$}; % y axis
    \draw[nodal] (NP0) --++ (0.15*\L+2.1*\A,0); % green nodal lines
    \foreach \i in {1,...,\Nlines}{ % green nodal lines
      \draw[nodal] (NP+\i) --++ (2,0);
      \draw[nodal] (NP-\i) --++ (2,0);
    }
    \draw[myred,thick,variable=\y,samples=\nsamples,smooth,domain=-\H/2:\H/2]
      plot({\A*intensity(\y,\lambd,\a,\L)},\y);
    \tick{0,{\L/sqrt((\a/(\lambd*1))^2-1)}}{180} node[right=0,scale=0.85] {$\pi$};
    \tick{0,{\L/sqrt((\a/(\lambd*2))^2-1)}}{180} node[right=0,scale=0.85] {$2\pi$};
    \tick{0,{\L/sqrt((\a/(\lambd*3))^2-1)}}{180} node[right=0,scale=0.85] {$3\pi$};
    \tick{0,{-\L/sqrt((\a/(\lambd*1))^2-1)}}{180} node[right=0,scale=0.85] {$-\pi$};
    \tick{0,{-\L/sqrt((\a/(\lambd*2))^2-1)}}{180} node[right=0,scale=0.85] {$-2\pi$};
    \tick{0,{-\L/sqrt((\a/(\lambd*3))^2-1)}}{180} node[right=0,scale=0.85] {$-3\pi$};
    \end{scope}
    \end{tikzpicture}
    }
    \caption[Double-slit experiment.]{Double-slit experiment with coherent light.}
    \label{fig:two-slit}
\end{figure}

Perhaps you have encountered the scenario of a double-slit experiment from one of your physics classes.
In Fig.~\ref{fig:two-slit}, coherent light produced by a laser is incident on a screen with two narrow slits.
The peaks of the waves are represented by thick lines while the troughs are represented by the thin lines.
The two slits diffract the incident light and act as two sources of spherical waves.
The waves propagate towards a second screen, where we observe an interference pattern of alternating bright and dark fringes.
Bright fringes are a result of constructive interference, where two peaks or two troughs meet.
Dark fringes on the other hand are a result of destructive interference where a peak is cancelled by a trough.
The dashed lines mark the points of constructive and destructive interference.

For example, consider the point right in the middle of the screen in Fig.~\ref{fig:two-slit}.
The distances travelled by light passing through the upper and lower slits are equal, therefore there is no phase difference between the two waves, $\delta=0$.
This leads to constructive interference and the average intensity $\langle I\rangle$ reaches its maximum.
This is also true for the cases when the phase difference is an even integer multiple of $\pi$.
We observe repeating bright fringes when the phase difference is $\delta=\pm 2\pi n$, for $n\in\{0,1,2,\ldots\}$.
On the other hand, when the phase difference is an odd integer multiple of $\pi$, that is when $\delta=\pm \pi n$, for $n\in\{0,1,2,\ldots\}$, the two waves cancel each other due to destructive interference.
This results in vanishing average intensity $\langle I \rangle$.
The bright fringes become fainter as we move away from the center of the screen due to diffraction effects modulating the average intensity.

Let's consider the scenario where we attenuate the laser light to such low levels that only single photons are incident at the first screen with the two slits.
In order to gain some intuition into the pattern that is observed on the second screen, let's analyze the case when one of the slits is covered, as shown in Fig.~\ref{fig:single-slit-attenuated}.
The incident photons can pass only through the open slit where they get diffracted.
Most photons pass through the slit without too much diffraction and hit the second screen at a point that is in line with the open slit's position.
Few photons do get diffracted significantly, resulting in an intensity distribution that shows a bring peak in the middle but decreases as we move away from this peak.
We can swap which slit is open and which is closed, and the observed pattern will be nearly the same.
The only difference will be that the peak of the average intensity would shift to be in line with the open slit.

\begin{figure}[t]
    \centering
    \resizebox {0.95\textwidth} {!} {
    \begin{tikzpicture}[
    nodal/.style={black!75,dashed,very thin},
    declare function={
      xnode(\n,\dn,\lam,\f) = \lam/\f*sqrt( \n^2*(\f^2-\dn^2)+\n*\dn*(\f^2-\dn^2)+\dn^2*\f^2/2-(\f^4+\dn^4)/2 );
      ynode(\n,\dn,\lam,\a) = (2*\n*\dn+\dn^2)*\lam/(2*\f);
      intensity(\y,\a) = exp(-(\y-\a/2)^2);
    }
  ]
  
  \def\L{3.8}       % distance between walls
  \def\H{5.4}       % total wall height
  \def\h{2.8}       % plane wave height
  \def\t{0.15}      % wall thickness
  \def\a{1.15}      % slit distance
  \def\d{0.20}      % slit size
  \def\N{21}        % number of waves
  \def\lambd{0.20}  % wavelength
  \def\R{\N*\lambd} % wave radius
  \def\Nlines{3}    % number of nodal lines
  \def\A{1.6}       % amplitude
  \def\nsamples{100}
  \def\ang{62}
  
  \begin{scope}
    \clip (-\t/2,-\H/2) rectangle (\L,\H/2);
    \newcommand*\Angles{{5,2,-3,20,-1,-10,0,4,-7,6,1,2,-3,1,15,-8,-6,2,-4,5,2,3}}
    % WAVES
    \foreach \i [evaluate={\R=\i*\lambd;}] in {1,...,\N}{
      \ifodd\i
        \filldraw[fill=myred,rotate around={\Angles[\i]:(0,\a/2)}] (\R,0\a/2) circle (0.08);
      \fi
    }
  \end{scope}
  
  % PLANE WAVES
  \foreach \i [evaluate={\x=-\i*\lambd;}] in {0,...,5}{
    \ifodd\i
      \filldraw[fill=myred] (\x,0) circle (0.08);
    \fi
  }
  
  % WALL
  \fill[wall]
    (\t/2,\a/2-\d/2) rectangle (-\t/2,-\a/2+\d/2)
    (\t/2,\a/2+\d/2) rectangle (-\t/2,\H/2)
    (\t/2,-\a/2-\d/2) rectangle (-\t/2,-\H/2)
    (\L,-\H/2) rectangle (\L+\t,\H/2)
    (\t,-\a/2+\d/2) rectangle (\t/2,-\a/2-\d/2);
  
  % SHADE
  \begin{scope}[shift={(1.08*\L,0)}]
    \clip (0,-\H/2) rectangle (1.1*\A,\H/2);
    \fill[black] (0,-\H/2) rectangle++ (\A,\H); % to fill seams
    \path[left color=black,right color=black, middle color=myred, shading angle=0] (0,-\H/2+\a/2) rectangle (\A,\H/2+\a/2);
  \end{scope}
  
  % INTENSITY
  \begin{scope}[shift={(1.1*\L+1.1*\A,0)}]
    \draw[->,thick] (-0.08*\A,0) -- (1.3*\A,0) node[right=0] {$\langle I\rangle$};
    \draw[->,thick] (0,-0.52*\H) -- (0,0.54*\H) node[right] {$\delta$}; % y axis
    \draw[myred,thick,variable=\y,samples=\nsamples,smooth,domain=-\H/2:\H/2]
      plot({\A*intensity(\y,\a)},\y);
  \end{scope}

    \end{tikzpicture}
    }
    \caption[Single-slit experiment with single photons.]{Single-slit experiment with single photons.}
    \label{fig:single-slit-attenuated}
\end{figure}

What happens when both slits are open?
Naive expectation would suggest that since the photons can pass through either slit, we should add the two intensity distributions obtained from the photons passing through the top slit and the photons passing through the bottom slit.
The resulting pattern should look like the one displayed in Fig.~\ref{fig:two-slit-expectation}.
Individual intensity distributions are shown in thin pink lines, while the total expected distribution is their sum, shown in thick red.
Our naive expectation does not display any interference pattern, rather it is a simple consequence of diffraction of photons from the two slits.
However, this is \emph{\textbf{not}} what is observed when the experiment is performed in a laboratory.

\begin{figure}[H]
    \centering
    \resizebox {0.95\textwidth} {!} {
    \begin{tikzpicture}[
    nodal/.style={black!75,dashed,very thin},
    declare function={
      xnode(\n,\dn,\lam,\f) = \lam/\f*sqrt( \n^2*(\f^2-\dn^2)+\n*\dn*(\f^2-\dn^2)+\dn^2*\f^2/2-(\f^4+\dn^4)/4 );
      ynode(\n,\dn,\lam,\a) = (2*\n*\dn+\dn^2)*\lam/(2*\f);
      intensity_single(\y,\a) = exp(-(\y+\a/2)^2)/1.5;
      intensity_total(\y,\a) = (exp(-(\y-\a/2)^2)+exp(-(\y+\a/2)^2))/1.5;
    }
  ]
  
  \def\L{3.8}       % distance between walls
  \def\H{5.4}       % total wall height
  \def\h{2.8}       % plane wave height
  \def\t{0.15}      % wall thickness
  \def\a{1.15}      % slit distance
  \def\d{0.20}      % slit size
  \def\N{21}        % number of waves
  \def\lambd{0.20}  % wavelength
  \def\R{\N*\lambd} % wave radius
  \def\Nlines{3}    % number of nodal lines
  \def\A{1.6}       % amplitude
  \def\nsamples{100}
  \def\ang{62}
  
  \begin{scope}
    \clip (-\t/2,-\H/2) rectangle (\L,\H/2);
    \newcommand*\Angles{{5,2,-3,20,-1,-10,0,4,-7,6,1,2,-3,1,15,-8,-6,2,-4,5,2,3}}
    % WAVES
    \foreach \i [evaluate={\R=\i*\lambd;}] in {1,...,\N}{
        \ifodd\i
        \filldraw[fill=myred,rotate around={\Angles[\i]:(0,\a/2)}] (\R,\a/2) circle (0.08);
        \else
        \filldraw[fill=myred,rotate around={\Angles[\i]:(0,\a/2)}] (\R,-\a/2) circle (0.08);
      \fi
    }
  \end{scope}
  
  % PLANE WAVES
  \foreach \i [evaluate={\x=-\i*\lambd;}] in {0,...,5}{
    \ifodd\i
      \filldraw[fill=myred] (\x,0) circle (0.08);
      % \filldraw[fill=myred] (\x,-\a/2) circle (0.08);
    \fi
  }
  
  % WALL
  \fill[wall]
    (\t/2,\a/2-\d/2) rectangle (-\t/2,-\a/2+\d/2)
    (\t/2,\a/2+\d/2) rectangle (-\t/2,\H/2)
    (\t/2,-\a/2-\d/2) rectangle (-\t/2,-\H/2)
    (\L,-\H/2) rectangle (\L+\t,\H/2);
  
  % SHADE
  \begin{scope}[shift={(1.08*\L,0)}]
    \clip (0,-\H/2) rectangle (1.1*\A,\H/2);
    \fill[black] (0,-\H/2) rectangle++ (\A,\H); % to fill seams
    \path[left color=black,right color=black, middle color=myred, shading angle=0] (0,-4) rectangle (\A,4);
    % \path[left color=none,middle color=myred, shading angle=0] (0,-\H/2-\a/2) rectangle (\A,\H/2-\a/2);
  \end{scope}
  
  % INTENSITY
  \begin{scope}[shift={(1.1*\L+1.1*\A,0)}]
    \draw[->,thick] (-0.08*\A,0) -- (1.3*\A,0) node[right=0] {$\langle I \rangle$}; % I axis
    \draw[->,thick] (0,-0.52*\H) -- (0,0.54*\H) node[right] {$\delta$}; % y axis
    \draw[myred!50,thin,variable=\y,samples=\nsamples,smooth,domain=-\H/2:\H/2]
      plot({\A*intensity_single(\y,\a)},\y);
    \draw[myred!50,thin,variable=\y,samples=\nsamples,smooth,domain=-\H/2:\H/2]
      plot({\A*intensity_single(\y,-\a)},\y);
    \draw[myred,thick,variable=\y,samples=\nsamples,smooth,domain=-\H/2:\H/2]
      plot({\A*intensity_total(\y,\a)},\y);
    \end{scope}

    \end{tikzpicture}
    }
    \caption[Double-slit experiment with single photons, naive expectation.]{Naive expectation of the observed pattern for a double-slit experiment with single photons. This pattern is not observed in a real experiment.}
    \label{fig:two-slit-expectation}
\end{figure}

The observed pattern is that of bright and dark interference fringes as shown in Fig.~\ref{fig:two-slit-single-photon-pattern}.
These are nearly the same as the ones we observed when coherent light was incident on the screen with the two slits.
This may not be that surprising upon a quick reflection.
We have started with strong coherent light in Fig.~\ref{fig:two-slit}, where we observed the interference pattern.
All we have done was we attenuated the light to the level of single photons, therefore there is no reason to expect the observed pattern to change.
It may take longer for the pattern to emerge clearly on the screen due to the attenuation of the light, but that is the only difference.

The situation gets a lot more interesting if we attenuate the incident light even further, such that with high probability there is only a single photon travelling between the two screens.
After enough time, the clear interference pattern of Fig.~\ref{fig:two-slit-single-photon-pattern} emerges yet again.
This is a lot spookier than the previous scenario since it seems a single photon can interfere with itself.
The incident photon has a two possibilities when it reaches the first screen in the form of the two open slits.
It is these possibilities that further interfere between the two screens, resulting in some places on the second screen having zero probability of detecting a photon.

\begin{figure}[t]
    \centering
    % \includegraphics[width=0.8\textwidth]{lesson6/block_neither_reality.png}
    \resizebox {0.95\textwidth} {!} {
    \begin{tikzpicture}[
    nodal/.style={black!75,dashed,very thin},
    declare function={
      xnode(\n,\dn,\lam,\f) = \lam/\f*sqrt( \n^2*(\f^2-\dn^2)+\n*\dn*(\f^2-\dn^2)+\dn^2*\f^2/2-(\f^4+\dn^4)/4 );
      ynode(\n,\dn,\lam,\a) = (2*\n*\dn+\dn^2)*\lam/(2*\f);
      intensity(\y,\lam,\a,\L) = exp(-\y*\y/5)*cos(180*\a*\y/(2*\lam*sqrt(\L*\L+\y*\y)))^2;
    }
  ]
  
  \def\L{3.8}       % distance between walls
  \def\H{5.4}       % total wall height
  \def\h{2.8}       % plane wave height
  \def\t{0.15}      % wall thickness
  \def\a{1.15}      % slit distance
  \def\d{0.20}      % slit size
  \def\N{21}        % number of waves
  \def\lambd{0.20}  % wavelength
  \def\R{\N*\lambd} % wave radius
  \def\Nlines{3}    % number of nodal lines
  \def\A{1.6}       % amplitude
  \def\nsamples{100}
  \def\ang{62}
  
    % PHOTONS
    \filldraw[fill=myred] (-2*\lambd,0) circle (0.08);
    \filldraw[fill=myred] (17*\lambd,0) circle (0.08);
  
  % WALL
  \fill[wall]
    (\t/2,\a/2-\d/2) rectangle (-\t/2,-\a/2+\d/2)
    (\t/2,\a/2+\d/2) rectangle (-\t/2,\H/2)
    (\t/2,-\a/2-\d/2) rectangle (-\t/2,-\H/2)
    (\L,-\H/2) rectangle (\L+\t,\H/2);
  
  % SHADE
  \begin{scope}[shift={(1.08*\L,0)}]
    \def\yz{\L/sqrt((\a/\lambd)^2-1)} % m = +- 1/2
    \def\yZ{\L/sqrt((\a/\lambd/2)^2-1)} % m = +- 1
    \clip (0,-\H/2) rectangle (1.1*\A,\H/2);
    \fill[myred] (0,-\H/2) rectangle++ (\A,\H); % to fill seams
    \foreach \i [evaluate={\n=0.5*\i;\yn=\L/sqrt((\a/(2*\lambd*\n))^2-1);
                 }] in {1,...,\Nlines}{
      \ifodd\i % if even
        \fill[myred] (0,{-\yn-0.1}) rectangle++ (\A,0.2); % to fill seams
        \fill[myred] (0,{ \yn-0.1}) rectangle++ (\A,0.2); % to fill seams
      \fi
    }
    \path[left color=black,right color=black,middle color=myred,shading angle={180}]
      (0,{-\yz}) rectangle (\A,{\yz});
    \foreach \i [evaluate={
                  \n=0.5*\i;
                  \m=0.5*(\i+1);
                  \yn=\L/sqrt((\a/(2*\lambd*\n))^2-1);
                  \ym=\L/sqrt((\a/(2*\lambd*\m))^2-1);
                  \dang=mod(\i,2)*180;
                 }] in {1,...,\Nlines}{
      \path[left color=black,right color=myred,shading angle={\dang}]
        (0,\yn) rectangle (\A,\ym);
      \path[left color=black,right color=myred,shading angle={180+\dang}]
        (0,-\yn) rectangle (\A,-\ym);
    }
  \end{scope}
  
  % INTENSITY
    \coordinate (NP0) at (\L,0);  % to avoid "Dimension too large error"
    \foreach \dn [evaluate={
                   \f=\a/\lambd;
                   \nmin=2.5+0.2*\dn; %0.501*(-\dn+\f)
                   \nmax=10; %(NP0)
                   \c=int(\dn<\f);
                   \y=\L/sqrt((\a/(\lambd*\dn))^2-1);
                 }] in {1,...,\Nlines}{
      \coordinate (NP+\dn) at (\L,\y);  % to avoid "Dimension too large error"
      \coordinate (NP-\dn) at (\L,-\y); % to avoid "Dimension too large error"
    }
    
  \begin{scope}[shift={(1.1*\L+1.1*\A,0)}]
    \draw[->,thick] (-0.08*\A,0) -- (1.3*\A,0) node[right=0] {$\langle I\rangle$}; % I axis
    \draw[->,thick] (0,-0.52*\H) -- (0,0.54*\H) node[right] {$\delta$}; % y axis
    \draw[myred,thick,variable=\y,samples=\nsamples,smooth,domain=-\H/2:\H/2]
      plot({\A*intensity(\y,\lambd,\a,\L)},\y);
  \end{scope}

    \end{tikzpicture}
    }
    \caption[Double-slit experiment with single photons.]{Interference pattern remains even if the photons arrive slowly enough that a single photon is present between the screens.}
    \label{fig:two-slit-single-photon-pattern}
\end{figure}



%%%%%%%%%%%%%%%%%%%%%%%%%%%%%%%%%%%%%%%%%%
\section{Interference with qubits}
\label{sec:6-4_interference_single_qubits}
%%%%%%%%%%%%%%%%%%%%%%%%%%%%%%%%%%%%%%%%%%

We have finished Sec.~\ref{sec:6-3_interference_single_photons} by saying that a single photon's all possible routes to the second screen interfere together, resulting in a pattern displaying bright and dark fringes.
In this Section, we will learn more about this concept in its simplest form.
But before we do that, let's consider interference from a more abstract perspective of a single qubit.

Consider the action of a Hadamard operation on a qubit.
We will consider two initial states, state $\ket{0}$ and state $\ket{1}$.
The initial states and the Hadamard operation can be represented in matrix form as follows,
\begin{equation}
    |0\rangle = \begin{pmatrix} 1 \\ 0 \end{pmatrix}, \quad
    |1\rangle = \begin{pmatrix} 0 \\ 1 \end{pmatrix}, \quad
    H = \frac{1}{\sqrt{2}} \begin{pmatrix} 1 & 1 \\ 1 & -1 \end{pmatrix}.
\end{equation}
If the initial state is $\ket{0}$, applying the Hadamard operation creates an equal superposition of the states $\ket{0}$ and $\ket{1}$,
\begin{equation}
    |0\rangle \longrightarrow H|0\rangle=\frac{1}{\sqrt{2}}(|0\rangle+|1\rangle).
\end{equation}
We can apply the Hadamard again, this time to the superposition,
\begin{equation}
    H\left[\frac{1}{\sqrt{2}}(|0\rangle+|1\rangle)\right] = \frac{1}{2}(|0\rangle+|1\rangle+|0\rangle-|1\rangle) = |0\rangle.
    \label{eq:qubit_interference}
\end{equation}
This may seem like a trivial calculation.
The Hadamard is a unitary operation, therefore applying it twice should return the initial state.
But it is worth pausing briefly and thinking about the calculation in Eq.~(\ref{eq:qubit_interference}).
We can observe that both $|0\rangle$ terms have the same probability amplitude, $+1$.
This is another example of constructive interference.
On the other hand, the terms $|1\rangle$ have probability amplitudes $+1$ and $-1$, leading to destructive interference.

We can draw a parallel between the double-slit experiment with single photons and the interference of a single qubit in Eq.~(\ref{eq:qubit_interference}).
In the double-slit experiment, the photon had infinitely many possibilities where to hit the second screen.
Due to interference, some of these possibilities had finite probability of occurring, while the probability vanished for others.
In the case of a single-qubit interference, there are only two possibilities, represented by states $|0\rangle$ and $|1\rangle$.
It is the \textbf{\emph{interference of probability amplitudes}} that determines whether the state of the qubit is $|0\rangle$ or $|1\rangle$.

Let's consider two transformations that are not hermitian conjugates of each other, denoted by BS1 and BS2,
\begin{equation}
    \text{BS}1 = \frac{1}{\sqrt{2}} \begin{pmatrix} 1 & 1 \\ 1 & -1 \end{pmatrix}, \quad
    \text{BS}2 = \frac{1}{\sqrt{2}}\begin{pmatrix} -1 & 1 \\ 1 & 1 \end{pmatrix}, \quad
    \text{BS}2 \cdot \text{BS1} \neq I.
    \label{eq:BS_transformations}
\end{equation}
The operation BS1 is none other than the Hadamard operation.
The choice of notation will become clear in a short while. 
BS2 appears to be similar to BS1, the difference being that -1 is not located in the bottom right, but instead is in the top left.
You can check for yourself that applying these gates in sequence is not the same thing as doing nothing.
In particular, the product of BS2 and BS1 is not equal to the identity.
To see this, let's apply BS1, followed by BS2, to the initial state $|1\rangle$,
\begin{align}
    \text{BS}2 \cdot \text{BS}1 |1\rangle & = \text{BS}2 \cdot \frac{1}{\sqrt{2}} \begin{pmatrix} 1 & 1 \\ 1 & -1 \end{pmatrix} \cdot \begin{pmatrix} 0 \\ 1 \end{pmatrix} \nonumber\\
    & = \text{BS}2 \cdot \frac{1}{\sqrt{2}} \begin{pmatrix} 1 \\ -1 \end{pmatrix} \nonumber\\
    & = \frac{1}{2} \begin{pmatrix} -1 & 1 \\ 1 & 1 \end{pmatrix} \begin{pmatrix} 1 \\ -1 \end{pmatrix} \nonumber\\
    & = \frac{1}{2} \begin{pmatrix} -2 \\ 0 \end{pmatrix} \nonumber\\
    & = -|0\rangle = |0\rangle.
    \label{eq:BS2-BS1}
\end{align}
The global phase in the last line can be ignored as it has no physically observable consequence.
Application of BS1 creates a superposition of states $|0\rangle$ and $|1\rangle$.
Subsequent application of BS2 results in destructive interference of probability amplitudes for $|1\rangle$.

The previous discussion of qubit interference may seem a little abstract.
Now is the time to make more concrete by considering an optical instrument called the \textbf{\emph{Mach-Zehnder interferometer}}\index{Mach-Zehnder interferometer}\index{interferometer}, as shown in Fig.~\ref{fig:mach-zehnder}.
\begin{figure}[t]
   \centering
    \includegraphics[width=0.8\textwidth]{lesson6/mach_zehnder.pdf}
    \caption{Mach-Zehnder interferometer.}
    \label{fig:mach-zehnder}  
\end{figure}
It consists of two beam splitters, BS1 and BS2, two mirrors, and two detectors.
The experiment that we would like to carry out is as follows.
We use a source of light as input for the beam splitter BS1.
The light can enter the beam splitter either from the top port or from the bottom one.
The detectors are placed at the two output ports of beam splitter BS2.
We are interested in the detection pattern of D0 and D1 based on the input pattern.
For the particular input in Fig.~\ref{fig:mach-zehnder}, only detector D0 registers a click, the light is never detected by D1.

The Mach-Zehnder interferometer can be used to encode a qubit.
An incident photon, regardless of the input port, has two possible paths that it can take to BS2.
If it is in the upper half of the interferometer then we say the state of the qubit is $|0\rangle$, as shown in Fig.~\ref{fig:m-z-upper}.
If the photon is found in the bottom half of the interferometer, the state of the qubit is $|1\rangle$ as shown in Fig.~\ref{fig:m-z-lower}.
We say that the qubit is \textbf{\emph{spatially encoded}}\index{spatial encoding}.
\begin{figure}[t]
   \centering
    \includegraphics[width=0.8\textwidth]{lesson6/0_ket_botttom.pdf}
    \caption[State $|0\rangle$ in the MI interferometer.]{State $\ket{0}$ is represented by a photon in the upper path.}
    \label{fig:m-z-upper}
\end{figure}
\begin{figure}[t]
   \centering
    \includegraphics[width=0.8\textwidth]{lesson6/1_ket_bottom.pdf}
    \caption[State $|1\rangle$ in the MI interferometer.]{State $\ket{1}$ is represented by a photon in the lower path.}
    \label{fig:m-z-lower}
\end{figure}

Let's consider the initial state to be $\ket{1}$, meaning the photon enters the interferometer from the bottom port.
Now it is more clear why we have called those previous transformations in Eq.~(\ref{eq:BS_transformations}) BS1 and BS2.
They correspond to the mathematical description of how the beam splitters affect the probability amplitudes of our qubit~\footnote{In fact, the quantum mechanical description of a beam splitter always has to involve writing down the state of both of the input ports to the beam splitter; here, we are being a little loose with the description. BS2 in this figure shows the correct accounting for photons coming from above and below the beam splitter, but for BS1 we have ignored the fact that we need to count the number of photons coming in from both above and below it, as well.  In future courses, you will see a more rigorous treatment.}.
We proved in Eq.~(\ref{eq:BS2-BS1}) that if we first act on the input qubit $|1\rangle$ with BS1, and subsequently with BS2, then the final state of the qubit will be $|0\rangle$.
This means that a photon in the bottom input port of BS1 will be always detected by detector D0 after passing through the Mach-Zehnder interferometer.
This is one of the most bare-bones demonstrations of single-photon interference and also shows how the abstract calculation in Eq.~(\ref{eq:BS2-BS1}) connects to the real world.

Let's do a simple test to better understand the behavior of the Mach-Zehnder interferometer.
Let's put some absorbing material and block the possibility of the photon going through the lower half of the Mach-Zehnder Interferometer, as in Fig.~\ref{fig:m-z-blocked}.
An input photon at BS1 in the bottom port has a chance to get reflected at BS1.
If it does, it hits the block, gets absorbed and we don't get any clicks at either detector.
However, if it gets transmitted at BS1, it bounces off the upper mirror and it is incident onto the second beam splitter, where again, it has an equal probability of being reflected or passing through the beam splitter.
Therefore, it has a probability of being detected by both detector D0 and the detector D1.
Effectively, by blocking the bottom path of the Mach-Zehnder Interferometer, we have prevented interference from taking place at BS2.
That is why we see both possibilities D0 and D1.
\begin{figure}[t]
   \centering
    \includegraphics[width=0.8\textwidth]{lesson6/bottom_blocked.pdf}
    \caption[Lower path blocked.]{Lower path blocked. Because there is no photon coming from the lower path at BS2, no interference occurs, and any photon coming from above has a 50/50 chance of being directed to either detector.}
    \label{fig:m-z-blocked}
\end{figure}


\newpage
\begin{exercises}

\exer{
\emph{Visualization of wave propagation.}
Significant portions of this Chapter discuss propagation of waves in time. Naturally, this is can be only partially visualized in the form of a book. This exercise gives you the opportunity to write some basic code that will create an animation depicting interference of two waves in time. You can choose any package or library. If you are not familiar with creating animations, we suggest you look in to {\tt Python} library called {\tt matplotlib}.
\subexer{
Begin by replicating Fig.~\ref{fig:6-2_superposition}. Choose any two sinusoidal waves, but make sure you can clearly see the fast and slow oscillations.
}
\subexer{
Create an animation that clearly shows how the interference of the two waves propagates in time.
}
\subexer{
Using a small square, mark a point of constant phase on the interference pattern.
Using a small circle, mark a point on the envelope that modulates the oscillations.
For your chosen parameters, which point has higher velocity?
}
\subexer{
Change the parameters and make sure you can create animations that demonstrate the following three cases clearly: $v_p>v_g$, $v_p=v_g$, $v_p<v_g$.
}
\subexer{
Can you find parameters for the two waves, such that the phase velocity of the resulting wave is negative, $v_p<0$?
}
}

\exer{
\emph{Single-qubit interference.}
Consider the Mach-Zehnder interferometer of Fig.~\ref{fig:mach-zehnder}.
\subexer{
Show that BS1 and BS2 are not hermitian conjugates of each other, that $\text{BS}1\cdot\text{BS}2\neq I$.
}
\subexer{
What are the detection probabilities of the D0 and D1 if the input state is an equal superposition $(|0\rangle + |1\rangle)/\sqrt{2}$?
}
\subexer{
Can you think of a way to physically implement such an input state?
}
}

\exer{
\emph{Destroying the interference.}
We said that the interference at BS2 can be destroyed by placing an obstacle in one of the arms of the Mach-Zehnder interferometer, see Fig.~\ref{fig:m-z-blocked}. Let's better quantitative intuition into this scenario by replacing the obstacle with another detector, labelled by D2.
\subexer{
What is the state of the qubit immediately after BS1?
}
\subexer{
What is the probability of detector D2 detecting a photon?
}
\subexer{
What is the state of the qubit given that detector D2 did not click?
}
\subexer{
What are the individual click probabilities for the three detectors?
}
\subexer{
If we input a photon in the top port of BS1, where is it most likely going to be detected?
}
}

\end{exercises}

