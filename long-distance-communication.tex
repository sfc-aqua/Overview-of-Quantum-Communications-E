\chapter{Long-distance communication}

\section{Introduction}

00:00
hi and welcome to lesson 11 on long distance communication
in this lesson we will give you the context of how we can communicate over
long distances in classical networks and see how that
fits within the context of quantum communication as well step one introduction
so let's consider a little bit of historical background to
long distance communication and let's go to the year of 1852.
so just before the widespread of the telegraph communication was extremely slow
to give you some idea how slow it was if you wanted to send a letter from london
let's say to new york it took around 12 days
to arrive if you wanted to send it to sydney then you had to wait 73 days for
the letter to be delivered so the telegraph immediately

00:01
sparked the dream of a transatlantic submarine cable because people realize
that if they can use the telegraph to communicate over land so quickly they
should be able to connect internets with submarine cables as well
so in 1850 uh was laid the very first submarine telegraphic cable connecting
england and france that worked fine so immediately people
wanted to lay it across the atlantic ocean
and for the next 15 years there are many failed attempts
but what this sparked was a real focused mathematical analysis of very long
distance communication then finally in 1866 the first
successful transatlantic telegraphic cable was laid and used and by 1871
only few years later all the continents except for the
for antarctica were connected connected and then between 1902 and 1906

00:02
the trans-pacific cables were being laid and they connected the mainland us with
hawaii guam and later philippines and finally japan
on the other hand after the telegraphic cable came telephone cables and in 1956
we had the first transatlantic telecommunications cable called t81
and this telephone cable was able to handle 51 consecutive telephone
conversations compare that with two decades later the
last of the telephone cables t87 which could handle a staggering 8
000 telephone conversations consecutively and after this came the fiber optic
cables as we saw uh in the previous lesson in 1988
one of the first five the first fiber optic cable called t88 was laid
and increased the bandwidth to 40 000 the equivalent of 40 000 telephone

00:03
communications telephone conversations so in this lesson what we are going to
mainly focus on is what were the main considerations when designing these cables
and particularly bandwidth and noise so bandwidth tells us approxim basically
how much information can a cable or fiber carry and this depends on both the
physical properties of the fiber itself as well as how
clever we are when it comes to encoding this
information before it gets transmitted so with modern techniques of dense
wavelength division multiplexing we're able to reach some staggering speeds
for example in a standard cable over the distance of
600 kilometers we can reach speeds of something like 65 terabits per second
in a more specialized cable and over shorter distances
this can be increased to over 150 terabits per second

00:04
and over very short distances uh we achieved speeds of one
petabit per second i mean this is something incredible a terabit it's just
for your reference it's 10 to the 12 bits well petabyte is 10 to the 15 bits
of course no system is 100 efficient and whatever we put in is not
what we're going to get out at the end of the
cable so we must consider what are the main sources of loss in optical fibers
and we will consider these following five losses
one is dispersion the next one is absorption scattering bending and coupling
the result of all of these sources of loss is that our signal becomes attenuated
meaning if we put some signal as an input with some power
what we get out as the output will have less power so it's very important to

00:05
know how much of the signal becomes attenuated and how we can
prevent this attenuation or combat this attenuation
so in this lesson we will first in step 2 talk about the dispersion in optical
fibers then we will consider the other sources of attenuation
then we will move on to overcoming these losses
and finally we will consider what these sources of losses present
in terms of challenges for in the context of quantum communication

\section{Mode dispersion}

00:00
step 2 mode dispersion mode dispersion is our first source of
losses in the fiber that we're going to consider
so let's consider the propagation of different modes
in a multimode fiber as we said a multimode fiber can contain many modes
all traveling with different parts for example you can have the axial path
which travels directly down the fiber or you can have other
modes which are being totally reflect internally reflected within the fiber
like that and as you can see the different modes
they propagate at different speeds because they have to traverse
different lengths so it's very important to compute what is the time difference
introduced by by the different modes traveling at different speeds down the
fiber and this as you can expect depends on the launch angle so depending on the

00:01
angle with which it is coupled to the fiber the path that it takes
will be different and therefore the time
that it takes to traverse the fiber will be also different
so the fastest mode as you can clearly see
travels directly down the fiber this is known as the axial mode
because it travels down the axis of the fiber and the slowest mode is
that one that is incident on the cladding just at the critical angle meaning it
just gets internally reflected so let's compute the time delay between
the fastest mode and the slowest mode why that is important is that your
initial digitized signal may look something like this it's very
sharp and it's very easy to read out when you have a zero and when you have a

00:02
one but as the different modes propagate down the fiber the whole
package will spread and disperse it will look something like this
what this means is that the readability of the output signal
worsens meaning it's very it's more easy to make a mistake
when you're actually trying to decode your signal after it propagates through
the fiber so here we are and let's start calculating the time delay between the
fastest mode and the slowest mode we're going to consider
some length capital l that the fastest axial mode traverses
and the time that it takes for this mode to traverse this distance
we will call t min so again it's the time for the axial ray to travel length
l of the fiber and we can very easily compute it the this the

00:03
t min is just the distance traversed over the speed
of light in the fiber and we have seen that the speed of the light in the fiber
is determined from the refractive index of the material
that the fiber is made out of so vf is equal to c over nf we can
substitute that into our expression for t min
and we obtain the following t min is equal to length
capital l times the refractive index of the fiber
all divided by the speed of light in vacuum c
now let's consider the time that it takes
for some non-axial mode to traverse the sa some distance capital l here
the distance that it travels is not actually capital l
but it's some other small l it's this distance that it
travels here plus this distance that it travels there

00:04
so we can break it into two parts small l is equal to small l1 plus small l2
let's start by computing the small l1 and this division into l small l1 and
small l2 also divides our capital l into capital l1 plus capital l2
so the angle over here we're going to denote as theta r for angle of refraction
and we know from snell's law so from basic trigonometry that this length small
l1 is equal to this capital l1 divided by the cosine of the refraction
angle theta r that gives us l1 now how do we compute
l2 well we can use a nice neat little trick and we can actually reflect
this distance this arrow here which describes our distance l2

00:05
over there so we are extending the distance like this
and then l2 is given just as capital l2 divided by cosine
of the refraction angle theta r so we can just sum them together and we see
that the length the path length that the
light ray traverses l is just the sum of these two terms small l1 plus small l2
so it's given just as capital l over cosine of theta r so we see
that the path length of the light ray only depends on the initial angle
described by theta r and the axial length capital l over here
so from that we can also compute the time that it takes to traverse

00:06
this length small l and again it's given by
small l divided by the speed of light in the in the fiber which substituting
all for l and for vf we obtain this following expression
so it's the axial length capital l times the refractive index
over fiber all divided by the speed of light in vacuum
times the cosine of the refraction angle theta r
so let's get back to computing t max t max is the time to travel a critical
non-axial path as we said that such a path where
we are just being reflected internally in meaning that the angle of incidence
on the cladding is theta c our critical angle
and we have seen in previous lessons that sine of theta c is related to the

00:07
cosine of the refraction angle over here so t max can then be computed as l
times the refractive index of the fiber squared
over the speed of light in vacuum times the refractive index of the cladding
now it's very easy to just put the expressions for tmax and tmin together
and we obtain the time delay as follows and it's given by this expression over
here now we are going to plug in some numbers and see why this
time delay is actually important so let's consider a particular example
where the refractive index of the fiber is given by
1.5 and the refractive index of the cladding is a little bit less and we will
consider the value of 1.489 we can plug these numbers into our
previous expression for the time delay and obtain that the time delay per one
kilometer is given as follows it's 37 nanoseconds

00:08
per kilometer so this doesn't seem like a very long time delay
but we will see what effects it has so the speed of light in this fiber is given
by the refractive index of this fiber so it's vf is equal to
c the speed of light in vacuum divided by the refractive index
which is just 2 times 10 to the 8 meters per second
that means that our pulse spreads over a distance as it travels through
the fiber because as we said the modes are becoming dispersed
and we can quantify it and obtain the following value
it's 7.4 nanometers per kilometer so every kilometer that our signal travels
it becomes more and more spread by this distance 7.4 nanometers
as we said this reduces the readability of the output signal

00:09
because our packages are becoming more spread we are losing the sharpness of
our signal and it's more difficult to decode our signal
so in order to be able to read our output signal we may demand that the pulses
which are coming out of our fiber are separated by
twice the value of by which of the spreading so this means that in order
for the for the attenuated signal for the dispersed signal that's coming
out of the fiber uh to be separate the pulses to be
separated by four 14.8 nanometers we require that the input
pulses are also separated by at least 14.8 nanometers this on
inadvertently places a limit on how fast we can
transmit uh information because the pulses have to be separated by a certain

00:10
amount so we cannot place them more closely together and we cannot
um send the information at a higher frequency
so dispersion has a direct consequence on the frequency of the input signal
meaning that it also limits our bandwidth

\section{Attenuation}

00:00
step three attenuation in this step we will consider the rest
of the sources of losses that we mentioned in step one
so let's look at absorption absorption happens due to interaction of light with
the material of the fiber there are two types of absorption there
is the intrinsic absorption where even if we have a perfect fiber
the intrinsic absorption is responsible for attenuating
the signal so this is something that we cannot get rid of
it's just present uh in there and it's due to the interaction between the
photons of the signal and the electrons in the material
the electrons absorb the photons from the signal and they become
excited therefore the overall signal the power of the signal decreases and
this is just an intrinsic property of the fiber and we cannot really affect it
it's something that we just have to live with and accept

00:01
but luckily it's not a very significant source of error
especially when we compare it to the other type of absorption
which is extrinsic extrinsic absorption this is due to some impurities that are
present in the fiber and these impurities are introduced
during the manufacturing process so this on one hand is a much more
significant source of absorption and therefore attenuation of the signal
but we can control it by perfecting our manufacturing process
and generally uh during the the manufacturers are trying to keep
the the amount of impurities that are present in the fiber below one percent
the other source of attenuation is due to scattering
this is similar to absorption but the light is not only absorbed it is
also re-radiated and re-emitted back into the fiber at random directions

00:02
now this is done again due to the impurities
in the fiber and can be controlled by the manufacturing process
and limiting the amount of impurities that are present in the fiber
and there are many different types of scattering there's a linear scattering
there is non-linear scattering but the details will not be discussed
in at the current in the current lesson and the final two sources of attenuation
are bending and coupling bending is when we actually
physically bend the fiber remember we said that uh
the angle of incidence is crucial uh for total internal reflection to take place
and therefore for the signal to propagate down the fiber
and then if we bend it we can have a situation where at first uh the
uh the light ray is traveling down the fiber it's becoming totally

00:03
uh internally reflected but then it hits the angle at the bend
and suddenly it it gets reflected in such a way that it
cannot satisfy the condition for total internal reflection anymore
and it just gets absorbed or it leaves the fiber
and generally the manufacturers specify some minimum bending radius
and typically it's around 10 to 20 times the diameter of the fiber
and the other source of coupling as a source of error
is the coupling error this is inevitably we will have to join two
fibers together and if there's a cap gap between the fibers
then what can happen is the light can of course escape through the gap
or another source of coupling error is when the fibers are not aligned together
so even if there is no gap but they are slightly misaligned like
here in this bottom example then light is allowed to escape

00:04
and leave the fiber therefore the overall signal will become attenuated so
those were the main sources of losses now we're going to discuss how to quantify
the attenuation in a fiber so as the signal propagates through the
fiber it loses power so we need to quantify how much power it lost
and this is done by a unit called the decibel
the decibel designates the ratio of the two power levels
the power in and the power out so the number of decibels is defined as
the following expression it's minus 10 times logarithm base 10 of the ratio
of power out and power in now this minus is here because this ratio
is less than one remember the power out has to be less than power in
because the signal is getting attenuated and these tens are here

00:05
because we are talking about decibels deci means 10 so therefore
we are multiplying by 10 over here and sets the overall
the overall scale now why are why do we have
this logarithm here and that's because we will be considering a large span of
orders of magnitude between the power in and the power out and
therefore if we take the logarithm then it will produce a sort of generally
nice nice scale for the number of decibels
so for example if we have the ratio of power out to power in
as 1 to 10 meaning that the 90 of the power becomes attenuated then
this corresponds to 10 decibels as you can convince yourself by
substituting for p naught over p i into this formula if the ratio is 1 to a 100
then this corresponds to 20 decibels if it's 1 to 1 000

00:06
this corresponds to 30 decibels so you can see that this ratio
on the of the power out over powering is getting smaller by an order of
magnitude whereas here the increase in terms of the decibel is just linear
and this is due to the definition in terms of the logarithm for the decibels
we can also define the attenuation parameter alpha
and this is the number of decibels per kilometer
so alpha is just our previous expression divided by the length of the fiber
l and we can now rearrange it bring l to the other side bring the minus to
the other side and divide by 10 to obtain this expression and we can also
get an expression for the fraction of the power out over powering

00:07
being equal to 10 raised to the power of negative alpha times l over 10.
so we said in the previous lessons that in 1970
the optical fiber managed to transmit one percent of the power that was put in
over a distance of one kilometer we can now just plug it into this formula
well this formula over here and we will obtain that
alpha the attenuation parameter is 20 decibels
per kilometer and then we said that two decades later the transmission rose
to around 96 percent over kilometer which corresponds to attenuation level
of 0.18 decibels per kilometer so let's plot these two values to see
what they look like here on the horizontal axis we are
plotting the length of the fiber through which our signal is

00:08
traveling and on the vertical axis we've got the ratio of the power out over
power in so if the length of the fiber is zero then of course
the ratio is one our signal hasn't traveled anything so it didn't have
chance to be attenuated but as it travels through this blue line
corresponds to the attenuation levels that were achieved in the 1970
so the alpha is set to be 20 decibels per kilometer and this orange line is the
attenuation parameter in the 1990 so we can see that how quickly how quickly the
ratio approaches zero for the very large attenuation
parameter alpha equals to 20 whereas for the very low attenuation parameter
it decreases much more slowly now the question is knowing what the
main sources of loss are and knowing how to quantify them how can

00:09
we protect against these losses how can we counteract these losses in our fiber
you

\section{Overcoming losses}

00:00
step 4 overcoming losses so we have seen what the sources of losses are
now let's discuss how we can overcome them
let's look at this persian first that's probably one of the easiest
we saw that the signal and the different modes become dispersed in a multi-mode
fiber but this is not the case in a single mode fiber there's only one mode
so there's nothing to be dispersed therefore if
in the in some situation mode dispersion is a big source of error
it's best to switch to a single mode fiber we saw that absorption and scattering
can be improved by manufacturing process because the main source there are the
impurities in the fiber so if we can limit the
the amount of impurities in the fiber we can also limit
these two sources of losses furthermore bending just don't bend your

00:01
fiber unless you absolutely have to and even then be very mindful how much
you bend your fiber and finally uh coupling errors can be limited eliminated
at least partially by ensuring that the fibers are aligned properly and there is
no gaps between them but even if we try to do all of these things
there will be still some attenuation and some losses
so let's let's go back to our expression for
how much power we lose for some input power
over a distance l with some attenuation parameter alpha
and always this alpha no matter how careful we are about the manufacturing
and using single mode fibers and aligning our coupler coupling
coupling the fibers together in a proper way we will always
have the attenuation parameter alpha be non-negative

00:02
therefore there will be some attenuation on the signal
and in the previous step we saw how much the signal becomes attenuated over a
short distance of 5 kilometers if the alpha parameter is set to 20
or something very low like 0.18 so it seems that over the distance of 5
kilometers the signal becomes attenuated but not by
that much it's still around eighty percent of the signal gets through
but five kilometers is not a very long distance let's try to extend this
distance and see what happens so let's extend it to
something like 100 kilometers and immediately you see that even
even in a low loss fiber that has alpha parameter of 0.18
eventually the signal will go to nearly zero after 20 kilometers it's around 44

00:03
of the initial power after only 50 kilometers it drops to 13 percent
and at the distance of 100 kilometers it is virtually zero
and hundred kilometers is not such a long distance we are talking about
thousands of kilometers so the question is how is it done how can we use fiber
optics to actually transmit signals over such extreme long distances
and that's done with the help of something called repeaters
and they are devices which are used to boost the
signal strength and there are many different repeaters
based on many different physical principles the ones that we are going to
talk about in this lesson are called erbium doped
fiber amplifiers or edfa for short how this works is that you introduce
erbium atoms into the fiber then you pump these atoms meaning
that we excite these atoms into their higher energy state

00:04
creating population inversion then as the weak
signal that we are aiming to boost goes in it can stimulate emission from these
erbium-doped fibers from this erbium atoms and we know that in the process of
stimulated emission the photons that are emitted are basically
of the same kind as the incoming signal photons
they have the same phase same coherence and they travel most importantly in the
same direction so basically this is using the principle that's behind
lasing to boost our weak signal and in this process we obtained a signal that
becomes amplified and can travel uh further distance before it needs
amplification again and just to remind you we will encounter this stem repeater
in the next lesson as well where we will be talking about quantum repeaters
but they work on a very different basis as classical repeaters here we are still

00:05
talking about classical repeaters so this is the image that we have in
mind we've got our fiber over here and we've got some portion of the fiber
where we introduced erbium atoms into it we are pumping them
strongly such that we create population inversion
and as the weak signal comes in it stimulates emission from this
red part where we where we introduce the erbium
atoms and in that process becomes boosted again
and we allow it to travel further for another 50 kilometers or so before it
needs boosting again but there is one problem with this
approach that we have to keep in mind and that is that edfas don't only
amplify the signal they can amplify the noise as well
let's see how that works some of the erbium atoms

00:06
they will actually decay via spontaneous emission remember we are exciting them
and they are not just waiting around for the signal to come in
and then cause stimulated emission sometimes they get a little bit impatient
and they decay spontaneously photons that originating
from spontaneous emission are considered noise
photons they're not carrying information about our original signal
but as these photons are traveling down the fiber
they can travel either backwards which is not really a problem for us
but they can also travel forwards and then as they do that
they can become amplified via stimulated emission
remember there's a lot of erbium atoms sitting around in the excited state
just waiting for some other photon to come in and cause stimulated emission
so what's happening in the edfa is signal amplification where the signal photon
so the photon that we want to amplify comes in

00:07
stimulates emission from an excited erbium atom
and in the process becomes amplified then we get
two signal photons which is good which is what we want
but as we said some of the erbium atoms decay spontaneously
so this is how noise amplification occurs
we have an excited erbium atom it decays spontaneously into its ground state
giving out one noise photon via spontaneous emission and that photon can
then be amplified via stimulated emission from other excited erbium atoms
producing two noise photons so what's very important
is to consider the ratio of the amplified signal to the amplified noise
so this is the basic principle of a classical repeater now we will see how
the sources of noise affect the propagation of singles signals over long

00:08
distances in the quantum case and mainly we will see that the set of
challenges that are facing us there are completely different

\section{Quantum challenges}


00:00
step 5 quantum challenges so in quantum communication the signal
is at a level of individual photons which is very different to
classical communication where we are sending many many photons
so the major problem becomes photon loss in the fiber
classically if we are communicating and we lose a single photon
this is a very insignificant the whole signal propagates and
still be we can still read it out at the output
however when this happens in a quantum communication protocol
and we lose a single photon then the entire protocol
fails go back to for example e91 or bb84 when a loss of a single pro
photon becomes a huge problem so how can we combat photon loss we saw
that it's possible classically let's think whether this is also

00:01
possible in the quantum communication so amplification of a signal worked in
classical communication can we do that for quantum signals as well
namely can we create backup backup copies of the single photons such that
if one becomes one gets lost we can still use the
backups in order to proceed with the protocol
and we can but we are limiting ourselves to orthogonal states
if the states that we are sending are for example just cubits
0 and 1 0 and 1 and so on fine we can do that and we can create backup copies
but in quantum communication all the magic happens with
non-orthogonal states and often these states are entangled with some other
qubits somewhere else therefore we have to be able to copy
arbitrary states and this is where we hit the roadblock that we have seen in
previous lessons which is the no cloning theorem

00:02
okay so amplification will not work in quantum communication
how about just sending the photon and hoping for the best
let's do a very quick calculation that will demonstrate that this is another
very good strategy so the probability that we transmit the photon
through a fiber over distance uh capital
l where the attenuation parameter in the fiber is alpha
is given by the following expression it's 10 raised to the
power of negative alpha times l divided by 10.
so now let's plug in some numbers to give us some intuition of what this
probability is so we will consider a long fiber of thousand kilometers
which we call long but in the context of global communication is not that long
actually and we will assume a best-case scenario where the
fiber has ultra low attenuation of mere 0.1 decibels per kilometer

00:03
so the probability of that we are actually successfully transmit
a single photon is given by 10 to the power of minus
10. so this looks like a very small number and indeed it is
but to give you some intuition of really how small
this number is let's consider that we have a source that
produces one photon every second so every second
we are sending a single photon down the fiber
how long do we need to wait in order for somebody that's at the end of this
fiber 1000 kilometers away to actually successfully receive a single photon
well we have to wait a pro on average 317 years
so you can see that even with ultra low fibers
over moderate distances sending a single photon down the fiber
is not a very good idea and this is only one source of error

00:04
that we have to contend with in long distance quantum communication
other sources of errors include unitary errors such as pauli
errors where we can randomly flip the state of our photons
or z errors where we introduce some phase to the photons
and then there is a whole bunch of non-unitary errors such as
the coherence dephasing relaxation most of these errors we don't have to
deal with in classical communication so the situation looks quite dire
is there any hope for long-distance quantum computation
and we will see that quantum problems require quantum solutions
so join me in the next lesson where we will talk about how we can overcome
these challenges


\newpage
\begin{exercises}
\exer{Consider the following quantum state:}
\begin{equation*}
\ket{\psi} = \frac{\sqrt{3}}{2}\ket{0} + \frac{1}{2}\ket{1}
\end{equation*}
\subexer{Find the probability of measuring a zero.}
\subexer{Find the probability of measuring a one.}


\end{exercises}

