
\chapter{Physical Layer Components}

\section{Introduction}

00:00
hi and welcome to lesson 13 on physical layer components
in lesson 12 you saw the basics of a quantum repeater and how it can
distribute entanglement over long distances in here we will go a
little bit deeper and see how each individual physical components of a
quantum repeater work we will begin with step one the introduction
so here we have a basic scheme for a quantum repeater we've got our
network nodes represented by these blue boxes over here and they have some
qubits and then the network nodes they fire photons towards this middle
note here which we know it implements a bell state measurement
and that way we can establish entanglement over long distances but what are the
individual physical components of a quantum repeater and more

00:01
importantly how can we implement them so first we need optical fibers to carry
our photons that's kind of obvious and already we have covered
optical fibers in some length the next thing that we need is the bell
state analyzer in the middle the analyzer implements our bell state measurement
it is crucial for implementing entanglement swapping
but while the photons are in flight and traveling towards our battle state
analyzer these qubits they are stored in the nodes
they are stationary they are not moving anywhere so we must have some physical
means of storing these qubits and we do that with the help of
quantum memories now the difference between classical memories and quantum
memories is quite substantial classical memories
are usually stored as zeros and ones only classical bits
whereas quantum memories have to be able to store one

00:02
zero but also any superposition and in fact any entangled state and
that way they can share entanglement over the entire quantum network then so
after talking about these three basic well two basic
components because we covered fibers already we will consider how all of
these components work together to implement a quantum repeater and
most importantly we will also consider what are the various factors that affect
the success rate of our quantum repeater and entanglement swapping scheme
so after this introduction we will spend the first half of this lesson talking
about battle state analyzer we will revisit measurements
and how they can actually be implemented and what does it mean
really to measure in different bases and how can we implement
different basis measurements with polyzat measurements and some unitaries
then we will move on to the quantum circuit for a bell state measurement

00:03
and from that we will talk about implementation of bell state
measurements with linear optics so we will consider how barrel state
measurements are done in real life and then in the latter half of this lesson
we will talk about memories we will begin with considerations about
what a good quantum memory should be like what are the requirements
then we'll move on to the candidate systems because at the moment there is no
leading physical system that is considered to be
the best quantum memory all of the existing candidate systems have some
advantages and some drawbacks which we will consider in the latter half of this
lesson so let's begin

\section{Bell State Measurements I}


00:00
step two battle state measurements one so let's remind ourselves what those
states look like sobel states are four entangled states
and they can be written in this form in the computational basis
5 plus is a superposition of 0 0 and 1 1. so is 5 minus but
with an a negative phase between the two and this is the expression for psi plus
and psi minus we saw that these states are orthonormal what that means
is that if you take the inner product of a
bell state with itself for example five plus then
you get a one it means it is normalized but if you take an inner product of
uh one of these states with some other state then you get a zero for example
the inner product between five minus and five plus is zero and so on
what this means is that we can take any state

00:01
any pure state of two qubits and write it out in the bell
basis for example let's consider a general pure two-qubit state in the
computational basis with probability amplitudes given by alpha beta
gamma and delta we saw in previous lessons how to rewrite this state
in the bell bases so now it's a superposition of
all the four bell states where of course these probability amplitudes have
changed for example the probability amplitude for
state 5 plus is alpha plus beta and so on for the other bell states
and we have been treating measurements as a question that we ask about the state
of our physical system and the question in this case
is which of the four bell states is our state in
is it the state phi plus is it the state 5 minus
psi plus or psi minus this is what the measurement reveals

00:02
about our system and usually we get we say that we get the answer
with some probability for example depending on the initial state
psi we might get the answer that the state
is 5 plus in this case it would be with probability
which is the modulus squared of this probability amplitude
this is a very abstract notion of what a measurement actually is and it's not
very useful when it comes to doing calculations
so let's see how to actually implement uh such a
such a measurement in terms of a quantum circuit that will tell us more about
what these measurements are actually doing and how we can implement them in a
laboratory so but before we do that let's step back a little bit and consider
something simpler let's look at measurement of a single qubit
and let's in let's be particular and say
that we want to measure the single qubit in a polyz basis
in the quantum circuit notation we would write it as this we've got some input

00:03
state psi and then we measure it this is the symbol for measurement and
here this little z is just reminding us that we are measuring in the z
basis so what we get is that initially if the state is some general
superposition of zero and one with probability amplitudes alpha and beta
then the measurement can give us a plus 1 outcome with corresponding probability
which is given by the inner product between the initial state
and our basis state zero modulus squared which is just
alpha modulus squared or we can get the minus one outcome
which is given by the probability given by beta modulus squared
the probability amplitude of the basis state 1.
now let's try to do a measurement in the x-bases
again in the quantum circuit notation it would look like this we've got our

00:04
initial state psi and we're doing a measurement and now it's in the x basis
which is over here written by this little x
again we are considering some general input state but in order to
be able to determine the probabilities of the different outcomes
we are going to rewrite this state in the x basis
so we have seen that zero is equal superposition
of the plus state and the minus state and similarly one is an equals
equal superposition as well but this time we have a negative phase between
plus and minus states so we can rewrite the initial state in this following form
and from this form we can easily read out the probabilities
of obtaining the plus one outcome of the measurement
which is given by half alpha plus beta the whole thing modulus squared and the
probability for the minus one outcome but now let's impose a restriction of

00:05
our on ourselves let's say that we cannot perform
measurement in the x basis let's say we're only allowed to do measurements in
the z basis what do we do then well we can consider the following quantum
circuit we've got the initial state and then we do some unitary operation on
that state here we are choosing the hadamard
and then we perform our measurement but this time in the z basis
so we start with the usual initial state it's a superposition of zero and one
but then after application of the hadamard gate
we get a new state which we are denoting psi plus
and this new state is given in this form just to remind you hadamard when it's
applied on zero gives us a plus state so an equal equal
superposition of zero and one whereas hadamard applied to one gives us the minus
state and then if we just expand the plus state and the minus state again in

00:06
the computational basis we arrive at this form for our state psi plus
and now because we are doing measurement in the z basis again it's easy to read
out the probabilities corresponding to outcomes plus one and minus one but look
even though here we have a different quantum circuit
we are obtaining the outcomes plus 1 and minus 1
with the same probabilities as we had done on the previous slide
so what this shows is that there are two ways of implementing a pauli x
measurement we can do it directly writing it like this
or if we want we can just measure in the z basis
but then we have to apply some unitary transformation on our pure state side
so why is that well the clue is in the probabilities if you look at
the probability of the plus one outcome in in this scheme where we are

00:07
measuring uh the x basis directly then it's given by
the inner product of psi and the plus state modulus squared
if we do it in our other scheme where we are using the hadamard followed by a z
measurement then it's given by the corresponding expression here
now it's the inner product between the state psi prime
so this is our initial state after we apply hadamard to it
and the inner product is with a plus state and then what we get is
psi times hadamard times zero so what we are getting is really if we
are comparing these two probabilities we see that this psi state in a
with a has an inner product with a plus and plus it's just written as
hadamard applied to a zero state so that's what we are doing here
so what we are really uh doing is we are asking

00:08
what unitary takes me from a plus state to a zero state or from a zero state to
a plus state and we already know this unitary it's
the hadamard similarly for the minus one outcome
we look at the different probabilities and what we get is again
we go from a minus state to a one state via the hadamard transformation
so the halamart transforms our desired basis of measurement
which in this case is a polyx into a powly z basis
now how to do a bell measurement using only pauli zeds
well we have two qubits so we're going to need two poly measurements
and we are going to need to apply some two-qubit unitary before we
actually measure the state so the question now is how do we find this unitary
well we know one thing that the unitary should when applied to a state

00:09
psi plus should give us a 0 0 when applied to state 5 minus it should give
us state 1 0 and so on and so forth so this is our rule of transforming from our
bell bases into our computational basis and if we can do that then we can just
measure in the z uh basis in the same way we saw how it
worked for the pauli x basis we had to find a unitary that transforms from the
x spaces into the z basis which was given by hadamard
here we are looking for a more general unitary because it's acting on two qubits
so without further ado this is the circuit that actually achieves
our desired transformation it's given by a c note gate
which is given by this two qubit gate followed by a hadamard on the first qubit
and then two measurements in the z basis and what happens that if we get uh
outcomes zero zero which rep which corresponds to plus one plus one

00:10
then we know that we have a bell state five plus and similarly for all the
other three possible measurement outcomes and each individual
measurement outcome represents uniquely a bell state measurement
so this is true in general that any measurement
basis can be implemented by polyz measurements and a suitable unitary
applied before we measure our qubits and this trick is very useful
particularly in quantum computation and also in quantum communication and in
the next step we will actually see how we can do this in real life using
linear optics

\section{Bell State Measurements II}

00:00
step three barrel state measurements two in this step we will talk about how to
actually implement a bell state measurement with linear optics
so the actual scheme of how to do it really depends on our encoding
so let's just pick some encoding for our
qubits and let's choose linear polarized single photons
there's two reasons behind that one is that it's one of the most nearly used
encodings in real experiments and two it's very easy to imagine what's
going on therefore it's it is intuitive and in this accounting we encode our
qubit in a state 0 as a single photon that's polarized in the vertical direction
and our other computational basis state one as a single photon polarized in the
horizontal direction so the first question that we should ask

00:01
is how do we implement a pauli set measurement with this encoding
well we have to measure and distinguish the two different
polarizations we have to distinguish horizontal and vertical that can be done
with a little piece of crystal called polarizing beam splitter what happens
is that if you have some photons or some light coming in
with some polarization it splits the beam of light into two beams
one that is transmitted through the polarizing beam splitter
but comes out at the other end only with a vertical polarization and the
other one that gets reflected by the polarizing beam splitter
and that's that one is polarized in the horizontal direction
so we see that we are splitting our beam of light
into two beams one for r0 and one for r1

00:02
so consider that we have a single photon polarized in the vertical direction
and we place two detectors after the polarizing beam splitter to detect in which
path was which path was taken by the single photon
so what happens as we saw before if the photon is vertically polarized
it gets transmitted through the polarizing beam splitter
therefore it gets detected by this detector here
with probability one this represents our measurement
for plus one it never happens that a vertically polarized photon gets
reflected therefore this detector never gets triggered so
the probability of the outcome minus one is always zero on the other hand
if our initial photon is horizontally polarized
then you might guess it always gets reflected

00:03
and travels down into this detector here triggers it and always gets detected so
the probability of the outcome minus one is one and the probability of
the outcome plus one is always zero now can you guess what happens if we
actually put in a superposition of these two linear polarizations our state
is given by alpha v plus beta h what happens is that the photon has a chance
to get transmitted and this probability is given by
mod alpha squared and it also has a chance to get reflected
and travel down into this other beam splitter corresponding to the
measurement outcome minus one and the probability of getting a minus one
which means probability of triggering this detector is given by mod beta squared
so this implements our z measurement but we have seen in the previous step that
for a ball state measurement we need two measurements in the z basis

00:04
and a suitable unitary so let's let's see what happens when we actually take
two polarizing beam splitter we beam splitters each with two detectors
and we have just a regular 50 50 beam splitter over here and we've
got two photons coming from the these parts
so let's to be able to analyze this we actually
have not developed the proper tools for that we need
a little bit more of quantum optics but we can give you the result so
depending on which of these detectors trigger we will know
which bell state has been measured so let's see what the different patterns
are and to not get confused we will label our detectors as detector d1

00:05
d2 d3 and d4 so if we get a joint detection in detectors d1 and d4
this means that we have implemented a successful build measurement
and the outcome corresponds to the projection onto the state psi minus also
if we get a joint detection in these two detectors
in d2 and d3 joint detection means that both of these detectors trigger
then we can also say that we have implemented a successful measurement
and the result corresponds to the state psi
minus a different pattern of detection that we can get
is a joint detection at these two detectors in the lower branch of our
belt state analyzer so if both d1 and d2 trigger
then we can conclude that we have a state psi plus so this corresponds to
another successful bell measurement equally

00:06
if we get detectors triggering in these in the right branch of our ball state
analyzer so if both d3 and d4 trigger then we can also conclude
they've got we've got a bale state psi plus what can also happen is that both
photons travel into a single detector for example a d1 or they might travel
into d2 or d3 or d4 all of these are equally probable
however then we cannot say that whether we have
5 plus or a five minus so whenever we get only one of these detectors triggering
we know that we have one of the state psi plus or psi minus but we cannot tell
which one and this is a bit of a problem because we cannot
fully implement a battle state measurement we cannot distinguish
all four bell states we can only distinguish two of them psi plus and psi minus

00:07
so this means that a better state measurement cannot always be successfully
implemented with linear optics actually the maximum
probability is limited to only 50 percent
so we we have seen in a previous lesson that quantum repeaters have to contend
with a lot of noise we have to deal with that noise we have to purify our states
to counteract the effects of this noise we have also seen in previous lessons
that we have to deal with uh lossy fibers which was the reason why we
started talking about quantum repeaters but even if we take away all of these
sources of error there is still a fundamental error
in the fact that bell state measurement cannot be
always successfully executed with linear optics
on top of that what we have to take into account is that the two photons
coming into our bell state analyzer have to be

00:08
synchronized they have to arrive at the same time
if they don't arrive at the same time we fail our bell state measurement
and we cannot establish entanglement between the network nodes

\section{Stationary and flying qubits}


00:00
step four stationary and flying qubits first let's begin talking about
what is important what does a good memory look like and what requirements should
it satisfy and we will start with the divincenzo criteria
these criterias were introduced in the context of quantum computation
but we will see that a lot of them apply also in the context of quantum
networking so first if you want to build a good quantum computer
you need a well-defined qubit now qubits
don't come for free in nature usually we have very complicated systems with many
different energy levels and in order to have a
good well-defined qubit you must be able to take a system
for which you can only address two levels
and distinguish them and control them in a very

00:01
good way then you need to be able to initialize this qubit initialization is
important because then you know uh exactly from what state your
quantum computation can start so if you have a good procedure for initializing
your qubit that allows you to also carry out good quantum computation
also you want long lifetimes meaning that your qubits when you put them in a
superposition of states they don't decohere very quickly long
lifetimes allow you to carry out longer and longer quantum computations
which is of course needed if you want to solve harder and harder problems
also you must be able to implement a universal set of gates
meaning that what your qubit and your physical system need to do
is be able to implement a finite set of gates which when put
together in some order can allow you to simulate a much more

00:02
complicated evolution and or you need efficient measurements uh
just carrying out and transforming the state in a quantum manner is not enough
you somehow have to extract the information at the end of the quantum
computation and in the context of quantum communication there's few more
requirements that we have to consider we already saw that it's we have to
somehow convert or entangle stationary and flying qubits stationary
quads are those qubits that are sitting in our
quantum network nodes they're loaded into the quantum memories
they don't really move which is why we call them stationary
flying qubits are those qubits that are used for
entanglement swapping in the bsas to create link level entanglement between
between the quantum memories and we must be able to
entangle photons i.e the flying qubits with the stationary qubits inside the
memories but also we must be able to use entanglement swapping to create

00:03
end-to-end entanglement so perform entanglement swapping on
stationary memories themselves and then we also must be able to
transport flying qubits over long distances
so here we will in this lesson sorry in this step we will uh
look at these three requirements so why is memory lifetime important well
we said that in computers uh if our memory lifetime is long and also our
gate speeds are fast we we are able to implement
implement uh longer computations we can implement more steps of a computation
in the context of quantum communication what's really important is not the gate
speed itself but the number of gates that we can apply
per round trip time or per rt uh let's consider how we establish link

00:04
level entanglement we start with quantum memories and they
emit photons these photons are entangled with the memories and they travel
let's say to a bsa analyzer which is found halfway between
the quantum nodes there we perform a bell state measurement
but then we also have to uh communicate classically
back to the notes about the outcome of the bell state measurements
and this is our round trip time so if our if our lifetime
of the memory is shorter than that then we cannot really do much because even if
we can perform the bell state measurements on those
photon pairs by the time this happens our memories
decohere and are not useful anymore so just to give you some idea of the
numbers that we're talking about the speed of light
in a fiber is approximately 200 millimeters per nanosecond
so if our nodes are separated one kilometer away

00:05
one round trip from one node to the other and back that takes
10 microseconds for 100 kilometers it increases to
one millisecond and for 10 000 kilometers it goes all the way up to
100 milliseconds per round trip time now what are the processes that are
degrading our memories the two main processes are energy relaxation
and dephasing and they are characterized by two different
time scales they are referred to as t1 time scale and t2 time scale
t1 characterizes the energy relaxation time
whereas t2 gives us the characteristic dephasing time
so first let's consider the energy relaxation time given by the time t1
this basically tells us how likely or how quickly does
our qubit decay from the excited state or from state one into a state zero

00:06
and the probability that we if we initialize our state in the state one
the probability that after some time small t we still find it
in the state one is given by this expression it's e to the power of negative t
over capital t one so the energy relaxation time
so the probability that after t1 seconds we find our state in uh
in the in the state 1 is given by 1 over e and
this process of going from 1 to zero captures the
fact that usually zero is um encoded into a
state of an of an atom for example that has a higher energy that's why we call
it the energy relaxation time now for the dephasing time this gives us
a time scale where um we we lose phase coherence in our qubit remember if

00:07
we are only using zeros and ones so basically we're using qubits
but implementing only classical communication t1 is important but t2 not so much
because there's no coherence there we are not using superpositions
but in quantum networking and quantum communication um superpositions uh are
crucial and those superpositions can be destroyed by this defacing
defacing process so if we start in a in an equal superposition of zero and
one so we are starting in the plus state the t1 is the characteristic time scale
that tells us uh when we will end in a completely
mixed state so completely mixed state is given over here it's a sum it's a
mixture of these outer products of 0 0 and 1 1 divided by 2.
and we saw in one of the earlier lessons the crucial difference between complete

00:08
mixtures and equal superpositions so here after some time t if we prepare
the state in the pure state sub after some time t we will have the following
mixed state where with probability p we will still be
in the ideal initial state and with probability one minus p
we will have the d cohered into a completely mixed state
and this probability is now given by the following expression
of e to the negative uh t over capital t two now both of these processes the
relaxation process and the dephasing process
are poisson process processes that's a little bit ironic since we're talking
about memories but these processes are memoryless dk processes
so we talked about the lifetimes of memories and why they are important
and we gave you some characteristic time scales
which are very important when you are talking about communication over
longer distances now let's address the question of

00:09
how do we actually entangle atoms and photons so
how where is our qubit zero and one where in our quantum memory so our
quantum memory is a two level system for now uh and it has a ground state g
and some excited state of higher energy which we can label e
so these are natural candidates for representing 0 and 1.
for example here this is our two level atom
this is the state g this is the excited state e
and in this particular case we prepared the uh the memory in the excited state
therefore it is in the state 1. now how do we represent
how do we represent the flying qubits where are the flying qubits
well there's different ways of encoding the information
into flying qubits and one possibility is that if we send a photon down a fiber

00:10
so there is a photon we can say that this is r1
so if we detect this photon we know that we send a one
however we can if we don't send the photon so there's nothing
that can encode our zero and here we can see that if the atom
decays from the excited state into its ground state
or if it makes a transition from one to zero
represented over here that can emit a photon now how about coherences how about
superpositions well we can prepare our memory in a superposition of the ground
state and the excited state so it's an equal superposition of
zero and one by applying an appropriately timed energy pulse and our question is
well what happens to the photon does it can get emitted does it not get emitted
and if it does get emit if it does get emitted in what state will it be

00:11
well here we see that the atom has an equal probability to be found in
the ground state so if it's in the ground state then it cannot emit any
any energy so our photon will be in the zero state
there is no photon or it has an uh fifty percent probability
to be in the excited state from where it can emit and when it does emit
then we will have a photon over here so in this way
we can think about the photon of being in a super position of zero
and one there is a photon and there is not a photon
so we are transferring the state plus from the atomic memory to the flying
qubit this is a very naive naive picture that demonstrates
only some of the basic uh principles of how stationary and flying qubits
interact together in real systems things are a lot more complicated

00:12
in particular when we look at this encoding of just having two levels for our
quantum memory then this is usable but it's not a very good qubit
because due to the energy relaxation process our
excited state will eventually decay into a zero so it will destroy
whatever message of whatever state we have encoded into the quantum memory
similarly this encoding for the flying qubits of having a no photon and a
photon representing our zero and one is not very good due to the attenuation
of light in fibers we described it in some detail that as
we send photons down fibers they're very likely to be lost and attenuated
so if we are waiting for some message at the end of the fiber and we don't
receive a photon we cannot be sure was the original message really zero
so uh is it correct that we are finding no photon or was the
initial message one and the photon just got lost along the way

00:13
so we have to be a little bit more careful and think how to
encode our information in a better more robust way
consider the following following atomic structure
we've got two degenerate ground states and we will label them as ground state up
and ground state down and these can represent the two spins of of our atom
for our flying qubits we can consider polarization
so zero will be represented by vertical polarization
and one will be represented by horizontal polarization
we can prepare our atom initially in the excited state
and then what can happen is that the atom can decay either
to the ground state would spin up or to the ground state would spin down
the thing is we can only see that there's a photon coming out

00:14
and we don't actually know into which ground state the atom decayed so
we are effectively implementing the following following transformation
we go from the excited state of the atom to a superposition of the atom being
found in the spin up state if that's true then the photon that gets emitted
just happens to have a vertical polarization
on the other hand if it decays into the other
ground state given by spin down then the photon will have a horizontal
polarization so really what we are doing is we are obtaining the following
superposition of two qubits we have an equal superposition of the
atom being in the spin up state and the emitted photon being in the
vertically polarized and the other term which is the atom being
found in the spin down state and the photon being horizontally polarized so

00:15
in this way we are entangling the flying photon with the stationary qubit of the
memory and to bring this all back this is another representation of our
bell state analyzer which we have before drawn very
abstractly but now you have a much better idea
how it actually works in practice so here we have two single mode fibers
and at the end we've got quantum memories each memory is prepared in
initially in the excited state it decays into one of its um ground states either
spin up or spin down we don't know which one therefore the flying photon is
entangled with its respective memory these flying photons now that
travel through the single mode fibers they hit the beam splitter
they interfere and we perform a bell state measurement
and in this way we can establish uh link level entanglement

00:16
between the atomic memories sitting at the end of the ends of the link
so and this is exactly the scheme that we have described previously
with memory interference memory mim or it can also be used in the following
scenario where we have direct memory to memory connection you


\newpage
\begin{exercises}
\exer{Consider the following quantum state:}
\begin{equation*}
\ket{\psi} = \frac{\sqrt{3}}{2}\ket{0} + \frac{1}{2}\ket{1}
\end{equation*}
\subexer{Find the probability of measuring a zero.}
\subexer{Find the probability of measuring a one.}


\end{exercises}

\newpage
\section{Quiz}

%\section{Learning more}

\section{Further reading Lessons 11-13}

Lesson 11

For those of you interested in how submarine fiber optic cables are made, laid and operated we recommend the following online article found here.

Our discussion of mode dispersion closely followed Section 5.6 of Hecht’s textbook and we encourage you to read it for the extra details that can be found in the book.

Qualitative review of classical amplifiers (with just the right amount technical detail) can be here:

Emmanuel Desurvire, The Golden Age of Optical Fiber Amplifiers, Physics Today 47, 20 (1994).

Unfortunately this article is behind a paywall so you will have to use your university’s online system to access it.

Lesson 12

Quantum repeaters are the “bread and butter” of quantum networks. A great place to learn more is here:

Rodney van Meter, Quantum Networking, Wiley-ISTE, 2014.

Those of you interested in the paper that introduced the idea of a quantum repeater (and are not scared off by maths) might have a look here:

Hans J. Briegel, Wolfgang Dür, Juan I. Cirac, Peter Zoller, Quantum repeaters: The role of imperfect local operations in quantum communication, Physical Review Letters 81, 5932, 1998.

The paper is behind a paywall and needs to be accessed through your university’s library online services. An earlier version of the paper can be accessed openly here.

Lesson 13

Great popular article about the physical layer components of quantum networks can be found here:
Dan Hurley, The quantum internet will blow your mind. Here’s what it will look like, Discover Magazine, 2020.

Another fantastic review of physical layer components can be found here:

Nicolas Sangouard, Christoph Simon, Hugues de Riedmatten, Nicolas Gisin, Quantum repeaters based on atomic ensembles and linear optics, Review of Modern Physics 83, 33, 2011.

Again, the published version is behind a paywall but the pre-publication version can be accessed for free here. Be warned though, this paper starts with an excellent introduction but the technical details ramps up quickly after that and relies on good grasp of quantum optics. So if you get lost after the introduction, don’t worry. You can come back to those parts later.
