\chapter{Teleportation}
\label{sec:8_teleportation}

Teleportation is one of the most wonderful and most fundamental protocols in quantum information processing.
It is used extensively in quantum computation and quantum communication.

%%%%%%%%%%%%%%%%%%%%%%%%%%%%
\section{Introduction}
\label{sec:8-1_introduction}
%%%%%%%%%%%%%%%%%%%%%%%%%%%%

This Chapter deals with the question of transmitting quantum information in a quantum network.
Before we deal with this question, let's see how we transmit classical information.
In Ch.~\ref{sec:1_Introduction}, we discussed the example of the Great Wall of China.
If a guard tower was under attack, the guards lit a fire to alert the other guard towers and ask for help.
The message, in the form or light, was generated at the place of attack and travelled to the neighboring guard towers.
Nowadays, if we want to download data from a server, we encode the request in the form of a light signal that is carried by optical fibers.
The physical systems that encode the information, in this case photons, are transmitted from from us to the server hop-by-hop using intermediate nodes of the classical network.
\begin{figure}[H]
    \centering
    \includegraphics[width=0.5\textwidth]{lesson8/quantum-network.png}
    \caption[A quantum network of nodes and channels.]{A quantum network consists of quantum nodes, and quantum channels that we will call links.}
    \label{fig:quantum-network}
\end{figure}
\begin{figure}[H]
    \centering
    \includegraphics[width=0.5\textwidth]{lesson8/hop-by-hop.png}
        \caption[Hop-by-hop transmission.]{Physically moving a quantum message hop-by-hop through a network.}
    \label{fig:hop-by-hop}
\end{figure}
\begin{figure}[H]
    \centering
    \includegraphics[width=0.5\textwidth]{lesson8/link-down.png}
        \caption{Rerouting when a link goes down.}
    \label{fig:link-down}
\end{figure}
\begin{figure}[H]
    \centering
    \includegraphics[width=0.5\textwidth]{lesson8/partitioned.png}
        \caption{A partitioned network.}
    \label{fig:partition}
\end{figure}

However, quantum networks offer different ways of transmitting information.
Consider a small quantum network in Fig.~\ref{fig:quantum-network}.
The circles represent our nodes and the lines represent the links, or quantum channels, between the nodes.
The sender, represented by the blue node, is in possession of a pure state $\ket{\psi} = \alpha\ket{0}+\beta\ket{1}$, and  wants to send this state to the target given by the orange network node.
The transmission can be done hop-by-hop by encoding it in a physical system and sending that physical system along a path that connects to the target node.
The network may decide to send the encoded quantum message via the path represented by the arrows in Fig.~\ref{fig:hop-by-hop}.
What happens if one of the quantum links is down, as in Fig.~\ref{fig:link-down}?
In this particular case, it's not a big problem because the message can be simply rerouted to go around the damaged link and still reach the target node.
But what happens if yet another link is down, as in Fig.~\ref{fig:partition}?
In this case, the quantum portion of our network is \textbf{\emph{partitioned}}.
It may seem that we cannot transmit the message to the target...unless we use \textbf{\emph{teleportation}}\index{teleportation}\footnote{Here, we assume that the supporting classical network remains intact, only the quantum channels are going up or down.}
We can teleport this state from the sender to the target, provided that some initial conditions are met.
We can move the information itself \textbf{\emph{without moving the actual physical system}}.
That physical system remains stationary.
% \begin{figure}[H]
%     \centering
%     \includegraphics[width=0.8\textwidth]{lesson8/teleportation-setup.png}
%         \caption[The setup for teleportation.]{The setup for teleportation: Alice has the data qubit to be teleported, and Alice and Bob share a Bell pair.}
%     \label{fig:teleportation-setup}
% \end{figure}

\begin{figure}[H]
    \centering
    \includegraphics[width=0.8\textwidth]{lesson8/teleportation.png}
        \caption{Outline of the teleportation protocol.}
    \label{fig:teleportation}
\end{figure}

This is the outline of the teloportation is the following, pictured in Fig.~\ref{fig:teleportation}.
\begin{enumerate}
    \item Let's call the sender ``Alice'', who wishes to communicate the quantum state $\ket{\psi}$ to her friend ``Bob''.
    \item Alice and Bob start by sharing an entangled pair of qubits.
    \item Alice performs a two-qubit measurement jointly on her qubits. That measurement provides her with two classical bits of information.
    Naturally, this means that there are four possible outcomes.
    \item Alice communicates the outcome of the measurement to Bob via a classical channel.
    \item Bob receives this classical message and applies appropriate local corrections.
    \item The state of Bob's qubit is the desired state $\ket{\psi}$.
\end{enumerate} 

Now you see the two conditions necessary to maintain our ability to teleport quantum data even when the quantum network is partitioned: we must have a supply of entanglement, and we must have the ability to exchange classical messages.



%%%%%%%%%%%%%%%%%%%%%%%%%%%%%%%%%%%%%%
\section{Teleportation protocol}
\label{sec:8-2_teleportation_protocol}
%%%%%%%%%%%%%%%%%%%%%%%%%%%%%%%%%%%%%%

In this Section, we will look at the teleportation protocol in more detail, and make the outline of the previous Section more mathematical.
First, consider the initial state in our protocol.
Alice has an arbitrary qubit given by
\begin{align}
    \ket{\psi} = \alpha\ket{0} + \beta\ket{1}.
\end{align}
Alice and Bob also share an entangled state.
It's one of the Bell states, $\ket{\Phi^+}$, which is an equal superposition of $\ket{00}$ and $\ket{11}$.
We label the qubits as follows: $A_1$ is Alice's first qubit, holding the state that she is trying to communicate to Bob. 
Her second qubit, $A_2$, is one half of the entangled Bell pair.\
We designate Bob's qubit, which is the other half of the entangled pair that he is sharing with Alice, as $B$.
If we write out the initial three-qubit state in its full form, we have
\begin{align}
    |\psi\rangle_{A_{1}}|\Phi^{+}\rangle_{A_{2} B} &=(\alpha|0\rangle+\beta|1\rangle)_{A_{1}} \frac{1}{\sqrt{2}}(|00\rangle+|11\rangle)_{A_{2} B} \nonumber\\
    &=\frac{1}{\sqrt{2}}(\alpha|000\rangle+\alpha|011\rangle+\beta|100\rangle+\beta|111\rangle)_{A_{1} A_{2} B}.
    \label{eq:teleportation_initial_state}
\end{align}

Alice then performs a two-qubit measurement in the Bell basis.
To refresh your memories, you may want to take a look at Sec.~\ref{sec:4_3-bell_states}, where we discussed Bell-state measurements.
Doing a measurement in the Bell basis is basically asking the question: which of the following states is Alice's state in? 
\begin{align}
\begin{array}{ll}
    \left|\Phi^{+}\right\rangle=\frac{1}{\sqrt{2}}(|00\rangle+|11\rangle), \quad & \left|\Phi^{-}\right\rangle=\frac{1}{\sqrt{2}}(|00\rangle-|11\rangle), \\
    \left|\Psi^{+}\right\rangle=\frac{1}{\sqrt{2}}(|01\rangle+|10\rangle), \quad & \left|\Psi^{-}\right\rangle=\frac{1}{\sqrt{2}}(|01\rangle-|10\rangle).
\end{array}
\end{align}
To answer that question, we begin by rewriting our initial state in the Bell basis.
We will use the following set of four identities relating computatinal basis to the Bell basis,
\begin{align}
\begin{array}{ll}
    |00\rangle=\frac{1}{\sqrt{2}}\left(\left|\Phi^{+}\right\rangle+\left|\Phi^{-}\right\rangle\right), & |01\rangle=\frac{1}{\sqrt{2}}\left(\left|\Psi^{+}\right\rangle+\left|\Psi^{-}\right\rangle\right), \\
    |10\rangle=\frac{1}{\sqrt{2}}\left(\left|\Psi^{+}\right\rangle-\left|\Psi^{-}\right\rangle\right), & |11\rangle=\frac{1}{\sqrt{2}}\left(\left|\Phi^{+}\right\rangle-\left|\Phi^{-}\right\rangle\right).
\end{array}
\end{align}

Going term by term of the initial state in Eq.~(\ref{eq:teleportation_initial_state}), we first look at the portion of Alice's state that is in $\ket{00}$.
We substitute for it our superposition of Bell states $\ket{\Phi^+}$ and $\ket{\Phi^-}$.
We do the same thing for the next term in the initial state $\ket{01}$, and we substitute for it the superposition of $\Psi^+$ and $\Psi^-$, and so on and so forth.
In full, we get the following superposition,
\begin{align}
    \begin{aligned}
        |\psi\rangle_{A_{1}}|\Phi^{+}\rangle_{A_{2} B} & = (\alpha|0\rangle+\beta|1\rangle)_{A_{1}} \frac{1}{\sqrt{2}}(|00\rangle+|11\rangle)_{A_{2} B} \\
        = & \frac{1}{\sqrt{2}}(\alpha|000\rangle+\alpha|011\rangle+\beta|100\rangle+\beta|111\rangle)_{A_{1} A_{2} B} \\
        = & \frac{1}{2}\left( \alpha\left(|\Phi^{+}\rangle+|\Phi^{-}\rangle\right)|0\rangle\right.\\
        & + \alpha\left(|\Psi^{+}\rangle+|\Psi^{-}\rangle\right)|1\rangle \\
        & + \beta\left(|\Psi^{+}\rangle-|\Psi^{-}\rangle\right)|0\rangle \\
        & \left. +\beta\left(|\Phi^{+}\rangle-|\Phi^{-}\rangle\right)|1\rangle\right). \\
\end{aligned}
\end{align}
Let's collect all the terms that have the same Bell state for Alice's qubits $A_1$ and $A_2$.
For example, the first term has a $\ket{\Phi^+}$, and also the last term has a $\ket{\Phi^+}$, but they differ in their amplitudes and also in the state of Bob's qubit.
For the first term, we have probability amplitude $\alpha$ and Bob's qubit in the state $\ket{0}$.
For the last term, we have probability amplitude $\beta$ and Bob's qubit is in the state $\ket{1}$.
We collect them together, and we get the expression
\begin{align}
\begin{aligned}
    |\psi\rangle_{A_{1}}|\Phi^{+}\rangle_{A_{2} B} & = \frac{1}{2}\left|\Phi^{+}\right\rangle_{A_{1} A_{2}}(\alpha|0\rangle+\beta|1\rangle)_{B} \\
    & + \frac{1}{2}|\Phi^{-}\rangle_{A_{1} A_{2}}(\alpha|0\rangle-\beta|1\rangle)_{B} \\
    & + \frac{1}{2}|\Psi^{+}\rangle_{A_{1} A_{2}}(\alpha|1\rangle+\beta|0\rangle)_{B} \\
    & + \frac{1}{2}|\Psi^{-}\rangle_{A_{1} A_{2}}(\alpha|1\rangle-\beta|0\rangle)_{B}.
\end{aligned}
\label{eq:teleport-alice-basis}
\end{align}

Remember, we are not really doing anything yet. 
We are still dealing with the initial state, just rewriting it in a more convenient form.
We are now in a position to answer the question that we asked earlier: in which of the Bell states are Alice's two qubits?
Alice performs the two-qubit measurement, and obtains two classical bits as the measurement outcome.
Probabilities of the four possible outcomes are all equal,
\begin{equation}
    \operatorname{Prob}\left(|\Phi^{\pm}\rangle\right)=\operatorname{Prob}\left(|\Psi^{\pm}\rangle\right)=\frac{1}{4}.
\end{equation}
Notice that the probabilities of the four possible outcomes are also independent of the probability amplitudes $\alpha$ and $\beta$, even though Alice's state was in an arbitrary superposition.

The outcome of Alice's measurement determines the state of Bob's qubit.
If Alice measures $\ket{\Phi^+}$, then we know that Bob has the desired output state $\alpha|0\rangle+\beta|1\rangle$, which was the initial state that Alice was trying to communicate to him.
So in that case, we can say, ``Great! The teleportation has succeeded.''

How about the other measurement outcomes?
Remember we said that all of these outcomes are equally likely, so Alice might get $\ket{\Phi^-}$, $\ket{\Psi^+}$ or $\ket{\Psi^-}$ as the outcome of her Bell state measurement.
For each of these measurement outcomes, Bob's qubit is different from the desired state $\alpha\ket{0}+\beta\ket{1}$.
Summarizing,

\hspace{2cm} If Alice measures $\left|\Phi^{+}\right\rangle \rightarrow$ Bob has $\alpha|0\rangle+\beta|1\rangle$.

\hspace{2cm} If Alice measures $\left|\Phi^{-}\right\rangle \rightarrow$ Bob has $\alpha|0\rangle-\beta|1\rangle$.

\hspace{2cm} If Alice measures $\left|\Psi^{+}\right\rangle \rightarrow$ Bob has $\alpha|1\rangle+\beta|0\rangle$.

\hspace{2cm} If Alice measures $\left|\Psi^{-}\right\rangle \rightarrow$ Bob has $\alpha|1\rangle-\beta|0\rangle$.

Does this mean that teleportation succeeds only with probability of 0.25?
No.
Luckily, all these other states can be transformed into $\ket{\psi}$ by a suitable unitary.
For example, notice that if we apply a Pauli $Z$ on the state $\ket{\psi}$, then we obtain $\alpha\ket{0}-\beta\ket{1}$, which happens to correspond to Bob's state when Alice's measurement outcome is $\ket{\Phi^-}$.
Applying a Pauli $X$ transforms \ket{\psi} to $\alpha\ket{1}+\beta\ket{0}$, while applying the Pauli product $XZ$ produces $\alpha|1\rangle-\beta|0\rangle$.
Summarizing,
\begin{align}
\begin{aligned}
    |\psi\rangle &=\alpha|0\rangle+\beta|1\rangle \\
    Z|\psi\rangle &=\alpha|0\rangle-\beta|1\rangle \\
    X|\psi\rangle &=\alpha|1\rangle+\beta|0\rangle \\
    X Z|\psi\rangle &=\alpha|1\rangle-\beta|0\rangle.
    \label{eq:teleportation_possible_states}
\end{aligned}
\end{align}
Note that all of those operations are unitary.  This means the states can be \textbf{\emph{returned}} to the original, desired state $\ket{\psi}$ using their adjoint operations.

Alice needs to let Bob know which measurement outcome she obtained, and she does so by sending two classical bits.
After receiving the classical bits, Bob knows exactly which correction he has to apply.
If Alice measures $\ket{\Phi^+}$, he does nothing because he already has the state $\ket{\psi}$.
If Alice measures $\ket{\Phi^-}$, then Bob knows he has to apply a Pauli $Z$ to his qubit.
If she tells Bob that she obtained the measurement outcome $\ket{\Psi^+}$, he simply applies Pauli $X$, and if she obtains a $\ket{\Psi^-}$ then he just needs to apply $ZX$ to his qubit.  Summarizing,

\hspace{2cm} If Alice measures $\left|\Phi^{+}\right\rangle \rightarrow$ Bob applies $I$.

\hspace{2cm} If Alice measures $\left|\Phi^{-}\right\rangle \rightarrow$ Bob applies $Z$.

\hspace{2cm} If Alice measures $\left|\Psi^{+}\right\rangle \rightarrow$ Bob applies $X$.

\hspace{2cm} If Alice measures $\left|\Psi^{-}\right\rangle \rightarrow$ Bob applies $Z X$.

Teleportation demonstrates the interchangeability of resources in quantum communication and quantum computation.
We have exchanged one qubit of communication for one entangled pair and two classical bits.
What does this mean?
It means that we can send the information about state $\ket{\psi}$ from Alice to Bob in two different ways.
We can send it directly, like in Fig.~\ref{fig:hop-by-hop}, which is counted as one qubit of communication.
Or, as we outlined with the teleportation protocol, if Alice and Bob share an entangled pair, \textbf{\emph{and}} are allowed to communicate two classical bits, then they can achieve the same task.



%%%%%%%%%%%%%%%%%%%%%%%%%%%%%%%%%%%%%%%%%%%%%%%%%%%%%%%%%%%%%%%%
\section{No-cloning theorem and Faster-than-light communication}
\label{sec:8-3_no-cloning}
%%%%%%%%%%%%%%%%%%%%%%%%%%%%%%%%%%%%%%%%%%%%%%%%%%%%%%%%%%%%%%%%

Let's go back to the teleportation protocol and ask a few important questions.
Alice communicated her state to Bob.
\textbf{\emph{Did she clone the qubit while teleporting it?}}
After all, she started with the state $\ket{\psi}$ and she did not send the physical qubit to Bob.
Yet at the end of the protocol, we saw that Bob does have the qubit $\ket{\psi}$.
What was going on there?
Initially Bob's qubit was part of a maximally entangled state.
He had one part of the entangled state and Alice had the other.
At the end of the protocol, Alice's qubit $A_1$ was part of a maximally entangled state.
She performed a measurement in the Bell basis that projected her two qubits onto one of the four possible Bell states, all of which are maximally entangled.~\footnote{Implementations commonly \emph{destructively} measure the two qubits at Alice as part of this process, in which case a more accurate description is, "The two qubits \emph{were} in Bell state..." rather than, "The two qubits \emph{are} in Bell state..." The mathematical implications for Bob's qubit and the no-cloning theorem remain the same.}
Initially, the state of the qubit $A_1$ was $\ket{\psi}$, and the state of the qubit that Bob had in his possession, considered alone, was a maximally mixed state.
After completing the protocol, the state of Alice's qubit $A_1$ became the maximally mixed state, whereas Bob's qubit became the state $\ket{\psi}$.
There was no cloning or copying of the state even though there was no direct physical transmission of the state $\ket{\psi}$ from Alice to Bob.
This leads to another interesting and important question.
\textbf{\emph{Is cloning possible?}}
Let's have a look at this question.

We will only consider cloning of pure states.
Fig.~\ref{fig:cloner} shows what a hypothetical cloning device.
It takes two qubits as inputs.
The first input is an arbitrary state $\ket{\psi}$ that we wish to clone, and the second one is initialized in \ket{0}.
The cloning operation is represented by a  unitary operation $U$.
The output is given by two copies of the arbitrary state \ket{\psi}.
The questions that we would like to naswer are teh following.
Is such a transformation possible? If so, what does the unitary look like that can achieve such a transformation?
\begin{figure}[t]
    \centering
    \begin{tikzpicture}
    \begin{yquant*}
    qubit {$|\psi\rangle$} in;
    qubit {$|0\rangle$} q;

    [shape=yquant-rectangle, rounded corners=.25em, fill=blue!10, y radius=10pt]
    box {cloning\\device\\$U$} (in, q);

    output {$|\psi\rangle$} in;
    output {$|\psi\rangle$} q;
    \end{yquant*}
    \end{tikzpicture}
    \caption[A hypothetical cloning device.]{A hypothetical cloning device would have to create an unentangled copy of the input state $\ket{\psi}$ on the second qubit initialized in $\ket{0}$.}
    \label{fig:cloner}
\end{figure}

For example if we start in the state $\ket{0}\ket{0}$, after applying the cloning unitary $U$, we should obtain state $\ket{0}\ket{0}$. If we start in the state $\ket{+}\ket{0}$ and apply our cloning unitary, we should obtain \ket{+} for the first qubit and \ket{+} for the second qubit.

Let's consider some candidate unitaries for the cloning unitary $U$.
As a first guess, let's consider a CNOT gate,
\begin{equation}
    \operatorname{CNOT} =|0\rangle\langle 0|\otimes I+| 1\rangle\langle 1| \otimes X = \begin{pmatrix}
        1 & 0 & 0 & 0 \\
        0 & 1 & 0 & 0 \\
        0 & 0 & 0 & 1 \\
        0 & 0 & 1 & 0 \\
    \end{pmatrix}
\end{equation}
If the state of the first qubit is zero, then apply the identity to the second qubit, meaning ``don't do anything''.
If the state of the first qubit is one, then apply Pauli $X$ to the second qubit.
Let's consider some inputs and their results,
\begin{equation}
\begin{aligned}
    \operatorname{CNOT} |0\rangle|0\rangle &=|0\rangle|0\rangle, \\
    \operatorname{CNOT} |1\rangle|0\rangle &=|1\rangle|1\rangle, \\
    \operatorname{CNOT} |+\rangle|0\rangle &=\frac{1}{\sqrt{2}}(|00\rangle+|11\rangle)\neq|+\rangle|+\rangle.
    \label{eq:cloning_examples_CNOT}
\end{aligned}
\end{equation}
First, let's try a very simple input, $\ket{0}\ket{0}$, as in the first equation above.
Applying the CNOT, you can check for yourself that the output is also $\ket{0}\ket{0}$.
In this case, the cloning of the input state is trivial since we do not have to actually do anything.
The second case is more interesting.
We wish to clone the state \ket{1}, and so the initial state is \ket{1}\ket{0}.
Since the first input qubit is \ket{1}, the CNOT gate is applied, flipping the state of the second qubit to \ket{1}.
We went from input $\ket{1}\ket{0}$ to output $\ket{1}\ket{1}$, cloning the qubit.

How about if we consider a superposition of zero and one as our input? The input is $\ket{+}\ket{0}$, and the desired cloned output is $\ket{+}\ket{+}$.
However, applying the CNOT, we obtain an entangled state $(\ket{0}\ket{0}+\ket{1}\ket{1})/\sqrt{2}$, which is not the product state that we were aiming for. Cloning has failed in this case.

The two cases in Eq.~(\ref{eq:cloning_examples_CNOT}) that were cloned successfully, states \ket{0} and \ket{1}, were orthogonal.
On the other hand, the one case that failed, state \ket{+}, was a superposition of these two.
Let's keep this observation in mind, and consider the case of a general cloning unitary applied to a general input state.

We start by assuming that the cloning unitary can be applied successfully to two different states \ket{\psi} and \ket{\phi}.
In other words applying the cloning unitary $U$ to the input states \ket{\psi}\ket{0} and \ket{\phi}\ket{0} produces the following outputs,
\begin{align}
    U(\ket{\psi}\ket{0}) & = \ket{\psi}\ket{\psi}, \label{eq:cloning_first_eq}\\
    U(\ket{\phi}\ket{0}) & = \ket{\phi}\ket{\phi} \label{eq:cloning_second_eq}
\end{align}
We next take the inner product between the two equations.
This may seem like a strange things to do, but it is a standard trick that you will see on numerous occasions in the future.
We begin by taking the adjoint of Eq.~(\ref{eq:cloning_second_eq}),
\begin{equation}
    (\bra{\phi}\bra{0}) U^{\dagger} = \bra{\phi}\bra{\phi}. 
\end{equation}
Next, we multiply the left-hand side Eq.~(\ref{eq:cloning_first_eq}) by the left-hand side of Eq.~(\ref{eq:cloning_second_eq}), and similarly for the right-hand sides of both equations,
\begin{equation}
    (\bra{\phi}\bra{0}) \underbrace{U^{\dagger} \cdot U}_{=I} \ket{\psi}\ket{0} = (\bra{\phi}\bra{\phi}) \cdot (\ket{\psi}\ket{\psi}).
\end{equation}
The cloning operation is assumed to be unitary, meaning $U^{\dagger}U=I$.
Therefore we can write
\begin{equation}
    \braket{\phi}{\psi} \cdot \underbrace{\braket{0}{0}}_{=1} = \braket{\phi}{\psi} \cdot \braket{\phi}{\psi}, \longrightarrow
    \braket{\phi}{\psi} = \braket{\phi}{\psi}^2.
    \label{eq:cloning_scalars}
\end{equation}
The inner product \braket{\phi}{\psi} is complex scalar.
Only scalars that satisfy Eq.~(\ref{eq:cloning_scalars}) are
\begin{equation}
    \braket{\phi}{\psi} = 0, \quad \text{and } \quad \braket{\phi}{\psi} = 1.
\end{equation}
This tells us something interesting about the initial states \ket{\psi} and \ket{\phi}.
Either the states are orthogonal, meaning $\braket{\phi}{\psi}=0$, or the states are the same, meaning $\braket{\phi}{\psi}=1$.
Any other state, for example a superposition of \ket{\psi} and \ket{\phi}, will not satisfy Eq.~(\ref{eq:cloning_scalars}), and therefore will not be cloned successfully by the unitary $U$.

This proves that our cloning really works \textbf{\emph{only}} for states that can be distinguished with certainty -- there is no ambiguity or overlap in their superpositions.
(Remember, orthogonal states are always distinguishable deterministically.)
So we can say that we cannot clone an arbitrary state in quantum mechanics, which is in stark contrast to classical physics.
This is the essence of \textbf{\emph{no-cloning theorem}}\index{no-cloning theorem}.

The second interesting question that we would like to answer about teleportation is the following.
\textbf{\emph{Is teleportation instantaneous?}}
Does it allow us to communicate faster than the speed of light?
The short answer is no, otherwise teleportation would violate the special theory of relativity.
In order to understand why this is the case, we have to consider the timeline of events during the teleportation protocol.
This is pictured in Fig.~\ref{fig:teleportation-timeline}.
We start at time $t_0$, when we initialize our system.
At some later time $t_1$, Alice performs her measurement in the Bell basis on the two qubits that she holds.
At some later time $t_2$, Alice sends the outcome of the measurements to Bob.
He receives these outcomes at time $t_3$.

Let's go step by step and consider what the state of Bob's qubit at these different times.
At time $t_0$, we said that Bob's qubit is part of a maximally entangled pair, which means that the reduced density matrix of his qubit is a maximally mixed state,
\begin{equation}
    \left|\Phi^{+}\right\rangle_{A_2 B}=\frac{1}{\sqrt{2}}(|00\rangle+|11\rangle) \longrightarrow \rho_B=\frac{1}{2}\ketbra{0}{0} +\frac{1}{2}\ketbra{1}{1}.
\end{equation}
This means that if Bob measures his qubit in the Pauli $Z$ basis, he will get the outcome zero ($+1$ eigenvalue) with probability one half, or the outcome one ($-1$ eigenvalue) with probability one half. In fact, if he measures in any basis, he will get the outcome $+1$ or $-1$ with the same probability, so he's maximally unsure about his own state. That also means that he has no idea what state Alice is trying to send him.
\begin{figure}[t]
    \centering
    \includegraphics[width=0.8\textwidth]{lesson8/teleportation-timeline.png}
        \caption[Event timing in teleportation.]{The timeline of necessary events and messages in teleportation makes it clear that the speed of light is not violated.}
    \label{fig:teleportation-timeline}
\end{figure}

At time $t_1$, Alice performs her measurement.
Just before the measurement, the total state of all three qubits is given by the expression in Eq.~(\ref{eq:teleport-alice-basis}).
Bob does not know anything about Alice's state $A_1$.
The initial state written in the form of Eq.~(\ref{eq:teleport-alice-basis}) might suggest that he knows something about the state of his own qubit.
This is however not true, Bob's qubit is still in a maximally mixed state.

Let's look at what is known immediately after Alice measures in the Bell basis at time $t_1$.
Just after the measurement, Bob knows that with probability one quarter, Alice measured the outcome $\ket{\Phi^+}$, and with the same probability she could have obtained each of the other three  Bell states.
So he knows that with probability $1/4$, he has the state $\ket{\psi}$, or with probability $1/4$ he has some equivalent state given by these following expressions,

\hspace{2cm} $\operatorname{Prob}\left(\left|\Phi^{+}\right\rangle\right)=1 / 4, \quad$ Bob has $\alpha|0\rangle+\beta|1\rangle$

\hspace{2cm} $\operatorname{Prob}\left(\left|\Phi^{-}\right\rangle\right)=1 / 4, \quad$ Bob has $\alpha|0\rangle-\beta|1\rangle$

\hspace{2cm} $\operatorname{Prob}\left(\left|\Psi^{+}\right\rangle\right)=1 / 4, \quad$ Bob has $\alpha|1\rangle+\beta|0\rangle$

\hspace{2cm} $\operatorname{Prob}\left(\left|\Psi^{-}\right\rangle\right)=1 / 4, \quad$ Bob has $\alpha|1\rangle-\beta|0\rangle$.
This is a distribution which we can write in the density matrix formalism.
The calculation is a bit lengthy but stick with it.
Bob's qubit can be written as follows:
\begin{equation}
\begin{aligned}
    \rho_B &=\frac{1}{4}(\alpha|0\rangle+\beta|1\rangle)\left(\alpha^*\langle 0|+\beta^*\langle 1|\right) \\
    &+\frac{1}{4}(\alpha|0\rangle-\beta|1\rangle)\left(\alpha^*\langle 0|-\beta^*\langle 1|\right) \\
    &+\frac{1}{4}(\alpha|1\rangle+\beta|0\rangle)\left(\alpha^*\langle 1|+\beta^*\langle 0|\right) \\
    &+\frac{1}{4}(\alpha|1\rangle-\beta|0\rangle)\left(\alpha^*\langle 1|-\beta^*\langle 0|\right) \\
    &=\frac{1}{2}\left(|\alpha|^2+|\beta|^2\right)\ketbra{0}{0} + \frac{1}{2}\left(|\alpha|^2+|\beta|^2\right)\ketbra{1}{1} \\
    &=\frac{1}{2}\ketbra{0}{0}+\frac{1}{2}\ketbra{1}{1}\\
    \label{eq:Bobs_qubit_t2}
\end{aligned}
\end{equation}
In the penultimate line, we used the fact that Alice's state to be teleported is normalized, meaning $|\alpha|^2+|\beta|^2=1$.
Alice's measurement breaks the entanglement between her and Bob's qubit.
Yet, Bob's qubit is still in a maximally mixed state, meaning he has no information about the state \ket{\psi}.

Let's move to the later time $t_2$ when Alice sends her measurement outcome to Bob.
Nothing really changes on Bob's side compared to the situation a time $t_1$.
His qubit is still in a maximally mixed state described by Eq.~(\ref{eq:Bobs_qubit_t2}).
Bob still has no information about the state \ket{\psi} because the information sent by Alice about her measurement outcome requires a finite amount of time to travel to Bob.

Finally, at time $t_3$, Bob receives Alice's message about the measurement outcome.
This changes the state of his qubit to one of the four possibilities in Eq.~(\ref{eq:teleportation_possible_states}).
He applies the appropriate correction unitary ensuring his qubit is in the desired state \ket{\psi}.





\newpage
\begin{exercises}

\exer{
\emph{Filling in the gaps.}
During our discussion of the teleportation we have skipped some details, which we will consider now.
\subexer{
Consider the input state of the three qubits $A_1,A_2$, and $B$ in Eq.~(\ref{eq:teleportation_initial_state}).
Write down, both in Dirac and matrix notation, the initial state of each of the three qubits.
}
\subexer{
Write down Alice's measurement operators in the computational basis and in both Dirac and matrix notation.
}
\subexer{
Convince yourself that the probabilities of the four possible outcomes are independent of the initial state $A_1$.
}
}

\exer{
\emph{Teleporting the state \ket{0}.}
Let's work through an example of teleportation to reinforce our understanding of the protocol.
Rather than teleporting a general pure state \ket{\psi}, we wish to teleport the state \ket{0}.
\subexer{
Write down the initial state of all three qubits in the computational basis.
}
\subexer{
Rewrite the initial state, such that Alice's qubits $A_1$ and $A_2$ are written in the Bell basis, while Bob's qubit is in the computational basis.
}
\subexer{
How many possible outcomes of Alice's Bell-basis measurement are there?
What are their corresponding probabilities?
}
\subexer{
How many classical bits of information does Alice need to communicate to Bob to inform him of the measurement outcome?
}
\subexer{
What correction operations does Bob need to apply to ensure he receives the state \ket{0} after the teleportation protocol is finished?
}
}

\exer{
\emph{Teleporting a mixed state.}
So far we have only looked at the case when Alice wishes to teleport a pure state.
Let's teleport a mixed state, for example a state that is diagonal in the computational basis,
\begin{equation}
    \rho = a \ket{0}\bra{0} + b \ket{1}\bra{1}.
\end{equation}
\subexer{
Write down the initial state of all three qubits in the computational basis.
}
\subexer{
Compute the probabilities of Alice's measurement outcomes on the two qubits.
Did you expect your answer?
}
\subexer{
Find the state of Bob's qubit when Alice obtains \ket{\Phi^+} as the outcome of her measurement.
}
\subexer{
Find the state of Bob's qubit when Alice obtains \ket{\Phi^-} as the outcome of her measurement.
}
\subexer{
What are the correction operations that Bob needs to apply in order to successfully receive the state $\rho$, regardless of Alice's measurement outcome?
}
}

\exer{
\emph{Different initial entangled states.}
Up until now, we have consider the initial entangled state between Alice and Bob to be \ket{\Phi^+}.
Let's see what happens when we use the other Bell states.
\subexer{
Go through the teleportation protocol in order to find out Bob's correction operations in the following cases.
}
\subexer{
When the shared state is \ket{\Phi^-}.
}
\subexer{
When the shared state is \ket{\Psi^+}.
}
\subexer{
When the shared state is \ket{\Psi^-}.
}
\subexer{
Can you see a pattern?
}
}

\end{exercises}

