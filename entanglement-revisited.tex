\chapter{Entanglement Revisited}

\section{Bipartite entanglement}

00:00
hi and welcome to lesson 14 entanglement revisited here we will talk
a little bit more in depth about bipartite entanglement
we will also talk about multipartite entanglement so entanglement between
more than two parties and then we will conclude
with some applications of entanglement in quantum networks
step one bipartite entanglement so the one of the main jobs of quantum
network is to distribute entanglement so let's start by
considering a concrete example where we have the following networks
where these circles represent quantum nodes and these dashed lines represent
physical links between certain nodes of the network
and let's say that these green nodes are trying to engage
in a quantum communication so they would like to establish an entangled pair
between themselves so we're going to label them n1

00:01
and n2 and we saw in previous lessons that how this works in a quantum network
is at first we have to establish link level entanglement
so entanglement between neighboring networks that are connected directly by
via physical link and then we can use entanglement swapping and the intermediate
uh nodes in order to end up with a entangled connection between n1
and n2 and we also talked about that the network has to consider things
like routing and multiplexing in order to be satisfy
the need for quantum communication between multiple users of the quantum network
so but one question which we have not addressed in detail
is how do we evaluate the quality of the established
entanglement after all we saw that the quality directly impacts some of the

00:02
protocols so we saw that in entanglement-based qkd
the end quality of the entanglement that's shared between
alice and bob directly impacts the key that they're trying to
establish namely if the entanglement is not perfect
if they don't share a pure maximally entangled state
then the key that they are going to end up with after at the end of the protocol
will also be only partially correlated more crucially the uh
quality of the entanglement also tells us something about the security of the
entire protocol if they are not sharing a maximally
entangled state there is some chance that their state their qubit is
correlated with some qubit that alice has in her possession
which means that clever alice can find some information about the secret key
that alice and bob are trying to establish
so in this in this scenario we see that better quality of entangled states

00:03
lead to stronger security during establishing
during the establishment of a secret key so what are the different methods of
evaluating the quality of the distributed entanglement when we talk
about bipartite entanglement we saw that fidelity is one function
that we can use to talk about the quality of a state at the output either of a
quantum communication task or a quantum computation task
here we have some reference state which is a pure state that describes our ideal
state that we are aiming to get that's our desired state
but due to noise usually what we have is is some mixed state and if the noise
doesn't have much effect if it's very weak noise
this state will be close in some sense to the ideal state our reference states
psi and we can evaluate how close we are using the fidelity
which i remind you is computed in the following way

00:04
we have the fidelity f between the actual state rho
and our ideal state psi given simply as this inner product
in the context of quantum networks generally we are trying to establish
bipartite entanglement between end to end nodes
so we are mainly interested in the following expression and that's the fidelity
of the state that and two nodes are sharing
and the ideal state that they wish to share which is
a maximally entangled state in this case we will consider the phi plus state
and let's compute some examples so let's say that we have the ideal case
meaning there is no noise there's no eavesdroppers and the actual state
that the two parties are sharing is really the pure state
phi plus here written as a density matrix
well we can just simply substitute that into our expression for the fidelity

00:05
and we see that what we get is the square of the inner product between 5
plus and 5 plus which of course we have seen many times
is 1 because the state is normalized so good we get
fidelity of one meaning we have the ideal state now
what if we um end up with an orthogonal state
so maybe something has gone wrong um and due to some noise paulie z
matrix was applied to our ideal state and the two parties end up with the
following state it's still a pure state but it's not the desired status in this
case it's phi minus if we substitute that into our expression for the fidelity
we will see that fidelity drops to zero and that's because
uh our state that the two parties are sharing is orthogonal
with the desired state five plus but here's the problem we care that the
two states that the two parties alice and bob are entangled

00:06
and here they are clearly entangled they manage to share entanglement
it's not their desired entanglement it's not the desired belt state five plus
but it's still an entangled state nonetheless but if we just simply
evaluate the fidelity we get a zero meaning that our state is no good
and that's not quite what we're looking for so
we see that fidelity in this scenario is not a very good measure
so what can we do well we saw previously in the entanglement based qkd protocol
that we can test quantum correlations by violating the chsh inequality
to remind you we have considered the chsh inequality
and the state psi plus which we used in the e 91 protocol
and there alice measured in two different bases given
by these vertical and horizontal lines in the

00:07
exact plane of the block sphere so a1 corresponds to measurement of the pauli
z observable a2 corresponds to the measurement of the pauli x observable
and pop measures in this rotated bases b1 and b2 given by this following
expression so b1 is given by z minus x whereas b2 is given by z plus x
and alice and bob they have to repeat this process many times
perform these measurements build up statistics so that they can compute
their expectation values for the following measurement settings where
alice measures in a1 and bob measures in
b1 alex measures in a1 but both measures in b2
and so on and together we can form this following expression
known as the chsh expression and if this is larger than two then we
can conclude that the state that alice and bob
are sharing is really in fact entangled so okay that seems to work let's try and

00:08
use uh the chsh inequality for our state 5 that we are trying to verify
well we can again consider the same measurements and we construct the same chsh
expression if you go through the calculation for the expectation values
you can quickly find out that the expectation value for a1b1
for a is equal to a1 b2 and also a2b2 and it's given by 1 over square root of 2.
and the expectation value for a to b 1 is
negative 1 over square root of 2. so we can substitute
all these values back into our expression for the chsh expression
so here we have 1 over square root of 2 plus
1 over square root of 2 and we have here we have
minus 1 over square root of 2 because we are substituting this expression
and then minus 1 over square root of 2. so we get that s is equal to 0 and we

00:09
said that in order to certify that this the state is truly entangled it has to
be larger than 2. but phi plus is clearly entangled so
what went wrong well the answer is actually very simple
and it's the fact that this inequality has been constructed specifically for
the state psi plus but now we are trying to use it on the state phi plus so
in order to use chsh inequality as a test for entanglement we need to fix our
current expression for s luckily in this case it's very simple
all we have to do is just swap these signs this negative side negative sign
with a plus sign over here and with this new expression
we can actually find out that truly s is equal to 2 root 2 which is the
maximum violation of chsh inequality which always has to be true if they are

00:10
sharing maximally entangled state so we saw here that chsh inequality can detect
bipartite entanglement and to remind you if we go through the chsh test and we
obtain the value s larger than 2 then we have an entangled state
row if s is equal to 2 root 2 then we can say that the state that the
alice and bob are sharing is maximally entangled and if the
value of s is less or equal to 2 then the test is inconclusive
now notice that i have not said that the test does
means that the state is separable and that's because
there are some entangled states which do not violate this inequality
however if we find violation then we can
automatically conclude that the state is entangled
and also we have to be mindful of what a chsh expression we are using make sure

00:11
that it's the correct one make sure that you're using the right
combination of those four expectation values with the correct plus or minus
signs depending on which state you are trying to test is it five plus
five minus psi plus or psi minus you

\section{Multipartite entanglement}

00:00
step 2 multipartite entanglement in the previous step we have seen how
we can establish a bipartite entanglement that means entanglement only between
two nodes and between two qubits but of course we can consider also other
types of entanglement namely entangling between more than two nodes for example
in a network we we may wish to entangle three nodes like this
then we refer to to this case as tripartite entanglement
or it could be entanglement share between four nodes
so a four pie tart entanglement and we can go up and up
until we have entanglement between n a number of nodes or we have
multipartite entanglement so let's consider just three qubits
as starters how many basis states are there
we saw that for the case of two qubits there are four basis states zero zero

00:01
zero one one zero one one well you can guess that for the case of three qubits
there are eight basis states and they're
given by all these possibilities we have the three qubits being zero zero zero
zero zero one and so on and so forth all the way to one one
one and any general pure state can be written as a general superposition of
all of these eight basis states for example we can write any general pure
cubits three-qubit state as follows here as a superposition of all of these with
some arbitrary probability amplitudes given by c1 c2 all the way up to c8
now depending on the particular particularities of the superposition
that we take we may end up in a separable state or an entangled state
so let's consider some examples of entangled states
let's look at all of our possible 8 basis states

00:02
and take a very simple superposition of just the term 0 0 0 and 1
1 1. and to keep things simple we will consider this superposition to be an
equal superposition so we've got equal chance to be found in 0 0 0 and 1 1 1.
now this state is known as ghz state and plays a very important role in
quantum communication as we will see shortly
let's consider a measurement on one of the qubits of the three cubic zerj state
and let's say that we measure it in the z basis
and for concreteness let's just measure qubit one
so here's our three qubit g exit state it's an equal superposition of zero zero
zero one one one with probability of a half we can get the plus one outcome that
means we project the state of qubit one onto a zero and if that happens
we can see that we end up with the state of the three qubits collapsing

00:03
onto the following state all of them can be found in zero
zero zero so we can see that the qubits are correlated
the other possibility is that we measure the outcome minus one
this happens with equal probability of a half and then you can see from this
expression that the state of all three qubits collapses
onto one one one so again they are correlated and we saw similar
behavior when we were talking about establishing a secret
key in entanglement based qkd what we are looking for was a scenario
where performing measurements in the same basis can establish
establish a secret key between two parties here you can see that you can
do that uh do the same thing between three parties which is why
jirget states are used in quantum communication
in fact you can see that by doing that just like in the case for the
entanglement based qkd measurements consume the

00:04
entanglement that was shared between the three parties
so by measuring just one qubit in the um in the state means the total
uh entanglement is destroyed and in fact it doesn't matter which
qubit we measure in the z basis we would have ended up with the same
case with the same probabilities if we had measured in the qubit 2 or qubit
3. now let's move on to a different example known as a w
state again here are our eight possible basis states and this time
we'll choose a different superposition we will consider these three states
and we will take an equal superposition of these states
now the normalization factor is one over square root of three
because we have three states uh three basis states in the superposition
rather than two that we had in the previous example of the ghz state

00:05
and as i said this is known as the w state so how do measurements affect the w
state again let's measure in the z basis and
perform this measurement on qubit 1. what can happen is that with probability
two-thirds we get the outcome of plus one because we have
these two contributions from these two terms in the superposition
therefore the probability is equal to two-thirds
when that happens our first qubit is projected on
onto the state zero but our remaining two qubits qubits two and three
uh remain in an entangled state and this time they share
the maximally entangled bell pair denoted by phi plus
and the other outcome is the minus outcome of the z measurement on qubit one
which can happen with probability one third and if that happens
then we just end up in a separable state one zero zero
so we see that unlike in the case of a ghz state
here we don't necessarily always destroy the entanglement

00:06
that was initially present in the system and in this sense we can say that w
state and ghj state are not equivalent now
we have been talking about the various correlations that three parties can
share we said that ghz state if we measure in the z basis one of the qubits
all of the entanglement is destroyed whereas if we take the w state and we
measure we don't necessarily destroy all of the entanglement so somehow
this tells you that the correlations that are shared
with in these two states are not quite the same but there must be some
constraints on how correlated the various parties
of the state can be how correlated one and two are with three and so on
so here we have our three qubits qubits one two and three and they are in some
correlated state maybe in some entangled state

00:07
how strongly can various qubits be correlated well
that there is a constraint known as monogamy of entanglement which we have seen
uh uh in a previous lesson on the e91 protocol the entanglement with qkd
protocol and we see that there is a trade-off if
two pairs of the qubits are strongly correlated
then their correlation with the third party is limited
in particular if qubits 1 and 2 are in a maximally entangled state
then we know that the total state must be of this form
we have the maximally entangled state of qubits one and two
tens of product with some arbitrary single qubit state so we say that qubit 3 is
completely uncorrelated with the remainder of the state of
winning qubits 1 and 2. similarly if we have maximal
entanglement between 1 and 3 then qubit 2 must be completely uncorrelated

00:08
with the with the pair 1 and 3. this this principle of monogamy is one
of the most fundamental properties of quantum mechanics
it's unlike anything that exists in classical
classical physics and is one of the cornerstones and building blocks
which we use in quantum technologies and particularly in quantum communication
so we have seen examples of three qubit states can we extend the ghz state and w
state to many qubits let's say n qubits and of course
we can do that quite easily the gh state for n n parties or n qubits has the
following form again here we pick only two terms
one corresponding to n zero zero zero zero and the other one we have
n one so 1 1 1 1 1 1 1 and we take the equal superposition of them notice that

00:09
the normalization factor here has not changed even though we have n
n qubits that's because we only have two terms in the superposition
on the other hand the w state for n parties we have the following superposition
of n terms therefore the normalization factor must change to 1
over square root of n and what it is is really we have n
minus 1 0's and then 1 1 and we place this one at all the possible places
and we take the superposition of all the
possible cases where we have n minus one zeros and
a single one so we have n qubits and n terms
but this is not the only example of a useful entangled state of many qubits
there are many other examples but two are extremely important both in quantum
computation and quantum communication let's say that we have a network of

00:10
qubits and we prepare every qubit in the state plus so this is
an equal superposition of 1 and 0. and then we apply an
entangling gate known as a control z gate in the matrix form it looks
something like this so it's very similar to a c not gate a control not gate
but this time instead of applying the pauli x
in a controlled way we are applying the pauli z in the controlled way
so if the control qubit is in the state zero we don't do anything
if it's in a state one we apply the pauli z matrix
and this gate is entangling and entangles all of the qubits together into into a
multipartite entangled state and this is the preparation procedure
for a state called a graph state normally denoted by ket
g now if the state has a regular lattice structure
we can still refer to as a graph state but commonly is referred to as

00:11
a cluster state and denoted by ket c and by regular structure i mean
the following the following topology like this
now graph states and cluster states are very important in quantum computation
they form the resource states for a particular
computational model known as measurement based quantum computation
but they are also very useful in content communication where
multiple parties are trying to communicate and they are crucial in quantum
error correction as well which will which you will see in a different module

\section{Clock synchronization}

00:00
step three clock synchronization so we have seen in this module the
applications of uh how we can use entanglement we saw the
example of teleportation and we saw the example of entanglement
based qkd here in the remaining two steps of this lesson we will consider a
few other examples of applying entanglement and in
here in step three we will look at clock synchronization
so before we say how we can synchronize clocks
let's ask the question why we want to synchronize clocks
well establishing universal time standard is fundamentally important
in many areas of modern life the telecommunications networks require
synchronized clocks global positioning system financial
markets transportation networks these are all just a few examples of
crucial importance to modern way of living
for example if we look at the gps how it works

00:01
how it uh calculates the exact the very accurate position where we are is
it computes the distances to four uh satellites
and then from that it can locate with very high degree of accuracy
where we are placed on earth and in order to calculate this distance is
this r1 r2 r3 and r4 accurate timing is very important because even tiny errors
in in timing can result in huge positional errors
so how can we synchronize clocks well there are really two classical methods
and one quantum method which we will mention
the first one is due to einstein imagine that we have two clocks
one is in possession of alice and the other one is in possession bob
and they are trying to synchronize alice is trying to synchronize
her clock with bob's clock so what she can do is she can

00:02
fire a pulse of light towards bob so the light is traveling to bob and it
reaches bob as soon as it reaches his clock it bounces back and as it's
doing that it sets the bob's clock ticking alice is measuring the time that it
takes for the photon for the light pulse to travel from her to bob and bounce
back from that she knows uh she can estimate
the distance to bob she calculate the distance to bob
and also she will know what's the round trip time and also what's the
time in order for the photon to reach from bob to alice
therefore she knows how she can set her clock
and this will synchronize the two clocks together
however in this scheme alice needs to know where bob is located
in order to correctly synchronize their clocks
a different scheme is uh to for alice to have some other
smaller clock which she synchronizes locally
with her clock and then the small clock is

00:03
very slowly in fact adiabatically slowly
transferred to bob where he receives the clock and then he can locally
synchronize his clock with this received clock and this scheme has to be done
adiabatically slowly in order because of a theory of relativity
now the third scheme is realizing that qubits can act as
tiny clocks how does that work well qubits they evolved in time
for example if we prepare a qubit in an equal position superposition of 0 and 1
we can make it change its state in time such that it goes around this
x y plane in the block sphere with angular frequency capital omega
so it starts we can initialize it in the state plus
and after some time capital t which is the period of precession

00:04
it completes one full circle so it goes around an angle of 2 pi
radians so the period of precession is given by 2 pi
divided by the angular frequency of precession
so really in this sense it's it's the same as a grandfather clock
if we can track how many times it goes around the x y plane
we can track time knowing the the angular frequency of precession
so the question now is how can we synchronize two qubits
let's say alice has qubit 1 that is processing at a frequency a capital omega
and bob also has one that's also processing at the same frequency
but this time there is some offset so it's lagging behind alice's
cubit clock so this uh this delta is quantifying
the lag between in terms of the angle between the point that's on the on the

00:05
block sphere in the xy plane for bob and for
alice and what we are trying to do is we are trying to eliminate
this uh offset delta in order for their clocks to be synchronized and show
exactly the same time so where is the main problem well the
problem is that alice has her local time frame and bob has his
own local time frame and somehow they need to communicate like say my time
frame is this what is your time frame and they have to
exchange messages in order to agree on a global time frame
in fact what they can do is they can do that by sharing
entangled pairs of qubits they can use these qubits
to establish a global global time frame so in that sense entanglement is used as
a global resource and we have seen the similar scenario
also in the entanglement based qkd there we use entanglement and classical

00:06
communication to correlate to in order to establish a correlated secret key
between alice and bob that was the e 91 protocol
in clock synchronization of qubits we are using
the global correlations that are present in a maximally entangled state
in order to establish a global tie frame and correlate
these two qubit clocks together therefore synchronizing them
and quantum networks are instrumental in distributing bipartite entanglement
therefore allowing the various nodes in the network to
synchronize their clocks and establish a global global time frame

\section{Distributed and blind quantum computation}


00:00
step 4 distributed and blind quantum computation
this will be our final example of the application of entanglement
and quantum network in this lesson so let's talk about distributed quantum
computation first there is no reason for alice to perform
all of her quantum computation locally let's say that her resources are limited
maybe she doesn't have a the large number of qubits or the cubits that she has
are not sufficient of sufficient quality but that doesn't mean that she cannot
perform her quantum computation what she can do is she can contact her
friends bob charlie dave and eve who are also in possession of some
limited computational resources they also have some small
quantum computers and together they can coordinate
their resources and perform a larger quantum computation

00:01
um compared to if alice did it only locally
so here alice bob charlie dave and eve they all share some number small number
of qubits and they can entangle in some them in
some way and also they can exchange quantum information
uh by using the quantum network or they can also exchange some entangled qubits
they can connect those local uh cluster states
together in some larger cluster states and then perform computation on that and
there are many questions of interest such as what is the most efficient way
of networking the quantum processors together how many messages do they need to
exchange in order to coordinate their computational efforts
and also issues like trust how can alice trust maybe she doesn't trust
all of the parties involved in the quantum computation what does she need
to do how can net quantum network help her in this situation

00:02
also here we are assuming that she has some quantum
resources what if her quantum resources are extremely limited
so what do we mean by extremely limited let's say that she cannot do anything
she doesn't have a quantum computer but she would still like to delegate her
quantum computation and for simplicity let's just assume
that bob has a full-fledged very powerful large quantum computer where he can
perform any quantum computation that he wishes or that alice would ask him to do
so in this scenario what she can just do is she can send him
a classical message composed of many bits describing to bob
exactly what computation she wishes him to perform
so she needs to describe the input of the computation
the computation itself and then the bot will
just take this information and carry out the computation and then
tell alice the output the result of the computation

00:03
but in the process what happens is he learns everything there is to know about
the computation he learns the input he learns the computation itself
and he learns the output now what if alice doesn't want bob to learn
some of this information or any of this information
maybe she's trying to gain a competitive edge by performing some simulation of a
molecule or a new material and she doesn't want
bob to find it out because it's a trade secret
can she can she still do something in this case
well she can if she's allowed to have some quantum resources
and particularly we assume that alice has the ability to generate single qubit
states so what she can do is she generates these states
and then she can randomly rotate them then she takes these keywords and she
sends those to quantum uh those to bob so she's using a quantum

00:04
channel to communicate with bob but these uh these qubits are not
entangled there's really just a bunch of single qubit states randomly uh rotated
and then she sends him classical uh instructions about the computation
so she directs bob take qubit 1 perform this operation take ub2 perform that
operation take these two qubits and entangle them using this gate and so on
and in this way bob can perform whatever she is instructing him to do and then
just return the final outcome he communicates either
in quantum way or classically to alice about the result of the computation
and in this way bob does not know the random rotations
because those are still kept secret by alice so this prevents him from learning
anything about the input anything about the computation itself and also
anything about the output he's basically doing this computation

00:05
blind he's performing some operations according to alice's instructions
but because he doesn't know the initial states of the keywords
he doesn't really know what these uh operations
mean he cannot interpret them therefore whatever he gets he just performs the
in the operations blindly he gets some output
but he cannot uh interpret this output he just returns it to alice
the only thing that he can learn is the upper limit to alice's computation
for example if she sends him four qubits then he knows that she alice
could not have performed any computation that requires more than four qubits


\newpage
\begin{exercises}
\exer{Consider the following quantum state:}
\begin{equation*}
\ket{\psi} = \frac{\sqrt{3}}{2}\ket{0} + \frac{1}{2}\ket{1}
\end{equation*}
\subexer{Find the probability of measuring a zero.}
\subexer{Find the probability of measuring a one.}


\end{exercises}

