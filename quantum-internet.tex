\chapter{Quantum Internet}

\section{Networks of networks}

\rdv{Missing subtitles!}

\section{Integration with classical systems}
\label{sec:classical-integration}

00:00
step two integration with classical systems rod the title of these steps is
integration with classical systems does that mean that the quantum internet
is not going to replace the classical internet
that's a good question but yes the the quantum internet will not replace the
classical internet there will be two networks right so how are they going to
talk to each other how are they going to integrate together
let's see so on this diagram here we've got what says quantum network and an ip
network what's the quantum network do well we've seen that in a number of
lessons here the main purpose is to distribute
entanglement between parties whether it be bipartite entanglement multipartite
entanglement so that it can be used as a resource for
communication tasks such as teleportation
and qkd whether it's entanglement based or single photon based

00:01
and qkd is in fact the example that's written here on on the on the diagram so
the qkd devices what are they doing across across the quantum network remind
us what's what's the service they provide well the
thing that they're trying to do is they're trying to establish a secret
correlated key between one communicating party and the other one
without any eavesdropper knowing what the key is right so what's a key in this
in this context the key is the key is a secret secret a bit of strings
classical bit of strings that we can use to hide our message
all right so it's a string of random bits that are guaranteed to be
secret based on on what we've done so far so what we want to do once we've got
those is to use them ah they're classical bits yes they're classical bits yes
no no no no no okay good that's why we can use them with the classical internet
yes so in fact that's what we're going to do next is we're going to actually
use them with with the classical internet so the first job that happens

00:02
is the quantum network makes those keys and it makes those classical
bits out of it and then it's going to share those bits
with a couple of boxes that are connected to an ip network
that are called ipsec gateways and there will be one of those at each end
have you heard of ipsec not really no okay so ipsec is one of the standard
internet protocols for how you encrypt data that you're going to share across
across the network one of the other famous protocols is a
protocol that's called tls which is used for web browsing and for a lot of other
things ipsec is actually slightly older than tls and
its original design was to connect one network to another network
securely and so the the you send data to your eye to
your nearby ipsec gateway and it encrypts the data and sends it to
another ipsec gateway at the other end where it gets decrypted and sent to
your partner over there so it's a standard protocol for doing that

00:03
now so once we've got the keys to the ipsec gateways
before qkd the way that these ipsec gateways work together is they have to
negotiate a a key exchange of some sort so we talked
in the encryption lesson about three phases of an encrypted
conversation right there was a authentication
key generation and then the bulk data encryption right so
the uh the authentication is often done using public key
the key generation prior to qkd was primarily done using a mechanism called
diffie-hellman key exchange so that's what we're going to replace with
the qkd and then that data will be used to actually
that key that comes out of the qkd network or the quantum internet and the qkd
devices attached to it will be used to create the keys that are used by the
ipsec gateways for encrypting large amounts of data which gets sent

00:04
using an encryption mechanism well most commonly
these days an encryption mechanism called aes the advanced encryption standard
so that's so you get this encrypted connection um and
yeah it's important to note here that the qkd connection
itself actually requires that you also have an authenticated classical channel
between the nodes in order to prevent someone
standing in the middle of the network and pretending to be
alice in one direction and pretending to be bob in the other direction
and that's what's called a man in the middle attack and we want to try to get
rid of try to avoid having that happen i see
so let me get this straight the role of the quantum network really is just
one step in the whole number of steps uh during a secret communication exactly
so we start classically we authenticate classically
but then when we require the generation of the key
that's where the quantum part comes in that's where the quantum magic happens

00:05
either through no non-fully distinguishable states such
such as we saw in the bb-84 protocol or entanglement based quantum key
distribution using the e91 protocol right and whatever the result that of
that is it should be a secret correlated key
that's not known to any other malicious party which then gets passed back
into the classical network and we proceed classically again exactly
good and so all of this put together makes for an integrated quantum and
classical system so you're the expert on a couple of
these other things that we might want to do
right oh yeah yeah we've we've talked about
distributed quantum computation where uh alice bob eve dave and charlie they
share some quantum resources but they're individually they cannot
perform the computation that they want to they don't
have enough quantum power so they have to figure out how to share this quantum

00:06
power with each other and to create something that allows them then to
tackle much larger quantum problems so for that again they have to share
some quantum resources that's done by the quantum network but
then they have to coordinate with classical messages using the classical network
what they need to perform they need to call they basically use the classical
control of the classical provided by the classical network to
then perform the quantum computation on the quantum part of their memories
their processors and so on so in that in that case this network of
quantum computers that's going to do this distributed computation
is that distributed computation is that a complete application is that like
quantum photoshop or or quantum oracle database or some something or
well i would say never say never but for me but for now the the problems that we
envision tackling with quantum stuff are things that are genuinely quantum where

00:07
classical problem classical computers can help but are too slow
for example diagonalizing large matrices finding new properties of new drugs new
materials and so on i don't think that there will
be an application of a quantum photoshop where we can
where we can i don't even know how that would work in reality i don't either so
the chemistry one's a really good example because because they're
as i understand it there will be some classical pre-processing and then
we'll do a quantum step and then after you get back your quantum answers
then there's going to be some more classical computing to do something
that's true that's true that's known as the vqe or the variation
quantum eigensolver where really again the the quantum part
in the entire computation is just one tiny step in a series of steps
but it's a very important one that's particularly slow using only classical
computation and classical devices okay and so all of that will fit together

00:08
into a distributed and integrated quantum and classical that's right
system the information system that other people will then use that's right
okay what's the other one well the other one is our
clock synchronization where we mentioned that
having a global time standard where everybody agrees on the same time is of
crucial importance in many many areas of our modern life and again
it's not that we want to do something special with quantum time
the time is classical but the way on how we agree
on the same global standard of time comes through using quantum resources
such as entanglements such as bipartite entanglement and so on
okay and then once you've got that then it gets used in
it goes back it gets bumped back into the classical network and it's very
important for transportation for gps so global positioning system financial
market where even fractions of a second are crucial
uh they can either gain you millions or you can lose millions

00:09
and and so on and so forth so time time is very important and keeping
keeping uh the same time as somebody on the other side of the world
is very very important as well right so all of this will fit together
so there's going to be a quantum network and there's going to be some ip network
you know the existing classical internet or what have you
and these two things are going to have to come together in order to build these
complete integrated hybrid quantum classical distributed systems
synchronized synchronized synchronous that's an important point too
okay so let's see in the next step how we can bring these things together
all right so with standardization right the first thing you're going to do of
course is you're going to actually build and test some
sort of system you've got a prototype that's up and running right
um and then what well then once that is working we would like everybody else to

00:10
use it yeah and in communications one of the things that means
is that it's really nice if my communications device will talk to your
communications yes even though i was the one who made
my device and you were the one who created your device
maybe using completely different means so
how do we standardize these things well there are a lot of meetings and things
that go on in making all this happen but in particular for a concrete example
we might have to worry about the physical layer and the protocol
layer and those might actually be standardized
by different organizations that are actually involved in this so
at the physical layer so i know how you might do that with like you know
electrical signals for the ethernet or something
and the organizations will define voltage levels and how long signals
are allowed for and one of the things you talked about
earlier was distortion of signals due to modal dispersion we were talking about
that in terms of optical fibers right but that same kind
of phenomenon happens with any kind of signal and so one of the things you have

00:11
to worry about is how much of that is allowed so that you
can still guarantee that the end nodes will be able to reconstruct things
so all that's for for classical signals what are sort of the equivalence of that
for photons well we said that in quantum communication we're
interested in exchanging information via single photons so we have to think
about how do we get two different devices to talk
particularly at the border of two networks so they will have to exchange
photons so how can we ensure that whatever photons i provide
to the other network it gets accepted recognized and vice versa
so i imagine things that will be important will be the wavelength
of the photons with the shape of the wave packet of the photons
and basically ensuring that if we are trying to establish an entangled
link between two networks we saw that that happens with the bsa

00:12
right and we mentioned we stressed that the two photons that are coming in
in order for them to interfere they must be indistinguishable
so somehow we must have a procedure and a standard procedure that makes sure
that whatever photons you are sending me are the same photons as i'm sending to
you and then they can hit each other interfere at the bsa and create
entanglement between us and between the two networks that we are part of
sounds good to me so the kinds of organizations that do that
are organizations that like um the ietf and the ieee and enc
and organizations like that they'll be involved in all of this at some point too

\section{Putting it all together}

00:00
step three putting it all together right so let's put all of these things
that we learned in this module into some context
all right but before we do that let's say this is the last video of this entire
module it's not the last step there are a couple of other things after this but
we should thank everybody who's helped us put all of this together so we should
thank you know husnie and we should thank the editors uh
and we should thank all of the students who have done all of the work on all of
this without you guys it wouldn't have happened right okay thank you very
very much so what else let's see so just to go all the way back to the
core idea here there are four things that a repeater
has to do and then once you have repeaters
you can build a network up right so the first thing
well establish entanglement between two neighboring nodes so
link level entanglement good that involves handling loss of photons

00:01
oh yeah second thing we extend this from neighboring nodes to nodes which
are separated by larger distance by multiple hops so we
establish end-to-end entanglement the primary mechanism we used for doing that
was entanglement swapping that's right then we have to think about how do we
handle state and operation errors so we saw a few examples which we used
in our calculations of unitary errors but there's also other types of errors
and the main protocol that we used was the purification
all right and then the last thing and the last thing we have to think more
uh in terms of networking and how to handle things like
routing uh how to handle things like multiplexing
when there's uh contention for resources and how to manage all of these things
things including security so that's things like the routing protocol
that we talked about in the multi-level system and the internet and things like

00:02
that in this earlier in this lesson but also a lot of that back in lesson
12 i think possibly all right so all of those are technical
requirements for how you go about building boxes and an internet but
people who are taking this module they're people they're people
right and so people are the ones who design and build and operate these networks
so this is the first module in what will be a pretty
extensive and pretty in-depth sequence on quantum communications and quantum
computation you can choose sort of your specialty
within that area but primarily here we're focusing on the quantum communications
and they're going to be a whole lot of jobs that are that are possible a whole
lot of career paths for you to go forward with this
as you complete this entire set of modules and complete a degree
or just even for your own purposes so so what are these specializations

00:03
well i'm an architect so i list architecture first
that involves defining the subsystems within a larger system and defining the
relationships so you define block a and block b and what's the
relationship between the two what's the contractual agreement between
them what messages do they exchange what uh behavior are expected of the
different subsystems and with all of that that allows you to
make forward progress by dividing the problem a large problem into a set of
smaller problems and then each of those can be worked on
to a certain extent independently protocols so the protocols define
the messages that get exchanged on the wire and the behavior what you do when
you get a message but also what you do when you don't get a message and sort of
other sets of rules you have to follow what sort of promises you make as
participating in in a particular network communication for example

00:04
as we saw just recently we talked about standardization right
hardware oh hardware is very important that's
closer to my heart the physicist but hardware are the
physical things that basically allow all of these more abstract things such as
architecture and protocols to work so we need to design the hardware we need to
analyze it we have to ensure that really our boxes are quantum boxes
our memories that are sitting at quantum notes they are distributing
entanglement if something goes wrong what do we do and uh uh
how do we test things and how do we build them
yeah so all of that everything we're talking about here is all
super critical you have to have all of these things in order to build a network
but fundamentally if the hardware doesn't work we're just toast right off
the bat right so all right software you got to have software too
right and that some of that is software that's internal to the boxes you're
building and so nobody cares if you change that as long as they

00:05
what they care about is the external behavior of the box but software also
includes implementation of those protocols that we talked about earlier
right and that building hardware is hard right yep building software
people think it's easy but it doesn't happen automatically
yeah yeah and it's also the case that software can be iterated more quickly
than hardware can you can make small changes to things in in
software and redeploy hardware redeploy software fairly quickly
but where software meets protocols once the protocols are defined it starts
to get to be a lot harder to change and even that software that you're
actually iterating quickly even getting that right also takes a lot of work
and so you have to have the right balance between hardware and software
although it often seems like it's easier to get
software up and running quickly and easily building a complete and robust
system that does everything that's expected and nothing it's not expected to do

00:06
that's all hard work too right we also have to have operations and
management of the networks so once all these boxes are built you know
we talked about the routing protocols and talked about all these other things
somebody's job is going to be to order boxes from the companies that provide
this stuff and bring them into your building and unbox
them and set them up and turn them on and run them
and make sure that they connect to the local systems and that everything works
and that keep track of what's working what's not all of that's operations and
maintenance and that's an important career path
in the classical internet as well next one is education and community
exactly what we're doing now we are educating people and we're
participating in the community yes so we're both researchers and educators
but this applies not only at the university level where we are right now
but it also includes high school and and also post-secondary
education so it can be educating the public as well as educating people
within a formal school context um so all that's really important

00:07
so some of you could wind up as high school teachers and that's a perfectly fine
career path for going forward because i'm certain that there are going to be
quantum classes in high schools in the future and somebody's got to be
trained the people who teach those one other aspect
of this is where we included a community is there's also a lot of discussion
these days about the ethics of artificial intelligence or ai
there is just beginning to be a conversation about the ethics of of quantum
as well but there's also relationships with how all this is going to fit into
society and legal things as well because in particular quantum networks
and quantum computers both have a significant impact on
cryptography which is considered to be a sensitive issue
it seems like a very disruptive technology anything quantum whether it's
communication or computation oh yeah yeah and so businesses and governments
both care about this and how it's going to affect

00:08
their their own operations as well as their own societies
and then theory that would be you wouldn't that would be me yes
there's always theories needed particularly when it comes to quantum
computation and quantum technologies the the things that are made quite
important are information theory so the the formal mathematical theory of
how to process information how to distribute information
how to communicate information uh these are these are all included under under
the theory and it doesn't need to be quantum it can be also classical uh but
quantum is the the more fun one more more surprising one
and the design of new algorithms of course we we
always want to know how to do things better or how to do things which we
haven't thought of and quantum is the perfect playground for
that there there's a truly a large opportunity for creativity to kind of like
uh let it run free and and see what uh amazing new algorithms and applications

00:09
we can we can come up with so all of this um there's it's a
fantastic and exciting time to be in quantum whether it's quantum networking
or quantum communication i've been doing this since 2003
prior to that i was doing primarily classical systems
and you know every year it gets to be more and more
more and more exciting and more and more real particularly with the
recent developments in by ibm and google and other new startup
companies like ironco inq psi quantum it's a truly
truly mesmerizing time to be in quantum whether it's computation or
communication and now it's a great time for you all to
be quantum natives and to come to come join join
all of us in the process of this so thank you all for participating in this
module and we'll see you again see you bye


\newpage
\begin{exercises}
\exer{Consider the following quantum state:}
\begin{equation*}
\ket{\psi} = \frac{\sqrt{3}}{2}\ket{0} + \frac{1}{2}\ket{1}
\end{equation*}
\subexer{Find the probability of measuring a zero.}
\subexer{Find the probability of measuring a one.}


\end{exercises}


\newpage
\section{Quiz}

% \section{Learning more}

\section{Further reading Lessons 14-15}

Lesson 14

For those interested in how GHZ state can be produced with linear optics we recommend:

Dirk Bouwmeester, Jian-Wei Pan, Matthew Daniell, Harald Weinfurter, Anton Zeilinger, Observation of three-photon Greenberger-Horne-Zeilinger entanglement, Physical Review Letters 82, 1345 (1999).

Freely accessible version of this paper can be found here.

Extension of the Bell inequality to three particles and its experimental violation is reported here:

Jian-Wei Pan, Dirk Bouwmeester, Matthew Daniell, Harald Weinfurter, Anton Zeilinger, Experimental test of quantum nonlocality in three-photon Greenberger-Horne-Zeilinger entanglement, Nature 403, 515 (2000).

This paper can be accessed freely here.

Lesson 15

For discussion of the physical layer aspect of the Quantum Internet we recommend:

H. Jeff Kimble, The quantum internet, Nature 453, 1023 (2008).

Freely accessible version can be found here.

Kimble’s paper focuses on the hardware and does not discuss any of the networking aspects. For that we recommend Van Meter’s textbook:

Rodney Van Meter, Quantum Networking, Wiley-ISTE, 2014.
